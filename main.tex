%%
%% Copyright (c) 2017-2018 The Authors.  All Rights Reserved.
%%
%% Weitian LI, et al.
%% School of Physics and Astronomy, Shanghai Jiao Tong University, Shanghai, China.
%%
%% 2017-07-18
%%

\documentclass[modern]{aastex62}
%\documentclass[twocolumn]{aastex62}
%%
%% Defaults: single spaced, 10 point font article
%%
%%  twocolumn   : two text columns, 10 point font, single spaced article.
%%                This is the most compact and represent the final published
%%                derived PDF copy of the accepted manuscript from the publisher
%%  manuscript  : one text column, 12 point font, double spaced article.
%%  preprint    : one text column, 12 point font, single spaced article.
%%  preprint2   : two text columns, 12 point font, single spaced article.
%%  modern      : a stylish, single text column, 12 point font, article with
%%                wider left and right margins. This uses the Daniel
%%                Foreman-Mackey and David Hogg design.
%%
%% Note that you can submit to the AAS Journals in any of these 6 styles.
%%
%% There are other optional arguments one can envoke to allow other stylistic
%% actions. The available options are:
%%
%%  astrosymb    : Loads Astrosymb font and define \astrocommands.
%%  tighten      : Makes baselineskip slightly smaller, only works with
%%                 the twocolumn substyle.
%%  times        : uses times font instead of the default
%%  linenumbers  : turn on lineno package.
%%  trackchanges : required to see the revision mark up and print its output
%%  longauthor   : Do not use the more compressed footnote style (default) for
%%                 the author/collaboration/affiliations. Instead print all
%%                 affiliation information after each name. Creates a much
%%                 long author list but may be desirable for short author papers

%% Extra packages
\usepackage{amsmath}
\usepackage{csquotes}  % `enquote'
\usepackage{bm}  % bold math
\usepackage{savesym}

\savesymbol{tablenum}  % workaround symbol conflict
                       % credit: https://tex.stackexchange.com/a/269074
\usepackage{siunitx}  % typeset units; from `texlive-science'
\restoresymbol{SI}{tablenum}  % `tablenum' from `siunitx' becomes `SItablenum'

%% Chinese
\usepackage{xeCJK}
\setCJKmainfont{SimSun}
\setCJKsansfont{KaiTi}
\setCJKmonofont{KaiTi}

%% Bibliography style
\bibliographystyle{aasjournal-links}

%% Custom macros
\newcommand{\R}[1]{\mathrm{#1}}
\newcommand{\D}[1]{\R{d} #1}
\newcommand{\diff}[2]{\frac{\D{#1}}{\D{#2}}}
\newcommand{\pdiff}[2]{\frac{\partial #1}{\partial #2}}
\newcommand{\lcdm}{$\Lambda$CDM}
\newcommand{\Halpha}{\text{H$\alpha$}}
\newcommand{\Hi}{H\textsc{i}}
\newcommand{\Hii}{H\textsc{ii}}
\newcommand{\klos}{\text{$k_{\parallel}$}}
\newcommand{\kperp}{\text{$k_{\bot}$}}
\newcommand{\fov}{\text{Fo\!V}}

%% Marks
\newcommand{\NOTE}[1]{\textcolor{violet}{[\textbf{NOTE:}~\uwave{#1}]}}
\newcommand{\TODO}[1]{\textcolor{magenta}{[\textbf{TODO:}~\uuline{#1}]}}
\newcommand{\OK}[1]{\textcolor{cyan}{[\textbf{OK:}~\uline{#1}]}}
\newcommand{\XU}[1]{\textcolor{blue}{[\textbf{XU:}~\uline{#1}]}}
\newcommand{\LI}[1]{\textcolor{purple}{[\textbf{LI:}~\uline{#1}]}}

%% Custom math operators (requires `amsmath' package)
\DeclareMathOperator{\erf}{erf}
\DeclareMathOperator{\erfc}{erfc}

%% Settings for package `siunitx'
% credit: https://tex.stackexchange.com/a/194042
\sisetup{
  range-phrase=\text{--},
  range-units=single,
  product-units=power
}
%
%% Custom units (requires `siunitx' package)
\DeclareSIUnit\arcsec{arcsec}
\DeclareSIUnit\arcmin{arcmin}
\DeclareSIUnit\cMpc{cMpc}  % comoving Mpc
\DeclareSIUnit\cGpc{cGpc}  % comoving Gpc
\DeclareSIUnit\deg{deg}
\DeclareSIUnit\erg{erg}
\DeclareSIUnit\gauss{G}
\DeclareSIUnit\hour{hr}  % overwrite hour from 'h' to 'hr'
\DeclareSIUnit\hubble{\text{$h$}}
\DeclareSIUnit\jansky{Jy}
\DeclareSIUnit\lightyear{ly}
\DeclareSIUnit\parsec{pc}
\DeclareSIUnit\rayleigh{Rayleigh}
\DeclareSIUnit\solarmass{\text{M$_{\odot}$}}
\DeclareSIUnit\year{yr}
%
\DeclareSIUnit\keV{\kilo\electronvolt}
\DeclareSIUnit\kpc{\kilo\parsec}
\DeclareSIUnit\mJy{\milli\jansky}
\DeclareSIUnit\mK{\milli\kelvin}
\DeclareSIUnit\uG{\micro\gauss}
\DeclareSIUnit\Gyr{\giga\year}
\DeclareSIUnit\MHz{\mega\hertz}
\DeclareSIUnit\Mpc{\mega\parsec}
\DeclareSIUnit\Gpc{\giga\parsec}

%% Hyperref-related settings
\def\sectionautorefname{Section}
\def\subsectionautorefname{Section}
\def\subsubsectionautorefname{Section}
\def\figureautorefname{Figure}
\def\tableautorefname{Table}
% Credit: https://tex.stackexchange.com/a/66150
\def\equationautorefname~#1\null{Equation~(#1)\null}

%% Custom subsubsubsection ...
\newcounter{sssseccount}
\newcommand{\sssseclabel}{\alph{sssseccount}}
\newcommand{\ssssec}[1]{%
  \vspace{1ex}%
  \stepcounter{sssseccount}%
  \noindent\emph{\sssseclabel. #1}%
}

%% Custom settings
\setlength{\parindent}{1.5em}
\graphicspath{{figures/}}  % NOTE: the trailing '/' matters

%% Journal information
\received{XXX, 2018}
\revised{XXX, 2018}
\accepted{XXX, 2018}
\submitjournal{ApJ}

%% Add a light gray and diagonal water-mark to the first page
\watermark{DRAFT}  % watermark text: DRAFT
%% Control the water-mark size with:
%% \setwatermarkfontsize{dimension}

%% Short title to be used as the running head (<= 44 characters)
%\shorttitle{Impacts of Radio Halos on EoR Detection}
\shorttitle{Low-frequency Contributions of Radio Halos}
\shortauthors{Li~\textit{et~al.}}


%%%%%%%%%%%%%%%%%%%%%%%%%%%%%%%%%%%%%%%%%%%%%%%%%%%%%%%%%%%%%%%%%%%%%%%%%

\begin{document}

\title{%
  Radio Halos in Galaxy Clusters and Their Contributions
  to the Low-Frequency Radio Sky
}

%% The \author command is the same as before except it now takes an optional
%% argument which is the 16 digit ORCID.  The syntax is:
%% \author[xxxx-xxxx-xxxx-xxxx]{Author Name}
%%
%% This will hyperlink the author name to the author's ORCID page. Note that
%% during compilation, LaTeX will do some limited checking of the format of
%% the ID to make sure it is valid.
%%
%% Use \affiliation for affiliation information.  Please use multiple
%% \affiliation calls for to document more than one affiliation.  AASTeX v6.1
%% will automatically index these in the header.  When a duplicate is found
%% its index will be the same as its previous entry.
%%
%% The new \altaffiliation can be used to indicate some secondary information
%% such as fellowships. This command produces a non-numeric footnote that is
%% set away from the numeric \affiliation footnotes.  NOTE that if an
%% \altaffiliation command is used it must come BEFORE the \affiliation call,
%% right after the \author command, in order to place the footnotes in
%% the proper location.
%%
%% Use \email to set provide email addresses. Each \email will appear on its
%% own line so you can put multiple email address in one \email call.
%%
%% A new \correspondingauthor command is available in V6.1 to identify the
%% corresponding author of the manuscript.  It is the author's responsibility
%% to make sure this name is also in the author list.
%%
%% While authors can be grouped inside the same \author and \affiliation
%% commands it is better to have a single author for each.  This allows for
%% one to exploit all the new benefits and should make book-keeping easier.
%%
%% If done correctly the peer review system will be able to automatically
%% put the author and affiliation information from the manuscript and save
%% the corresponding author the trouble of entering it by hand.

\correspondingauthor{Haiguang Xu}
\email{hgxu@sjtu.edu.cn}

\author[0000-0002-7527-380X]{Weitian Li}
\email{liweitianux@sjtu.edu.cn, liweitianux@live.com}
\affiliation{School of Physics and Astronomy,
  Shanghai Jiao Tong University,
  800 Dongchuan Road, Shanghai 200240, China}

\author{Haiguang Xu}
\affiliation{School of Physics and Astronomy,
  Shanghai Jiao Tong University,
  800 Dongchuan Road, Shanghai 200240, China}
\affiliation{Tsung-Dao Lee Institute,
  Shanghai Jiao Tong University,
  800 Dongchuan Road, Shanghai 200240, China}
\affiliation{IFSA Collaborative Innovation Center,
  Shanghai Jiao Tong University,
  800 Dongchuan Road, Shanghai 200240, China}

\author[0000-0003-1263-9453]{Zhixian Ma}
\affiliation{Department of Electronic Engineering,
  Shanghai Jiao Tong University,
  800 Dongchuan Road, Shanghai 200240, China}
\author{Dan Hu}
\affiliation{School of Physics and Astronomy,
  Shanghai Jiao Tong University,
  800 Dongchuan Road, Shanghai 200240, China}
\author{Zhenghao Zhu}
\affiliation{School of Physics and Astronomy,
  Shanghai Jiao Tong University,
  800 Dongchuan Road, Shanghai 200240, China}
\author{Chenxi Shan}
\affiliation{School of Physics and Astronomy,
  Shanghai Jiao Tong University,
  800 Dongchuan Road, Shanghai 200240, China}
\author{Jingying Wang}
\affiliation{Department of Physics and Astronomy,
  University of the Western Cape,
  Cape Town 7535, South Africa}
\author{Junhua Gu}
\affiliation{National Astronomical Observatories,
  Chinese Academy of Sciences,
  20A Datun Road, Beijing 100012, China}
\author{Xiaoli Lian}
\affiliation{School of Physics and Astronomy,
  Shanghai Jiao Tong University,
  800 Dongchuan Road, Shanghai 200240, China}
\author{Qian Zheng}
\affiliation{Shanghai Astronomical Observatory,
  Chinese Academy of Sciences,
  80 Nandan Road, Shanghai 200030, China}
\author{Jie Zhu}
\affiliation{Department of Electronic Engineering,
  Shanghai Jiao Tong University,
  800 Dongchuan Road, Shanghai 200240, China}

%%%%%%%%%%%%%%%%%%%%%%%%%%%%%%%%%%%%%%%%%%%%%%%%%%%%%%%%%%%%%%%%%%%%%%%%%

%% The abstract should summarize concisely the content and conclusions
%% of the article.  The abstract should be a single paragraph of not more
%% than 250 words.

\begin{abstract}
\TODO{TODO} \quad
\NOTE{NOTE} \quad
\XU{XU} \quad
\LI{LI} \quad
\OK{OK}
\end{abstract}

%% Authors should select subject keywords from the given list.  No more
%% than 6 keywords should be given, and they should be listed in
%% alphabetical order.
%%
%% See the online documentation for the full list of available subject
%% keywords and the rules for their use.
%% http://journals.aas.org/authors/keywords2013.html
\keywords{
  dark ages, reionization, first stars ---
  galaxies: clusters: intracluster medium ---
  methods: numerical ---
  radio continuum: galaxies
}


%########################################################################
\section{Introduction}
\label{sec:intro}
%########################################################################

The Epoch of Reionization (EoR; $z \sim \numrange{6}{15}$) refers to
a period of our Universe preceded by the Cosmic Dawn
($z \sim \numrange{15}{30}$) and the Dark Ages ($z \sim \numrange{30}{200}$)
and is expected to last from about 300 million to about 1 billion years
after the Big Bang \citep[see][and references therein]{koopmans2015rev}.
During the EoR, the reionization of
neutral hydrogen (\Hi) caused primarily by the ultra-violet and soft
X-ray photons emitted from the first generation celestial objects
efficiently surpassed the cooling of the gas through
recombination of electrons and protons.
As a result, the majority of baryonic matter was again in a
highly ionized state.
Comparing with the observations of distant quasars and cosmic microwave
background (CMB), which have provided some important but loose
constraints on the reionization process, the detection the redshifted
21~cm line emission from \Hi{} in \SIrange{\sim 50}{200}{\MHz},
especially those from the adjacent regions of the ionized bubbles,
is recognized to be the decisive tool to directly reveal the EoR
\citep[see][for reviews]{zaroubi2013rev,furlanetto2016rev}.

In order to reveal the EoR, a number of radio interferometers working
at the low-frequency radio bands have been designed to target the
redshifted 21~cm signal, among which there are
the Square Kilometre Array (SKA; \citealt{mellema2013rev,koopmans2015rev}),
the Hydrogen Epoch of Reionization Array (HERA; \citealt{deboer2017}),
and their pathfinders, such as
the LOw Frequency ARray (LOFAR; \citealt{vanHaarlem2013}),
the Murchison Widefield Array (MWA; \citealt{bowman2013,tingay2013}),
the Precision Array for Probing the Epoch of Reionization
(PAPER; \citealt{parsons2010}),
and the 21 CentiMeter Array (21CMA; \citealt{zheng2016}).
The challenges met in these experiments, however, are immense
due to a variety of complicated instrumental effects,
ionospheric distortions, radio frequency interference, and the
strong celestial foreground contamination, which overwhelms the
redshifted 21~cm signal by \numrange{4}{5} orders of magnitude
\citep[see][for a review]{morales2010rev}.

Among various contaminating foreground components, the Galactic diffuse
radiation (including both the synchrotron and free-free emissions)
and extragalactic point sources are the most prominent and contribute
the majority of the foreground
\citep[e.g.,][]{shaver1999,diMatteo2004,gleser2008,liu2012,murray2017}.
At about \SI{150}{\MHz}, it is estimated that they may account for
about \SI{71}{\percent} and \SI{27}{\percent} of the total foreground,
respectively \citep{shaver1999}.
Most of the remaining foreground arises from the synchrotron emission
from the extragalactic diffuse sources that include the large-scale
filaments embedded in the cosmic webs,
the intergalactic medium located at the cluster outskirts (virial shocks),
and the intracluster medium (ICM) of galaxy clusters (radio halos,
radio relics, and mini-halos).
{\color{cyan}%
Although these sources are less constrained by observations, especially
in the low-frequency regime, they are suggested to be tens to hundreds
of times brighter than the target EoR signal and hence impede the EoR
detection \citep[e.g.,][]{waxman2000,diMatteo2004,gleser2008}.
As estimated in \citet{diMatteo2004}, the intensity fluctuation at
\SI{115}{\MHz} caused by extended radio halos and relics is about three
orders of magnitude larger than that of the redshifted 21~cm signal.
Consequently, a careful treatment for the extragalactic diffuse sources
is indispensable in detecting the elusive EoR signal.}

In this work, we attempt to address this issue by focusing on the
evolution of diffuse radio halos in galaxy clusters via detailed
modeling and their impacts on detecting the EoR signal.
First discovered in the Coma cluster \citep{large1959}, radio halos
has been observed in about 80 merging galaxy clusters roughly at
their centers, showing relatively regular morphologies and spatial
extensions of $\sim\,$\si{\Mpc}, as well as an obvious correlation
with the X-ray-emitting ICM \citep[e.g.,][]{cassano2013}.
It should be noted that the angular sizes of radio halos happen to be
several to tens of arcminutes, which coincides with those theoretically
predicted for the ionizing bubbles during the EoR.
This, complemented with the potentially large number
(several to tens of thousand in the whole sky)
of radio halos to be revealed by the forthcoming
low-frequency radio telescopes \citep[e.g.,][]{cassano2015},
indicates that in order to better guide foreground removal and avoidance
techniques, it is essential to investigate the cumulative emission
of radio halos as a source of the foreground contamination
\citep[e.g.,][]{diMatteo2004,gleser2008}.
As of today, however, only few works have been dedicated to this topic
and are all based on relatively straightforward modeling methods,
e.g., using the \SI{1.4}{\GHz} radio flux function or radio--X-ray
scaling relations, which are barely constrained by current observations
\citep[e.g.,][]{gleser2008,jelic2008}.
The data acquired in these observations are far from complete due to
the lack of systematic surveys targeting these diffuse sources,
especially in the low-frequency radio bands \citep[e.g.,][]{kale2016rev}.

This paper is prepared as follows.
In \autoref{sec:sky-simu} we introduce our new open-source software
\texttt{FG21sim} and use it to simulate the low-frequency radio sky,
among which a detailed simulation of radio halos in galaxy clusters has
been developed by applying the merger-induced turbulence re-acceleration
model.
In \autoref{sec:obs-simu} we employ the latest SKA1-Low layout
configuration to incorporate the practical instrumental effects into
the simulated sky maps.
We then quantitatively evaluate the contamination of radio halos to the
EoR signal and discuss the results in \autoref{sec:results}.
Finally we summarize our work in \autoref{sec:summary}.
Throughout this work we adopt a flat \lcdm{} cosmology with
$H_0 = \SI{100}{\hubble} = \SI{71}{\km\per\second\per\Mpc}$,
$\Omega_m = 0.27$, $\Omega_{\Lambda} = 1 - \Omega_m = 0.73$,
$\Omega_b = 0.046$, $n_s = 0.96$, and $\sigma_8 = 0.81$.
The quoted errors are at the \SI{68}{\percent} confidence level unless
otherwise stated.


%########################################################################
\section{Simulation of Low-Frequency Radio Sky}
\label{sec:sky-simu}
%########################################################################

Based on our previous works \citep{wang2010,wang2013}, we have developed
the \texttt{FG21sim} software to simulate the realistic low-frequency
radio sky by taking into account the contributions of our Galaxy,
extragalactic point sources, and radio halos in galaxy clusters.
Our software is written in Python and is open-sourced under the MIT License
at \url{https://github.com/liweitianux/fg21sim},
where further documents and auxiliary data are provided.
Additional utilities are also developed to help the observational simulation,
which aims to integrate the instrumental responses of low-frequency radio
interferometers, as well as subsequent analyzing procedures.

By consulting the latest SKA science book on the EoR survey strategies
\citep[e.g.,][]{koopmans2015rev}, we perform simulations within
a \SI{8}{\MHz} band centered at \SI{158}{\MHz} for a
\SI[product-units=repeat]{10 x 10}{\degree}
sky region at (R.A., Dec.) = (\SI{0}{\degree}, \SI{-27}{\degree}),
which locates at high galactic latitude ($b = \SI{-78.5}{\degree}$)
and passes through the zenith of SKA1-Low\footnote{%
  This sky region is also targeted by the MWA (EoR0 field;
  e.g., \citealt{beardsley2016}).}.
The \SI{8}{\MHz} bandwidth is chosen to limit the cosmological evolution
effect of the EoR signal when calculating power spectra
\citep[e.g.][]{wyithe2004,thyagarajan2013},
and the relatively large region of \SI[product-units=repeat]{10 x 10}{\degree}
will cover the first side-lobe, which helps the image CLEAN procedure
in \autoref{sec:obs-simu}.
The simulated sky is pixelized into \num{1800 x 1800} with a pixel size
of \SI{20}{\arcsecond}.


%========================================================================
\subsection{Radio Halos in Galaxy Clusters}
\label{sec:cluster-halo}
%========================================================================

As a significant improvement over our past works \citep{wang2010,wang2013},
we consider in detail the evolution of radio halos in galaxy clusters
by employing the merger-induced turbulence re-acceleration model,
in terms of which the relativistic electrons in the ICM are re-accelerated
by the turbulence generated in merger events via the second-order
Fermi process,
and lose energies due to mechanisms including synchrotron radiation,
inverse Compton scattering off the CMB photons, and Coulomb collisions
with the thermal ICM.
For each galaxy cluster, we first simulate its merging history
by using the extended Press-Schechter theory,
which is quantitatively described by a merger tree.
Then we calculate the turbulent acceleration efficiency and duration
for each merger event.
By combining the effects of turbulent acceleration and energy loss,
we obtain the temporal evolution of the relativistic electron spectrum,
and hence the synchrotron emission and image of the radio halo.

%------------------------------------------------------------------------
\subsubsection{Mass Function}
\label{sec:mass-function}
%------------------------------------------------------------------------

The Press-Schechter formalism was originally advanced as one of the standard
method to predict the mass function of galaxy clusters and its evolution
in the Universe \citep{press1974}, and has been extended to combine with
the cold dark matter (CDM) models \citep[e.g.,][]{bond1991,lacey1993}.
In this formalism, the number density (i.e., the number per unit comoving
volume per unit mass) of galaxy clusters at a certain redshift $z$ is
\begin{equation}
  \label{eq:ps-mass-func}
  n(M, z) \,\D{M} = \sqrt{\frac{2}{\pi}} \frac{\langle{\rho}\rangle}{M}
  \frac{\delta_c(z)}{\sigma^2(M)} \left| \diff{\sigma(M)}{M} \right|
  \exp\!\left[ -\frac{\delta_c^2(z)}{2\sigma^2(M)} \right] \D{M},
\end{equation}
where $M$ is the virial mass of galaxy clusters,
$\langle {\rho} \rangle$ is the current mean density of the Universe,
$\delta_c(z)$ is the critical linear overdensity for a region to collapse
at redshift $z$ [see \autoref{eq:delta-crit}],
and $\sigma(M)$ is the current r.m.s.\ (root-mean-square) density
fluctuations within a sphere of mean mass $M$.

Considering the CDM model and the mass range covered by galaxy clusters,
it is reasonable to adopt the following power-law distribution for the
density perturbations \citep{sarazin2002,randall2002}
\begin{equation}
  \label{eq:sigma-mass}
  \sigma(M) = \sigma_8 \left( \frac{M}{M_8} \right)^{-\alpha},
\end{equation}
where $\sigma_8$ is the present-day r.m.s.\ density fluctuations on
a scale of \SI{8}{\per\hubble\Mpc},
$M_8 = (4\pi/3)(8 \,\si{\per\hubble\Mpc})^3 \langle{\rho}\rangle$
is the mass contained in a sphere of a radius \SI{8}{\per\hubble\Mpc},
and the exponent $\alpha = (n+3)/6$ with $n = -7/5$ \citep{randall2002}
is related to the fluctuations pattern whose power spectrum varies
with wavenumber $k$ as $k^n$.

Using a minimum galaxy cluster mass of
$M_{\R{min}} = \SI{e14}{\solarmass}$
and a maximum redshift cut at $z_{\R{max}} = 4$,
we apply \autoref{eq:ps-mass-func} to derive the total number of
galaxy clusters in any given sky patch, together with their mass and
redshift distributions.
Based on these distributions, we run Monte Carlo simulations to
generate the galaxy cluster sample with random masses
$M_{\R{sim}}$ and redshifts $z_{\R{sim}}$.

%------------------------------------------------------------------------
\subsubsection{Merging History}
\label{sec:merging-history}
%------------------------------------------------------------------------

Galaxy clusters are formed hierarchically through accretion and mergers.
The extended Press-Schechter theory outlined in \citet{lacey1993} provides
a way to describe their growth history in terms of the merger tree.
In order to build the merger tree for a galaxy cluster, we start with
its current mass $M_{\R{sim}}$ and redshift $z_{\R{sim}}$ as randomly sampled
in \autoref{sec:mass-function}, and trace its growth history back in time
by running Monte Carlo simulations to randomly determine the mass change
$\Delta M$ at each step, which may be regarded either as a merger event
(if $\Delta M > \Delta M_c$) or as an accretion event
(if $\Delta M \leq \Delta M_c$).
Considering that radio halos are usually associated with major mergers,
we choose $\Delta M_c = \SI{e13}{\solarmass}$ \citep[e.g.,][]{cassano2005}.

We assume that during each growth step the cluster mass increases
from $M_1$ at time $t_1$ to $M_2$ at a later time $t_2$ ($> t_1$).
Given $M_2$ and $t_2$, the conditional probability of the cluster had
a progenitor being in the mass range $M_1 \to M_1 + \D{M_1}$ at an
earlier time $t_1$ can be expressed as
\begin{equation}
  \label{eq:eps-condprob}
  \R{Pr}(M_1, t_1 | M_2, t_2) \,\D{M_1} = \frac{1}{\sqrt{2\pi}}
  \frac{M_2}{M_1}
  \frac{\delta_{c1} - \delta_{c2}}{(\sigma_1^2 - \sigma_2^2)^{3/2}}
  \left| \diff{\sigma_1^2}{M_1} \right|
  \exp \!\left[ -\frac{(\delta_{c1} - \delta_{c2})^2}
    {2(\sigma_1^2 - \sigma_2^2)} \right] \D{M_1},
\end{equation}
where
$\delta_{ci} \equiv \delta_c(t_i)$, $\sigma_i \equiv \sigma(M_i)$, and $i =
1, 2$ is used to denote parameters defined at time $t_1$ and $t_2$,
respectively \citep{lacey1993,randall2002}.
By further introducing $S \equiv \sigma^2(M)$ and
$\omega \equiv \delta_c(t)$, this equation reduces to
\begin{equation}
  \label{eq:eps-condprob-simp}
  \R{Pr}(\Delta S, \Delta \omega) \,\D{\Delta S} = \frac{1}{\sqrt{2\pi}}
  \frac{\Delta\omega}{(\Delta S)^{3/2}}
  \exp \!\left[ -\frac{(\Delta\omega)^2}{2 \Delta S} \right] \D{\Delta S}.
\end{equation}

In order to resolve mergers with a mass change $\Delta M_c \ll M$
during the backward tracing of a galaxy cluster, a time step $\Delta t$
(i.e., $\Delta\omega$) that satisfies
\begin{equation}
  \label{sec:dw-step}
  \Delta\omega \lesssim \Delta\omega_{\R{max}} = \left[
    S \left| \diff{\ln \sigma^2}{\ln M} \right|
    \left( \frac{\Delta M_c}{M} \right) \right]^{1/2}
\end{equation}
is required \citep{lacey1993},
{\color{cyan}%
and we adopt an adaptive step of
$\Delta\omega = \Delta\omega_{\R{max}} / 2$ \citep{randall2002}.}
At a certain step when $\Delta\omega$ is given, the mass change
$\Delta S$ can be randomly drawn from the following cumulative
probability distribution of sub-cluster masses
\begin{equation}
  \label{sec:cdf-sub-masses}
  \R{Pr}(<\!\Delta S, \Delta\omega) =
  \int_0^{\Delta S} \R{Pr}(\Delta S', \Delta\omega) \,\D{\Delta S'} =
  \erfc \!\left( \frac{\Delta \omega}{\sqrt{2 \Delta S}} \right),
\end{equation}
where $\erfc(x) = 1 - \erf(x)$ is the complementary error function,
and the cluster's progenitor mass $M_1$ is then obtained as
$S_1 = S_2 + \Delta S$.

Considering that observable radio halos are regarded to be associated
with recent (in the frame of the observer) major mergers
and have typical lifetimes $\tau_{\R{halo}} \lesssim \SI{1}{\Gyr}$
\citep[e.g.,][]{brunetti2009,cassano2016},
we trace the merging history of each galaxy cluster by a maximum
backward time of \SI{3}{\Gyr} from its \enquote{current} age
$t_{\R{sim}}$ (corresponding to $z_{\R{sim}}$).
For each merger tree that is built, we extract the information of all
the mergers associated with the main cluster to carry out the
radio halo simulation.


%------------------------------------------------------------------------
\subsubsection{Radio Halo Emergence and Evolution}
\label{sec:halos}
%------------------------------------------------------------------------

According to the re-acceleration model, there exists a population of
primary (or fossil) high-energy electrons, which permeate the ICM.
They are thought to be injected by active galactic nucleus (AGN)
activities and star formation \citep[see][for a review]{blasi2007rev}.
When a cluster experiences a merger, the turbulence is generated
throughout the ICM and can accelerate the primary electrons to be highly
relativistic, resulting in the observed radio halo.

In addition to the energy increase that is caused by the turbulent
acceleration, electrons in the ICM lose energy via synchrotron radiation,
inverse Compton scattering off the CMB photons,
Coulomb collisions, etc.\ \citep{sarazin1999}.
For a population of electrons with isotropic energy distribution, the
temporal evolution of the number density distribution $n(\gamma, t)$
is governed by the following Fokker-Planck diffusion-advection equation
\citep{eilek1991,schlickeiser2002}
\begin{equation}
  \label{eq:fokkerplanck}
  \pdiff{n(\gamma,t)}{t} = \pdiff{}{\gamma} \left[ n(\gamma,t) \left(
      \left| \diff{\gamma}{t} \right| -
      \frac{2}{\gamma} D_{\gamma\gamma}(\gamma, t) \right) \right] +
    \pdiff{}{\gamma} \left[ D_{\gamma\gamma} \pdiff{n(\gamma,t)}{\gamma}
    \right] + Q_e(\gamma,t),
\end{equation}
where $\gamma$ is the Lorentz factor of electrons,
$D_{\gamma\gamma}(\gamma, t)$ is the diffusion coefficient describing
the interactions between the turbulence and electrons,
$|\R{d}\gamma / \R{d}t|$ is the energy loss rate,
and $Q_e(\gamma, t)$ describes the electron injection.

%........................................................................
\setcounter{sssseccount}{0}
{\color{cyan}%
\ssssec{Thermal Electron Number and Energy Densities}}

\XU{Number Density of Thermal Electrons and the Associated Thermal Energy Density ???}

The number density of thermal electrons $n_{\R{th}}$ in the ICM can be
calculated as
\begin{equation}
  \label{eq:number-density-thermal}
  n_{\R{th}} \simeq \frac{3 f_b M_{\R{vir}}}{4\pi \mu m_u \,r^3_{\R{vir}}},
\end{equation}
where
$f_b \simeq \Omega_b/\Omega_m$ is the mean baryon fraction assumed for
galaxy clusters,
$M_{\R{vir}}$ is the cluster's virial mass,
$r_{\R{vir}}$ is the virial radius [see \autoref{eq:radius-virial}],
$\mu \simeq 0.6$ is the mean molecular weight \citep[e.g.,][]{ettori2013},
and $m_u$ is the atomic mass unit.
Therefore, the corresponding thermal energy density $\epsilon_{\R{th}}$
of the ICM is given by
\begin{equation}
  \label{eq:energy-density-thermal}
  \epsilon_{\R{th}} = \frac{3}{2} \,n_{\R{th}} \,k_BT_{\R{cl}},
\end{equation}
where the cluster's mean temperature is approximated as the virial
temperature, i.e.,
$k_BT_{\R{cl}} \simeq k_BT_{\R{vir}} = G M_{\R{vir}} \mu m_u / (2 \,r_{\R{vir}})$.

%........................................................................
\ssssec{Electron Injection}

As primary electrons are continuously injected into the ICM by AGN
activities and star formation, it is reasonable to assume an average
injection rate \citep[e.g.,][]{cassano2005,donnert2014} and a power-law
spectrum for the electron injection process
\begin{equation}
  \label{eq:electron-inj}
  Q_e(\gamma, t) \simeq Q_e(\gamma) = K_e \,\gamma^{-s},
\end{equation}
where $s = 2.3$ is the adopted spectral index \citep[e.g.,][]{sarazin1999}.
Moreover, the total energy of injected electrons is assumed to account for
a fraction ($\eta_e$) of the ICM's total thermal energy $\epsilon_{\R{th}}$
\citep{cassano2005}, i.e.,
\begin{equation}
  \epsilon_e \,\tau_{\R{cl}} \int_{\gamma_{\R{min}}}^{\gamma_{\R{max}}}
  Q_e(\gamma') \gamma' \,\D{\gamma'}
  = \eta_e \,\epsilon_{\R{th}},
\end{equation}
where $\tau_{\R{cl}} \simeq t_{\R{sim}}$ is the cluster's age at its
\enquote{current} redshift $z_{\R{sim}}$ as simulated in
\autoref{sec:mass-function},
and $\epsilon_e = m_e c^2$ is the electron's rest energy.
Given $\gamma_{\R{min}} \ll \gamma_{\R{\max}}$, the injection rate $K_e$
is derived to be
\begin{equation}
  \label{eq:injrate}
  K_e \simeq \frac{(s-2)\,\eta_e\,\epsilon_{\R{th}}}{\epsilon_e\,\tau_{\R{cl}}}
    \gamma_{\R{min}}^{s-2}.
\end{equation}

%........................................................................
\ssssec{Initial Electron Spectrum}

To determine the initial electron spectrum $n_e(\gamma, t_0)$, which will
be used to solve \autoref{eq:fokkerplanck},
{\color{cyan}%
we evolve the accumulated electron spectrum
$n_e'(\gamma) = Q_e(\gamma) \,t_0$ for \SI{\sim 1}{\Gyr} by applying the
same Fokker-Planck equation but with turbulent acceleration disabled,}
where $t_0$ is the beginning time of the first merger begins
\citep[e.g.,][]{brunetti2007}.

%........................................................................
\ssssec{Turbulent Acceleration}

The details of interactions between the turbulence and both thermal and
relativistic electrons are complicated and still poorly understood.
Among several particle acceleration mechanisms that can be potentially
triggered by the turbulence, the most important one is the transit time
damping process, i.e., the turbulence dissipates its energy and
accelerates particles by interacting with the relativistic particles
(e.g., cosmic rays) in the ICM
\citep[and references therein]{brunetti2007,brunetti2011}.
The associated diffusion coefficient is given as
\citep{pinzke2017,miniati2015}
\begin{equation}
  \label{eq:dpp}
  D_{\gamma\gamma} = \gamma^2 \frac{4\pi \zeta}{X_{\R{cr}} L_{\R{turb}}}
    \frac{\langle (\delta{v_t})^2 \rangle^2}{c_s^3},
\end{equation}
where
$\zeta \sim \numrange{0.1}{0.3}$ is an efficiency factor characterizing
the plasma instabilities (e.g., due to spatial or temporal intermittency),
$X_{\R{cr}} = \epsilon_{\R{cr}} / \epsilon_{\R{th}} \sim \SI{1}{\percent}$
is the relative energy density of cosmic rays with respect to the thermal ICM,
$L_{\R{turb}} \simeq r_{\R{vir}}/3$ is the turbulence injection scale
\citep{miniati2015},
$\langle (\delta{v_t})^2 \rangle$ is the turbulence velocity dispersion,
and $c_s = \sqrt{\gamma_{\R{gas}} k_BT_{\R{cl}} / (\mu m_u)}$ is the sound
speed in the ICM with $\gamma_{\R{gas}} = 5/3$ being the adiabatic index
of ideal monatomic gas.

In order to estimate the velocity dispersion of the turbulence, we assume
that a fraction ($\eta_{\R{turb}}$) of the kinetic energy carried by the
merger is transferred into turbulent waves, i.e.,
\begin{equation}
  \label{eq:energy-turb}
  E_{\R{turb}} = \frac{1}{2} M_{\R{turb}}^b
    \langle (\delta{v_t})^2 \rangle = \eta_{\R{turb}} E_{\R{merger}},
\end{equation}
where
$M_{\R{turb}}^b$ is the baryon mass enclosed in the turbulent region
(i.e., within radius of $L_{\R{turb}}$),
and $E_{\R{merger}}$ is merger's kinetic energy, which can be approximated
as the potential energy released from the infalling sub-cluster
\begin{equation}
  \label{eq:energy-merger}
  E_{\R{merger}} \simeq \frac{1}{2} f_b M_{\R{vir,s}} \,v^2_{\R{vir}}
    \simeq \frac{1}{2} f_b M_{\R{vir,s}}
    \frac{G M_{\R{vir,m}}}{r^2_{\R{vir,m}}},
\end{equation}
where
$M_{\R{vir,m}}$ and $r_{\R{vir,m}}$ are the virial mass and radius of the
main cluster, respectively, and $M_{\R{vir,s}}$ is the virial mass of the
sub-cluster.
By adopting the NFW density profile \citep{navarro1997}, the distribution
of mass within a radius of $s \,r_{\R{vir}}$ ($s$ is the normalized radius
in units of the virial radius) in fraction of the virial mass is
\begin{equation}
  \label{eq:mass-dist-nfw}
  f_{\R{nfw}}(s) = \frac{M(< s \,r_{\R{vir}})}{M_{\R{vir}}} =
    \frac{\ln(1 + c s) - c s / (1 + c s)}{\ln(1 + c) - c / (1 + c)},
\end{equation}
where
$c \sim 5$ is the concentration parameter for clusters \citep{lokas2001}.
Therefore we have
\begin{equation}
  \label{eq:mass-turb}
  M_{\R{turb}}^b = f_b M_{\R{merged}}(<L_{\R{turb}})
    = f_b f_{\R{nfw}}(1/3) (M_{\R{vir,m}} + M_{\R{vir,s}}).
\end{equation}
By substituting Equations~(\ref{eq:energy-merger}) and (\ref{eq:mass-turb})
into \autoref{eq:energy-turb}, we obtain the turbulence velocity dispersion
as
\begin{equation}
  \label{eq:turb-velocity-dispersion}
  \langle (\delta{v_t})^2 \rangle =
    \frac{\eta_{\R{turb}}}{f_{\R{nfw}}(1/3)}
    \frac{M_{\R{vir,s}}}{M_{\R{vir,m}} + M_{\R{vir,s}}}
    \frac{G M_{\R{vir,m}}}{r^2_{\R{vir,m}}}.
\end{equation}

Generally, the turbulence generated during major mergers is subsonic
with a Mach number of
$\mathcal{M}_{\R{turb}} = \sqrt{\langle (\delta{v_t})^2 \rangle} / c_s
  \sim \numrange{0.2}{0.5}$,
and the acceleration efficiency can be described by the systematic
acceleration timescale (i.e., the reciprocal of the acceleration rate)
$\tau_{\R{acc}} = 4 D_{\gamma\gamma} / \gamma^2 \sim \SI{0.1}{\Gyr}$,
suggesting that the turbulent acceleration may be a moderately efficient
mechanism to accelerate the relativistic electrons in the ICM and be used
to explain the generation of radio halos.

%........................................................................
\ssssec{Acceleration Periods}

The whole merger process may last for \SIrange{2}{3}{\Gyr}
\citep[e.g.,][]{tormen2004,cassano2016},
however, the period of strong turbulence $\tau_{\R{turb}}$, during with
the turbulent acceleration becomes effective, is relatively short:
$\tau_{\R{turb}} \simeq 2 L_{\R{turb}} / v_{\R{imp}}$,
where $v_{\R{imp}}$ is the relative impact velocity between the merging
clusters \citep{miniati2015}.
Starting from a sufficiently large distance with zero velocity,
the relative impact velocity of two merging clusters with mass
$M_{\R{vir,m}}$ and $M_{\R{vir,s}}$ is given by
\citep{sarazin2002,cassano2005}
\begin{equation}
  \label{eq:v-imp}
  v_{\R{imp}} \simeq \left[
    \frac{2G (M_{\R{vir,m}} + M_{\R{vir,s}})}{r_{\R{vir,m}}}
    \left( 1 - \frac{1}{\eta_v} \right)\right]^{1/2},
\end{equation}
with $\eta_v \simeq 4 \,(1 + M_{\R{vir,s}}/M_{\R{vir,m}})^{1/3}$.
Based on these, we estimate that major mergers generally have an effective
turbulent acceleration duration $\tau_{\R{turb}} \sim \SI{0.5}{\Gyr}$.

For each merger event
$(M^{(i)}_{\R{vir,m}}, M^{(i)}_{\R{vir,s}}, t^{(i)}_{\R{merger}})$
of a galaxy cluster as simulated in \autoref{sec:merging-history},
{\color{cyan}
where $t^{(i)}_{\R{merger}}$ is the beginning time of the $i$-th merger
event,} we calculate the corresponding turbulent acceleration duration
$\tau^{(i)}_{\R{turb}}$, and solve the Fokker-Planck equation by taking
into account both turbulent acceleration and energy losses among the
period $[\,t^{(i)}_{\R{merger}}, t^{(i)}_{\R{merger}}+\tau^{(i)}_{\R{turb}}\,]$,
\textcolor{cyan}{while there is no acceleration at other times}.

%........................................................................
\ssssec{Energy Losses}

There are several mechanisms through which the relativistic electrons in
the ICM can lose energies \citep{sarazin1999}, and we take into account
the following three major mechanisms in our simulation.
The first one is the inverse Compton scattering off the CMB photons,
the energy loss rate of which is
\begin{equation}
  \label{eq:eloss-ic}
  \left( \diff{\gamma}{t} \right)_{\R{IC}} =
    \num{-4.32e-4} \,\gamma^2 (1+z)^4
    \quad [\si{\per\Gyr}].
\end{equation}

Secondly, with the \si{\uG}-level magnetic field permeating the ICM,
relativistic electrons will produce synchrotron radiation and lose
energy at a rate of
\begin{equation}
  \label{eq:eloss-syn}
  \left( \diff{\gamma}{t} \right)_{\R{syn}} =
    \num{-4.10e-5} \,\gamma^2 \left( \frac{B}{\SI{1}{\uG}} \right)^2
    \quad [\si{\per\Gyr}],
\end{equation}
where $B$ is the averaged magnetic field strength.
We assume that the magnetic field is uniform and its energy density
is a fraction of the thermal energy density, i.e.,
$u_B = B^2 / (8\pi) = \eta_B \,\epsilon_{\R{th}}$,
where $\eta_B$ is the energy density ratio between the magnetic field
and the thermal ICM \citep[e.g.,][]{bohringer2016}.

The last mechanism considered here is that electrons lose energy by
interacting with the thermal electrons via Coulomb collisions, the
loss rate of which is
\begin{equation}
  \label{eq:eloss-coul}
  \left( \diff{\gamma}{t} \right)_{\R{Coul}} =
  \num{-3.79e4} \,n_{\R{th}} \left[ 1 +
    \frac{\ln(\gamma/n_{\R{th}})}{75} \right] \quad [\si{\per\Gyr}],
\end{equation}
where $n_{\R{th}}$ is the thermal number density
[\autoref{eq:number-density-thermal}] in units of [\si{\per\cm\cubed}].

The inverse Compton scattering and synchrotron radiation dominate
the energy losses at the high-energy regime ($\gamma \gtrsim 1000$),
while Coulomb collisions are the main energy-loss mechanism for electrons
with lower energies ($\gamma \lesssim 100$).
Therefore, electrons with intermediate energies (e.g., $\gamma \sim 300$)
have a very long lifetime (\SI{\sim 3}{\Gyr}) and can accumulate in the
ICM as the cluster grows \citep{sarazin1999}, which may become the primary
electrons to be accelerated \textit{in situ} by the turbulence.

%........................................................................
\ssssec{Numerical Method}

Now with all the coefficients in \autoref{eq:fokkerplanck} been determined,
we can evolve the initial electron spectrum $n_e(\gamma,\, t=t_0)$ via
solving the Fokker-Planck equation, and derive the \enquote{current}
electron spectrum $n_e(\gamma,\, t=t_{\R{sim}})$,
where $t_{\R{sim}}$ corresponds to the simulated redshift $z_{\R{sim}}$
(\autoref{sec:mass-function}).
An efficient numerical method proposed by \citet{chang1970} is utilized
to solve the equation, under the no-flux boundary condition \citep{park1996}.
To avoid unphysical pile-up of electrons around the lower boundary caused
by the boundary condition,
we define a \enquote{buffer} region below $\gamma_{\R{buf}}$, within which
the spectral data are replaced by extrapolating the data above
$\gamma_{\R{buf}}$ as a power-law spectrum \citep{donnert2014}.
We adopt a logarithmic grid for $\gamma \in [1, 10^5]$ with 200 cells,
and let the lower buffer region span 10 cells.

%........................................................................
\ssssec{Image Generation}

After deriving the \enquote{current} electron spectrum
$n_e(\gamma, t_{\R{sim}})$, the synchrotron emissivity at frequency
$\nu$ is given by \citep{rybicki1979}
\begin{equation}
  \label{sec:jnu-sync}
  % unit: [erg/s/Hz/cm^3]
  J_{\R{syn}}(\nu) = \frac{\sqrt{3} \, e^3 B}{m_e c^2}
    \!\int_{\gamma_{\R{min}}}^{\gamma_{\R{max}}} \!\!\!\int_0^{\pi/2}\!
    n_e(\gamma, t_{\R{sim}}) F(\nu/\nu_c)
    \sin^2 \!\theta \,\D{\theta} \,\D{\gamma},
\end{equation}
where $\theta$ is the pitch angle of electrons,
and $F(\nu/\nu_c) = (\nu/\nu_c) \int_{\nu/\nu_c}^{\infty} K_{5/3}(x) \,\D{x}$
is the synchrotron kernel,
with $K_{5/3}(x)$ the modified Bessel function of 5/3 order,
$\nu_c = (3/2) \,\gamma^2 \nu_L \sin\theta
= 3 e B \gamma^2 \sin\theta / (4\pi m_e c)$
the electron's critical frequency.
The simulated radio halo is assumed to be the same size as the turbulent
region \citep[e.g.,][]{vazza2011}:
$r_{\R{halo}} \simeq L_{\R{turb}} \simeq r_{\R{vir}}/3$,
therefore its radio power at a certain frequency (i.e., spectral power) is
\begin{equation}
  \label{eq:halo-power}
  % unit: [W/Hz]
  P(\nu) = \frac{4\pi}{3} r_{\R{halo}}^3 J_{\R{syn}}(\nu),
\end{equation}
and its flux density measured at the same frequency is
\citep[e.g.,][]{hogg1999}
\begin{equation}
  \label{eq:halo-flux}
  % unit: [Jy] = 1e-26 [W/Hz/m^2]
  S(\nu) = \frac{(1+z_{\R{sim}}) P(\nu(1+z_{\R{sim}}))}{4\pi D_{\!L}^2(z_{\R{sim}})},
\end{equation}
where $D_{\!L}(z_{\R{sim}})$ is the luminosity distance to the halo at
redshift $z_{\R{sim}}$,
and the factor $(1 + z_{\R{sim}})$ accounts for the $K$ correction.

To generate the image for the simulated radio halos, we adopt an exponential
profile for the azimuthally averaged brightness distribution \citep{murgia2009}:
\begin{equation}
  \label{eq:halo-profile}
  I_{\nu}(\theta) = I_{\nu,0} \exp(-3 \,\theta / \theta_{\R{halo}}),
\end{equation}
where $\theta = r / D_{\!A}(z_{\R{sim}})$ is the angular radius from the halo
center with $D_{\!A}(z_{\R{sim}})$ the angular diameter distance to the halo,
and $I_{\nu,0} = 9 S(\nu) / (2\pi \theta^2_{\R{halo}})$ is the central
brightness.
In addition, a random ellipticity $f_e$ and a random rotation angle
$\phi$ are applied to each of the simulated halos.

%........................................................................
\ssssec{Parameters Tuning}

Our model has the following 5 parameters:
(1) fraction of the merger energy transferred into the turbulence:
$\eta_{\R{turb}} \sim \SI{10}{\percent}$;
(2) relative energy density of the magnetic field:
$\eta_B \sim \SI{0.1}{\percent}$;
(3) ratio of the total energy of injected electrons to the total thermal
energy: $\eta_e \sim \SI{0.1}{\percent}$;
(4) efficiency of the ICM plasma instabilities:
$\zeta \sim \numrange{0.1}{0.3}$;
(5) relative energy density in the cosmic rays:
$X_{\R{cr}} \sim \SI{1}{\percent}$.
Although all of them have reasonable constraints from either observational
evidences or simulation studies, it is essential to further tune them in
order to achieve better agreements with the observations.
To this end, we have collected all current observed radio halos
(\autoref{tab:halos-observed}; as of 2018 January), and compare the
\SI{1.4}{\GHz} integrated flux function between our simulated radio halos
and the observations.  We have explored various parameter configurations,
and for each configuration we have repeated the simulation for 500 times
in order to take into account the statistical uncertainties as introduced
in Sections~\ref{sec:mass-function} and \ref{sec:merging-history}.
Considering that the current observations are far from complete,
especially in the low-flux regime, our strategy is requiring that the
integrated flux function of simulated radio halos agrees with the observed
one in the high-flux regime within simulation uncertainties.
Consequently, we have chosen a set of model parameters with
$\eta_{\R{turb}} = \SI{10}{\percent}$, $\eta_B = \SI{0.1}{\percent}$,
$\eta_e = \SI{0.1}{\percent}$, $\zeta = 0.1$,
and $X_{\R{cr}} = \SI{1.5}{\percent}$.
A comparison between the \SI{1.4}{\GHz} integrated flux functions of
radio halos obtained in observations and simulations is shown in
\autoref{fig:halos-simucomp-example}(a),
where we also plot the \SI{158}{\MHz} integrated flux function for the
simulated radio halos, which indicates that radio halos are much brighter
in the low-frequency regime.

With the purpose of better characterizing the impacts of radio halos
on the EoR detection, we adopt the above tuned parameter configuration
and repeat the radio halos simulation for 100 times, considering that
radio halos can vary significantly across the sky.
The mean (r.m.s.) brightness temperature of the 100 simulations is
$24.9_{-20.0}^{+74.3}$ \si{\mK} ($2\,753_{-1\,972}^{+10\,409}$ \si{\mK}),
and \autoref{fig:halos-simucomp-example}(b) shows a typical example
of the simulated radio halos at \SI{158}{\MHz} with mean (r.m.s.)
brightness temperature of \SI{25.2}{\mK} (\SI{3947}{\mK}).

\begin{figure*}
  \gridline{
    \fig{fluxfunc-simucomp-1400}{0.49\textwidth}{%
      (a) Integrated flux function comparison}
    \fig{halos-f158-C17}{0.45\textwidth}{%
      (b) Example map of radio halos at \SI{158}{\MHz}}
  }
  \caption{\label{fig:halos-simucomp-example}%
    \textbf{(a)} The \SI{1.4}{\GHz} integrated flux function comparison
    between the observed (red dashed line) and simulated (red solid line)
    radio halos.  We also plot the \SI{158}{\MHz} integrated flux function
    for the simulated halos (blue solid line), which shows that radio
    halos are much brighter at the low-frequency band.  The shadowed
    regions show the \SI{68}{\percent} uncertainties of the simulated
    radio halos derived from the 500 simulation runs.
    \textbf{(b)} A typical example from the 100 simulation runs showing
    the simulated radio halos at \SI{158}{\MHz}, and the colorbar is in
    units of \si{\kelvin}.
  }
\end{figure*}


%========================================================================
\subsection{Other Foreground Components}
\label{sec:fg-other}
%========================================================================

Following our previous work \citep{wang2010}, we have also simulated
several other foreground components, including the Galactic synchrotron
and free-free emissions as well as the extragalactic point sources,
in order to carry out comparisons of power spectra between radio halos
and other foreground components as an effort to better characterize the
contributions of radio halos to the low-frequency radio sky.

The Galactic synchrotron map is simulated by extrapolating the
Haslam \SI{408}{\MHz} all-sky map as the template to lower frequencies
with a power-law spectrum.
We make use of the $N_{\R{side}} = 2048$
(pixel size \SI{\sim 1.72}{\arcminute})
high-resolution version of the Haslam \SI{408}{\MHz} map\footnote{%
  The reprocessed Haslam \SI{408}{\MHz} map:
  \url{http://www.jb.man.ac.uk/research/cosmos/haslam_map/}.},
which was reprocessed by \citet{remazeilles2015} using significantly
better instrument calibration and more accurate subtraction of
extragalactic sources.
We also use the all-sky synchrotron spectral index map made by
\citet{giardino2002} to account for the index variation with sky positions.
The Galactic free-free emission is deduced from the \Halpha{} survey
data \citep{finkbeiner2003}, which is corrected for dust absorption,
by employing the tight relation between the \Halpha{} and free-free
emissions due to their common origins \citep{dickinson2003}.

The extragalactic point sources are simulated by taking into account the
following 5 types of sources: (1) star-forming and starburst galaxies,
(2) radio-quiet AGNs, (3) Fanaroff-Riley type I and type II AGNs,
(4) GHz-peaked spectrum AGNs, and (5) compact steep spectrum AGNs.
We simulate the former 3 types of sources by leveraging the simulation
results made by \citet{wilman2008}, and simulate the latter 2 types
by employing their corresponding luminosity functions and spectral models.
More details can be found in \citet{wang2010} and references therein.

Example maps simulated at \SI{158}{\MHz} for different foreground
components are shown in \autoref{fig:gsyn-gff-ps-21cm}(a--c).
The mean (r.m.s.) brightness temperatures of the Galactic synchrotron
emission, Galactic free-free emission, and extragalactic point sources
are \SI{251.4}{\kelvin} (\SI{251.7}{\kelvin}),
\SI{0.19}{\kelvin} (\SI{0.20}{\kelvin}),
and \SI{45.5}{\kelvin} (\SI{5.9e4}{\kelvin}), respectively.

\begin{figure*}
  \gridline{
    \fig{gsyn-f158-he}{0.45\textwidth}{(a) Galactic synchrotron emission}
    \fig{gff-f158-he}{0.45\textwidth}{(b) Galactic free-free emission}
  }
  \gridline{
    \fig{pointsources-f158-he-log}{0.45\textwidth}{(c) Extragalactic point sources}
    \fig{21cm-f158-he}{0.45\textwidth}{(d) EoR signal}
  }
  \caption{\label{fig:gsyn-gff-ps-21cm}%
    The simulated example maps of other foreground components and
    the EoR signal at \SI{158}{\MHz}.
    \textbf{(a)} Galactic synchrotron emission;
    \textbf{(b)} Galactic free-free emission;
    \textbf{(c)} Extragalactic point sources (color in logarithmic scale);
    \textbf{(d)} EoR signal.
    All colorbars are in units of \si{\kelvin}.
    \TODO{add text labels; zoom in colorbars}
  }
\end{figure*}


%========================================================================
\subsection{The EoR Signal}
\label{sec:eor-signal}
%========================================================================

The sky maps of the EoR 21~cm signal are created using the 2016
data release from the Evolution Of 21~cm Structure project\footnote{%
  Evolution Of 21~cm Structure project:
  \url{http://homepage.sns.it/mesinger/EOS.html}}.
The project used \texttt{21cmFAST} version 2 \citep{mesinger2011} to
simulate the cosmic reionization process from redshift 86.5 to 5.0 inside
a large cube which is 1.6 comoving \si{\Gpc} (1024 cells) along each side
\citep{mesinger2016}.
We extract the image slices at needed frequencies (i.e., redshifts) from
the light-cone cubes of the recommended \enquote{faint galaxies} case,
and then tile and re-scale them to have the same sky coverage and
pixel size as our foreground maps.
The extracted image at \SI{158}{\MHz} is shown in
\autoref{fig:gsyn-gff-ps-21cm}(d),
and the mean (r.m.s.) brightness temperature is
\SI{9.53}{\mK} (\SI{10.69}{\mK}).


%########################################################################
\section{Simulated SKA Observations}
\label{sec:obs-simu}
%########################################################################

In order to properly evaluate the possible contamination induced by radio
halos on the EoR detection, it is essential to incorporate the practical
instrumental effects of radio interferometers.
We therefore employ the latest SKA1-Low layout configuration\footnote{%
  SKA1-Low Configuration Coordinates:
  \url{https://astronomers.skatelescope.org/wp-content/uploads/2016/09/SKA-TEL-SKO-0000422_02_SKA1_LowConfigurationCoordinates-1.pdf}
  (released on 2016 May 21)
}
to simulate the \enquote{observed} images for the above simulated sky
maps of foreground components and EoR signal.
According to the latest configuration,
the SKA1-Low interferometer consists of 512 stations, with 224 of them
randomly distributed within the \enquote{core} of \SI{1000}{\meter} in
diameter, while the remaining stations are grouped into \enquote{clusters}
and placed on 3 spiral arms extending up to a radius of
\SI{\sim 35}{\kilo\meter}.
Each station has 256 antennas randomly distributed with a minimum separation
of $d_{\R{min}} = \SI{1.5}{\meter}$ inside a circular region of
\SI{35}{\meter} in diameter \citep[e.g.,][]{mort2017}.
With this layout configuration, the dense core makes the SKA1-Low very
sensitive to the EoR signal, while the long baselines greatly help the
foreground characterization and removal, as well as other scientific goals.

We divide the chosen \SIrange{154}{162}{\MHz} frequency band into 51
channels with a frequency resolution of \SI{160}{\kilo\hertz}.
For each component, we simulate the input sky maps at every frequency
channel by following the steps described in \autoref{sec:sky-simu},
and then use the \texttt{OSKAR}\footnote{%
  OSKAR: \url{https://github.com/OxfordSKA/OSKAR} (version 2.7.0)}
simulator \citep{mort2010} to perform simulated observations
for \SI{6}{\hour} ($\pm \SI{3}{\hour}$ in Hour Angle) with the
SKA1-Low layout configuration.
The simulated visibility data are imaged through the
\texttt{WSClean}\footnote{%
  WSClean: \url{https://sourceforge.net/p/wsclean} (version 2.5)}
imager \citep{offringa2014} using Briggs' weighting with
robustness of 0 \citep{briggs1995},
and the created images are cropped to keep only the central
\SI[product-units=repeat]{4 x 4}{\degree} regions for subsequent analyses.

\textcolor{cyan}{%
Since the Galactic synchrotron and free-free emissions have similar
diffuse structures, and there is not need to distinguish the two
emissions in this work, we combine them in the observational simulation.}
Similar to the real-time peeling of the brightest sources in practical
data analysis pipeline of radio interferometers
\citep[e.g.,][]{mitchell2008,intema2009}, we assume that the brightest
ones with a \SI{158}{\MHz} flux density $S > \SI{50}{\mJy}$
are peeled \citep[e.g.,][]{pindor2011}.
Moreover, we create foreground image cubes using the CLEAN algorithm
with joined-channel deconvolution in order to ensure the spectral
smoothness \citep{offringa2017}, which is crucial to extract the EoR
signal from the overwhelming foregrounds.
Thus we obtain the \enquote{observed} image cubes of the EoR signal,
radio halos, the Galactic diffuse emission (with both synchrotron and
free-free emission), and the extragalactic point sources (with the
brightest ones removed).


%########################################################################
\section{Results}
\label{sec:results}
%########################################################################

Using the image cubes obtained in \autoref{sec:obs-simu}, we can now
calculate both the two-dimensional (2D) and one-dimensional (1D) power
spectra for each component, from which we can quantitatively evaluate the
contamination to the EoR measurements contributed by radio halos.
\textcolor{cyan}{%
The power spectra of radio halos can vary remarkably due to the large
fluctuation of their distributions across the sky.
We investigate how the large fluctuation of radio halos may cause
significant difficulties for the EoR measurements and how to possibly
deal with it.}

%========================================================================
\subsection{2D Power Spectra and EoR Window}
\label{sec:ps2d}
%========================================================================

Current-generation radio interferometers do not have sufficient sensitivity
to directly image the structures of the elusive EoR signal,
thus statistically studies via measuring the power spectra seem to be
a more feasible and promising way.
The redshifted 21~cm EoR signal expected to be observed at different
frequencies represent a fully three-dimensional (3D) data cube, where the
two angular dimensions describe the transverse distances on the sky and
the spectral dimension maps to the line-of-sight distance of the source.
Since the spatial distribution of the \Hi{} during the EoR is homogeneous
and isotropic \XU{什么尺度?},
the 3D power spectrum $P(\mathbf{k})$ of the EoR signal
will have spherical symmetry and can be averaged in spherical shells of
radius $k$, yielding the 1D power spectrum $P(k)$, which effectively
compresses the information in the \Hi{} field and greatly increases the
signal-to-noise ratio (SNR) over the per-pixel SNR in a direct imaging
observation \citep{morales2004,morales2006,datta2010}.
However, the foreground continuum emissions are only isotropic among the
two angular dimensions, which are very different from the frequency
dimension.
It is more appropriate to average the 3D power spectrum $P(\mathbf{k})$
over angular annuli of constant $\kperp \equiv \sqrt{k_x^2 + k_y^2}$
for each line-of-sight plane $\klos \equiv k_z$, yielding the 2D power
spectrum $P(\kperp, \klos)$.
The spectral-smooth foreground emissions are supposed to reside in the
low-\klos{} region on the $(\kperp, \klos)$ plane,
but complicated instrumental and observational effects (e.g., chromatic
primary beams and side-lobes, calibration errors) throw the strong
foreground contamination from the purely angular (\kperp) modes into the
line-of-sight (\klos) dimension, which are collectively called
\enquote{mode mixing} and result in an expanded wedge-like contamination
region at the bottom-right of the 2D power spectrum, i.e., the
\enquote{foreground wedge} \citep[e.g.,][]{datta2010,morales2012,liu2014}.
Nevertheless, a region relatively free of foregrounds,
the \enquote{EoR window}, is preserved at the top-left of the 2D power
spectrum, and it can be characterized by \citep{thyagarajan2013}
\begin{equation}
  \label{eq:eor-window}
  \klos \geq \frac{H_0 E(z) D_{\!M}(z)}{c (1+z)} \left[
    \kperp \sin\Theta +
    \frac{e}{B} \frac{2\pi f_{21}}{(1+z) D_{\!M}(z)} \right],
\end{equation}
where
$c$ is the speed of light,
$B = \SI{8}{\MHz}$ is the frequency bandwidth of our simulated image cube,
$f_{21} = \SI{1420.4}{\MHz}$ is the rest-frame frequency of the 21~cm line,
$z = f_{21}/f_c - 1$ is the signal redshift corresponding to the central
frequency of the image cube,
$E(z) = \sqrt{\Omega_m(1+z)^3 + \Omega_{\Lambda}}$ is the redshift
evolution factor \citep{hogg1999},
$D_{\!M}(z)$ is the transverse comoving distance,
$e$ denotes the number of characteristic convolution widths
($\propto B^{-1}$) for the spillover region caused by the variations in
instrumental frequency response,
and $\Theta$ is the angular distance of the foreground source from
the field center.
A soft limit to the region of the foreground wedge is thus imposed by the
telescope's field of view (\fov), i.e., $\Theta \simeq \fov/2$
\citep{morales2012}.
In this work, we choose $e = 3$ and $\Theta_{158} = \SI{6}{\degree}$
(corresponding to about 3 times of the SKA1-Low's \fov{} at
\SI{158}{\MHz}) for a very conservative EoR window boundary.

\begin{figure*}
  \centering
  \includegraphics[width=0.9\textwidth]{ps2d-band158}
  \caption{\label{fig:ps2d}%
    2D power spectra of
    \textbf{(a)} the EoR signal,
    \textbf{(b)} radio halos (the typical example),
    \textbf{(c)} Galactic diffuse radiation consisting of the synchrotron
    and free-free emissions,
    and
    \textbf{(d)} extragalactic point sources
    among the frequency band of \SIrange{154}{162}{\MHz}.
    All sub-figures share the same logarithmic scale in units of
    [\si{\mK\squared\Mpc\cubed}].
    The magenta dashed lines mark the boundary between the EoR window
    (at the top left) and the contaminating wedge (at the bottom right).
  }
\end{figure*}

In order to suppress the side-lobes resulted from the Fourier transform of
the finite frequency bandwidth, the Blackman-Nuttall window function is
applied along the frequency dimension \citep[e.g.,][]{trott2015,chapman2016}.
The 2D power spectra of radio halos [the typical example as shown in
\autoref{fig:halos-simucomp-example}(b)], the EoR signal, and other
foreground components among the \SIrange{154}{162}{\MHz} frequency band
are presented in \autoref{fig:ps2d},
where we also mark the adopted EoR window boundary.
The EoR signal distributes its power across all \klos{} modes, reflecting
its rapid fluctuations along the line-of-sight dimension,
while the spectral-smooth foregrounds dominate the low-\klos{} regions
of $\klos{} \lesssim \SI{0.2}{\per\Mpc}$.
With regard to the \kperp{} (i.e., angular) dimension,
radio halos and the EoR signal cover the similar intermediate angular
scales of $\kperp \lesssim \SI{1.0}{\per\Mpc}$,
while the Galactic diffuse emissions and extragalactic point sources
dominate the low-\kperp{} (large angular scales) and high-\kperp{}
(small angular scales) regions, respectively.
Specifically, radio halos mainly appear among scales of
$\SI{0.2}{\per\Mpc} \lesssim \kperp \lesssim \SI{1.0}{\per\Mpc}$,
corresponding to about \SIrange[range-units=repeat]{2.4}{12}{\arcminute},
which are the typical angular sizes of radio halos.
The wedge-shaped contamination, which is caused by the mode mixing effect
and localized in the high-\kperp{}, low-\klos{} regions on the 2D power
spectrum, is very evident for radio halos and the extragalactic point
sources, as shown in \autoref{fig:ps2d}(b,d).
Although a very conservative EoR window boundary is chosen, radio halos
still show some power beyond the boundary and hence can contaminate the
EoR signal even inside the EoR window.

\begin{figure*}
  \centering
  \includegraphics[width=\textwidth]{ps2d-halos-ratio-band158}
  \caption{\label{fig:ps2d-ratio}%
    2D power spectrum ratios of radio halos (the typical example) with
    respect to
    \textbf{(a)} the EoR signal,
    \textbf{(b)} the Galactic diffuse emission,
    and
    \textbf{(c)} the extragalactic point sources.
    All sub-figures share the same colorbar with logarithmic scale.
    The white dashed lines mark the same EoR window boundary as in
    \autoref{fig:ps2d}.
  }
\end{figure*}

For the purpose of better illustrating the relative importance of radio
halos, we divide the 2D power spectrum of radio halos (the typical
example) by those of the other components \citep{trott2015} and present
the obtained 2D power spectrum ratios in \autoref{fig:ps2d-ratio}.
It is evident in \autoref{fig:ps2d-ratio}(a) that radio halos severely
contaminate a large wedge-shaped region on the $(\kperp, \klos)$ plane,
which also suggests that a conservative EoR window boundary as we choose
may be required for the practical EoR detection in order to effectively
avoid the overwhelming foreground contamination, even though it will
exclude more Fourier modes.
The power ratio of radio halos to the EoR signal varies from about
\SI{5}{\percent} to about \SI{50}{\percent} within the EoR window and
at angular scales of
$\SI{0.2}{\per\Mpc} \lesssim \kperp \lesssim \SI{1.0}{\per\Mpc}$,
the characteristic scales of radio halos,
and the power ratio can reach a few and even thousands near the EoR
window boundary and inside the wedge, respectively.
Moreover, radio halos obviously dominate the intermediate and small
scales ($\kperp \gtrsim \SI{0.2}{\per\Mpc}$) over the Galactic diffuse
foreground [\autoref{fig:ps2d-ratio}(b)],
and their power at the characteristic scales can reach about
\SI{1}{\percent} of that of extragalactic point sources
[\autoref{fig:ps2d-ratio}(c)].
Consequently, we argue that radio halos can typically imprint a
non-negligible contamination upon the reliable measurements of the
21~cm signal, even inside the EoR window where the strong foreground
contamination is effectively avoided.

%========================================================================
\subsection{1D Power Spectra and Distribution Fluctuation}
\label{sec:ps1d}
%========================================================================

The 1D power spectrum is derived from the 2D power spectrum by averaging
over constant $k = \sqrt{\kperp^2 + \klos^2}$, which has the advantage of
allowing to effectively avoid the contamination by excluding the
foreground wedge.
Since the distributions of radio halos vary remarkably across the sky,
we take this into account by making use of all the 100 simulations for
radio halos (\autoref{sec:cluster-halo}) as an effort to achieve a
comprehensive analysis on their contributions to the low-frequency sky.
The 1D power spectra obtained without applying the EoR window constraint
are presented in \autoref{fig:ps1d}(a), which shows that radio halos are
stronger than the EoR signal by about 1 to 5 orders of magnitude
throughout all scales within the \SI{68}{\percent} confidence interval
(c.i.), with the median power being about 1000 times stronger than the
EoR signal.
Compared to the other foregrounds, radio halos are generally about
\numrange{10}{100} times larger than the Galactic diffuse emission at
small scales ($k \gtrsim \SI{0.4}{\per\Mpc}$),
and the power ratio of radio halos to the extragalactic point sources
is generally only $\lesssim \SI{0.1}{\percent}$ on scales of
$\SI{0.2}{\per\Mpc} \lesssim k \lesssim \SI{1.0}{\per\Mpc}$,
but can reach about a few percents within the \SI{68}{\percent}
confidence interval.

\begin{figure*}
  \centering
  \includegraphics[width=0.9\textwidth]{ps1d-band158}
  \caption{\label{fig:ps1d}%
    1D power spectra of radio halos (gray), the Galactic diffuse
    emission (blue), the extragalactic point sources (purple), and the
    EoR signal (green) derived from the 2D power spectra among the
    \SIrange{154}{162}{\MHz} band.
    \textbf{(a)} Use all modes without applying the EoR window constraint;
    \textbf{(b)} Only use the modes inside the EoR window.
    The gray shaded regions represent the \SI{68}{\percent} confidence
    interval indicating the large fluctuation of radio halos across the
    sky, and the gray solid lines indicate the median power spectra.
    The power spectra of radio halos for the typical example are shown as
    red solid lines, which agree with the median power spectra well.
    The magenta hatched region does not appear on the right sub-figure,
    because those modes are excised along with the foreground wedge.
  }
\end{figure*}

Furthermore, we calculate the 1D power spectra by only using the modes
inside the chosen EoR window, which can effectively avoid the foreground
contamination.
As illustrated in \autoref{fig:ps1d}(b), the excision of wedge region
has little effect on the power spectrum of the EoR signal, but reduces
the powers of all the foreground components remarkably as we would expect.
Nevertheless, the power ratio of radio halos to the EoR signal is
typically about \SIrange[range-units=repeat]{1}{10}{\percent} on scales
of $\SI{0.2}{\per\Mpc} \lesssim k \lesssim \SI{1.0}{\per\Mpc}$,
and can be about \SIrange[range-units=repeat]{10}{100}{\percent} within
the \SI{68}{\percent} confidence interval.
It should be noted that the power spectrum of the Galactic diffuse
emission showing a peak at $k \sim \SI{1.0}{\per\Mpc}$ is caused by the
side-lobes of Fourier transforming the image cube along the finite
frequency dimension.
Although the Galactic foreground is very spectral-smooth, and the
Blackman-Nuttall window function is also applied, the side-lobes of Fourier
transform can still be strong compared to the faint EoR signal.

Consequently, we argue that radio halos are worth a serious treatment
in the forthcoming deep surveys aiming at the EoR signal.
Given the remarkable fluctuation of radio halos, it would be advantageous
to identify the sky regions with as less radio halos and bright point
sources as possible for the future deep EoR observations that will be
carried out by the SKA1-Low as well as other radio interferometers
(Zheng et al. in preparation).
We also note that the power spectra of radio halos for the typical
example as analyzed in \autoref{sec:ps2d} are marked in \autoref{fig:ps1d},
and they only slightly deviate from the median power spectra, which
further confirms that the results presented there are representative.


%########################################################################
\section{Discussions}
\label{sec:discussions}
%########################################################################

In addition to the focused frequency band around \SI{158}{\MHz}, we
discuss the impacts of radio halos at lower and higher bands centered
at \SI{124}{\MHz} and \SI{196}{\MHz}, respectively.
We also discuss how the power spectra may be affected by the instrumental
frequency structures which could be caused by the small calibration
uncertainties as well as other complicated instrumental and observational
effects.

%========================================================================
\subsection{Lower and Higher Frequency Bands}
\label{sec:freq-bands}
%========================================================================

In order to characterize the complete reionization process, it is crucial
to measure the redshifted 21~cm signal among a broad frequency band of
about \SIrange{100}{200}{\MHz}, and all current- and next-generation
interferometers targeting the EoR measurements were designed to have
sufficient frequency coverage.
Here we choose another lower frequency band of \SIrange{120}{128}{\MHz}
and a higher band of \SIrange{192}{200}{\MHz}, which are commonly studied
in the literature as well, and follow the same procedures as described
in Sections \ref{sec:sky-simu} and \ref{sec:obs-simu} to investigate the
contributions of radio halos at these different frequencies.
At \SI{124}{\MHz}, the middle of the former band, the mean (r.m.s.)
brightness temperatures of the simulated sky maps for radio halos (the
typical example) and the EoR signal are \SI{58.1}{\mK} (\SI{8891}{\mK})
and \SI{13.5}{\mK} (\SI{14.4}{\mK}), respectively.
Meanwhile, the mean (r.m.s.) brightness temperatures at \SI{196}{\MHz}
for the two components are \SI{11.6}{\mK} (\SI{1861}{\mK})
and \SI{2.48}{\mK} (\SI{3.44}{\mK}), respectively.
In \autoref{fig:ps2d-ratio-bands}, we present the 2D power spectrum
ratios of radio halos with respect to the EoR signal at these two
frequency bands, and we use $\Theta_{124} = \SI{7.5}{\degree}$ for the
lower band and $\Theta_{196} = \SI{4.8}{\degree}$ for the higher band
to determine the EoR windows since $\fov \propto f^{-1}$ with $f$ being
the observing frequency.
Inside the EoR window and on the same scales of
$\SI{0.2}{\per\Mpc} \lesssim \kperp \lesssim \SI{1.0}{\per\Mpc}$,
the power ratio of radio halos to the EoR signal can be about
\SIrange[range-units=repeat]{50}{500}{\percent} at the lower frequency
band and varies among about \SIrange[range-units=repeat]{1}{10}{\percent}
at the higher frequency band.
Together with the results at the \SIrange{154}{162}{\MHz} band
(\autoref{sec:ps2d}), these indicate that radio halos become a more
important foreground component at lower frequencies.

\begin{figure*}
  \centering
  \includegraphics[width=0.8\textwidth]{ps2d-halos-ratio-bands}
  \caption{\label{fig:ps2d-ratio-bands}%
    2D power spectrum ratios of radio halos (the typical example) with
    respect to the EoR signal among frequency bands of
    \textbf{(a)} \SIrange{120}{128}{\MHz} and
    \textbf{(b)} \SIrange{192}{200}{\MHz}, respectively.
    The colorbar is in logarithmic scale and shared by the
    two sub-figures.
    The white dashed lines mark the EoR window boundaries.
  }
\end{figure*}

\begin{figure*}
  \centering
  \includegraphics[width=0.9\textwidth]{ps1d-eorw-bands}
  \caption{\label{fig:ps1d-bands}%
    1D power spectra of radio halos (gray) and the EoR signal (green)
    among the
    \textbf{(a)} \SIrange{120}{128}{\MHz} and
    \textbf{(b)} \SIrange{192}{200}{\MHz} bands,
    with the EoR window constraint applied.
    The gray solid lines and shaded regions show the median and
    corresponding \SI{68}{\percent} confidence interval of power spectra
    of radio halos, respectively.
    The red solid lines stand for the power spectra of radio halos for
    the typical example.
  }
\end{figure*}

In order to account for the distribution fluctuation of radio halos, we
also repeat the radio halos simulation for 100 times in both frequency
bands.
The mean (r.m.s.) brightness temperatures of the 100 simulations at
\SI{124}{\MHz} and \SI{196}{\MHz} are
$63.3_{-49.5}^{+171.1}$ \si{\mK} ($6\,953_{-5\,041}^{+20\,986}$ \si{\mK})
and
$10.1_{-8.2}^{+35.7}$ \si{\mK} ($1\,235_{-921}^{+4\,859}$ \si{\mK}),
respectively.
We take advantage of the EoR windows as described above to calculate the
1D power spectra for radio halos and the EoR signal.
As shown in \autoref{fig:ps1d-bands}, at scale of $k \sim \SI{1.0}{\per\Mpc}$,
the power ratio of radio halos to the EoR signal is typically about
\SI{100}{\percent} at the lower frequency band and can exceed
\SI{2000}{\percent} by taking the \SI{68}{\percent} uncertainty into
account,
meanwhile the power ratio is generally $\lesssim \SI{2}{\percent}$ at
the higher frequency band and may reach $\lesssim \SI{100}{\percent}$
within the \SI{68}{\percent} confidence interval.

By aggregating all the results from the three frequency bands, is it
evident that radio halos impose more contamination on the EoR measurements
toward lower frequencies.
As the observing frequency decreases from about \SI{200}{\MHz} to about
\SI{100}{\MHz}, the EoR signal peaks at about \SIrange{120}{130}{\MHz}
\citep{mesinger2016} and then decreases, but radio halos keep being
brighter.
We therefore argue that radio halos need more serious treatment in
measuring the EoR signal, especially in the very low frequencies of
$\lesssim \SI{100}{\MHz}$.

%========================================================================
\subsection{Instrumental Frequency Structures}
\label{sec:freq-structures}
%========================================================================

The smoothness along the frequency dimension is the most crucial feature
of various foreground components, and is the key that enables us to
extract the faint EoR signal in the presence of overwhelming foreground
contamination.
However, artificial frequency structures may present in the final image
cubes due to residual calibration uncertainties and complicated
instrumental and observational effects (e.g., cable signal reflections,
ionospheric distortions), which break the spectral smoothness of the
foreground emission and hence damage the EoR measurements.
There are some simulation studies and observational analyses showing that
the chromatic instrumental response and insufficient foreground modeling
can lead to small calibration errors of about
\SIrange[range-units=repeat]{0.1}{1}{\percent} in frequency channels
\citep[e.g.,][]{barry2016,beardsley2016,ewallWice2017}.

\begin{figure*}
  \centering
  \includegraphics[width=0.9\textwidth]{ps1d-bands-crp}
  \caption{\label{fig:ps1d-bands-crp}%
    1D power spectra of radio halos and the EoR signal in the
    \SIrange{154}{162}{\MHz} band with frequency artifacts of different
    amplitudes being added.
    \textbf{(a)} Use all modes without applying the EoR window constraint;
    \textbf{(b)} Only use the modes inside the EoR window.
    The red and green lines show the original power spectra.
    The blue and yellow lines show the power spectra of image cubes with
    frequency artifact of \SI{0.1}{\percent}, and the corresponding
    shaded regions represent the \SI{68}{\percent} confidence intervals.
    The purple and olive lines and shaded regions show the power spectra
    and their \SI{68}{\percent} uncertainties with frequency artifact of
    \SI{1}{\percent} being added to the image cubes.
    The magenta hatched region does not appear on the right sub-figure
    due to those modes been excluded from the EoR window.
    Note the overlapping lines of the EoR signal in both panels and
    radio halos in the left panel, since the added frequency artifacts
    do not make a noticeable difference in these cases.
  }
\end{figure*}

To evaluate the influence on the power spectra caused by the frequency
artifacts, we make use of the \SIrange{154}{162}{\MHz} image cubes by
multiplying each slice by a random number drawn from a Gaussian
distribution with unity mean and then compare the resulting power
spectra \citep{chapman2016}.
Two cases are investigated: one is to mimic a frequency artifact of
amplitude \SI{0.1}{\percent} by using $\sigma = 0.001$ for the Gaussian
distribution, the other is a frequency artifact of \SI{1}{\percent}
by using $\sigma = 0.01$.
We make use of the image cubes of the EoR signal and the typical example
of radio halos, add frequency artifacts, and calculate the 1D power
spectra both with and without the EoR constraint applied.
This procedure is repeated for 100 times in order to take into account
the uncertainties that the frequency artifacts may cause, and the results
are presented in \autoref{fig:ps1d-bands-crp}.
The power spectra of the EoR signal do not exhibit obvious changes
regardless of the amplitudes of added frequency artifacts being either
\SI{0.1}{\percent} or \SI{1}{\percent} as well as whether the EoR window
constraint is applied, since the EoR signal already fluctuates greatly
along the frequency dimension.
Without applying the EoR window constraint, the power spectra of radio
halos remain almost unchanged in cases of the amplitudes of frequency
artifacts being either \SI{0.1}{\percent} or \SI{1}{\percent}.
The reason is that the power of radio halos mostly distributes in the
wedge region dominated by the angular (\kperp) modes, thus the extra
frequency (\klos) power resulted from the added frequency artifacts
cannot make a noticeable difference to the final averaged power.
By excising the wedge region, however, a frequency artifact of
\SI{1}{\percent} can increase the power spectrum of radio halos by
more than 10 times, which makes radio halos even stronger than the
EoR signal at scales of $k \sim \SI{1.0}{\per\Mpc}$.
If the amplitude of the frequency artifact reduces to \SI{0.1}{\percent},
the power spectrum of radio halos increases only slightly, which may be
acceptable and will not add major difficulties to the EoR measurements.
Considering that radio halos are about \SIrange{2}{3} orders of magnitude
weaker than the extragalactic point sources (\autoref{fig:ps1d}),
the results are consistent with the findings of \citet{barry2016},
who recommended that EoR instruments should have no spectral features
of amplitudes $\gtrsim \num{e-5}$ and scales $\lesssim \SI{8}{\MHz}$
in the antenna or receiver system.
Our results also suggest that even minor calibration uncertainties can
make radio halos become more problematic to the EoR detection, therefore
a careful treatment for radio halos can be valuable to achieve better
calibrations and thus help probe the EoR.


%########################################################################
\section{Summary}
\label{sec:summary}
%########################################################################

In this work, we have simulated the contribution of radio halos in galaxy
cluster to the low-frequency sky, and quantitatively evaluated the impacts
on the EoR measurements imposed by radio halos.
We have developed the \texttt{FG21sim} software to simulate the low-frequency
radio sky, including the contributions of the Galactic synchrotron and
free-free emission, extragalactic point sources, and radio halos in
galaxy clusters.
Among these, we significantly improved the modeling of radio halos by
employing the merger-induced turbulence re-acceleration model, which can
fully describe the emergence and evolution of radio halos.
The model parameters were further calibrated by utilizing all currently
observed radio halos, achieving good agreement between the simulation
results and observation data.
Adopting the latest SKA1-Low layout configuration, we input the simulated
sky maps into \texttt{OSKAR} to perform observational simulation, which
were then imaged through \texttt{WSClean}, yielding the \enquote{observed}
images with the instrumental effects properly incorporated.

Focused on the \SIrange{154}{162}{\MHz} frequency band, we calculated
and compared the 2D and 1D power spectra of radio halos, the EoR signal,
the Galactic diffuse emission, and extragalactic point sources.
The 2D power spectrum of radio halos shows clearly the wedge-shaped
contaminating region, dominating the intermediate and large \kperp{}
modes ($\SI{0.2}{\per\Mpc} \leq \kperp \leq \SI{1.0}{\per\Mpc}$,
corresponding to \SIrange[range-units=repeat]{2.4}{12}{\arcminute})
over the Galactic diffuse emission, and reaches several percents of the
power of point sources at those scales.
If we derive the 1D power spectra by naively averaging over all the
available modes, radio halos can typically be about 3 order of magnitude
stronger than the EoR signal due to the serious contamination from the
foreground wedge.
%% rewrite/improve...
By applying the EoR window constraint, only the modes inside the EoR
window are used to derive the 1D power spectra.
While the powers of the EoR signal barely change, the powers of radio
halos are substantially reduced by about 4 orders of magnitude, hence
only has limited influence on the EoR measurements.
However, radio halos vary significantly across the sky, leading to
large fluctuations in the contamination to the EoR window, which can be
as strong as or even stronger than the target signal, therefore
selecting clean sky regions avoiding radio halos and bright point sources
as much as possible for the future deep EoR observations is of great
importance (Zheng et al. in preparation).

We also investigated in the lower \SIrange{120}{128}{\MHz} and the higher
\SIrange{192}{200}{\MHz} frequency bands, and found that radio halos
produced severer contamination to the EoR measurements at the lower
frequencies, because the EoR signal peak at about \SIrange{120}{130}{\MHz},
while radio halos continue to be brighter toward lower frequencies.
The contamination situation by radio halos at the higher
\SIrange{192}{200}{\MHz} is similar to the situation at
\SIrange{154}{162}{\MHz}.
Moreover, the possible frequency structures resulting from calibration
errors and instrumental and observational effects were discussed by
adding variations to the image cubes along the frequency dimension.
Utilizing the 1D power spectra derived inside the EoR window, the added
frequency variations have little influence on the EoR signal, but increase
the powers of radio halos by about 10 times (\SI{1}{\percent} frequency
variations), imposing a serious contamination to the EoR measurements.
Our results show that frequency variations of amplitude \SI{0.1}{\percent}
is acceptable for radio halos to not cause significant difficulties in
EoR measurements.

Nevertheless the EoR window provide a powerful method to avoid the
overwhelming foreground and lead us to a promising prospect in EoR
measurements, the foreground avoidance method loses many Fourier modes
which account for a large portion of the EoR powers, restricting the
measurable EoR modes and compromising the measurement precision.
Therefore, sophisticated foreground modeling and removal methods are
still an active research field.
Among the various foreground components, we argue that radio halos
should be dealt with seriously for the forthcoming deep EoR surveys.
Better understanding and characterizing various foreground components
help subtract them more accurately, expanding the EoR window and
revealing more modes for more robust EoR measurements.

%%%%%%%%%%%%%%%%%%%%%%%%%%%%%%%%%%%%%%%%%%%%%%%%%%%%%%%%%%%%%%%%%%%%%%%%%
\acknowledgments

We would give thanks
to Fred Dulwich for sharing the latest SKA1-Low layout configuration as well
as the guidance on using the \texttt{OSKAR} simulator,
to Andr\'e Offringa for the help on interferometric imaging with
\texttt{WSClean},
to Mathieu Remazeilles for providing us with the high-resolution Haslam
\SI{408}{\MHz} Galactic synchrotron map,
and to Giovanna Giardino for the all-sky Galactic synchrotron spectral index
map.
We also acknowledge Emma Chapman and Uri Keshet for their help.
Part of the work involving \texttt{OSKAR} and \texttt{WSClean} was performed
on the high-performance cluster at Shanghai Astronomical Observatory, Chinese
Academy of Sciences.
This work is supported by the National Natural Science Foundation of China
(grant Nos. 11433002, 11621303, 61371147, \TODO{???}),
and the National Key Research and Discovery Plan (grant No. 2017YFF0210903).


%% To help institutions obtain information on the effectiveness of their
%% telescopes the AAS Journals has created a group of keywords for telescope
%% facilities:
%% http://journals.aas.org/authors/aastex/facility.html
%
%% Following the acknowledgments section, use the following syntax and the
%% \facility{} or \facilities{} macros to list the keywords of facilities used
%% in the research for the paper.  Each keyword is check against the master
%% list during copy editing.  Individual instruments can be provided in
%% parentheses, after the keyword, but they are not verified.

\vspace{5mm}

% \facilities{???}

%% Similar to \facility{}, there is the optional \software command to allow
%% authors a place to specify which programs were used during the creation of
%% the manuscript. Authors should list each code and include either a
%% citation or url to the code inside ()s when available.

\software{
  OSKAR \citep{mort2010},
  WSClean \citep{offringa2014},
  AstroPy \citep{astropy2013},
  HEALPix \citep{gorski2005},
  HMF \citep{murray2013},
  IPython (\url{https://ipython.org/}),
  Matplotlib (\url{https://matplotlib.org/}),
  NumPy (\url{http://www.numpy.org/}),
  SciPy (\url{https://scipy.org/}),
  Pandas (\url{https://pandas.pydata.org/}).
}

%%%%%%%%%%%%%%%%%%%%%%%%%%%%%%%%%%%%%%%%%%%%%%%%%%%%%%%%%%%%%%%%%%%%%%%%%
%%
%% Appendix material should be preceded with a single \appendix command.
%% There should be a \section command for each appendix. Mark appendix
%% subsections with the same markup you use in the main body of the paper.
%%
%% Each Appendix (indicated with \section) will be lettered A, B, C, etc.
%% The equation counter will reset when it encounters the \appendix
%% command and will number appendix equations (A1), (A2), etc. The
%% Figure and Table counter will not reset.
%%
\appendix

%########################################################################
\section{Collection of Current Observed Radio Halos}
\label{sec:halos-collection}
%########################################################################

\begin{longrotatetable}
\begin{deluxetable*}{lcccr@{$\,\pm\,$}lr@{$\,\pm\,$}lll}
\tabletypesize{\scriptsize}
\tablecaption{Collection of Current Observed Radio Halos (As of 2018 January)%
  \label{tab:halos-observed}%  % NOTE: must be placed inside the caption
}
\tablehead{
  \colhead{Cluster} &
  \colhead{Redshift} &
  \colhead{kpc/\si{\arcsecond}} &
  \colhead{Size} &
  \multicolumn{2}{c}{$S_{\SI{1.4}{\GHz}}$} &
  \multicolumn{2}{c}{$P_{\SI{1.4}{\GHz}}$} &
  \colhead{Notes} &
  \colhead{References} \\
  \colhead{} &  % cluster
  \colhead{} &  % redshift
  \colhead{} &  % kpc/arcsec
  \colhead{(\si{\Mpc})} &  % size
  \multicolumn{2}{c}{(\si{\mJy})} &  % flux
  \multicolumn{2}{c}{(\SI{e24}{\watt\per\hertz})} &  % power
  \colhead{} &  % notes
  \colhead{} \\
  \colhead{(1)} &
  \colhead{(2)} &
  \colhead{(3)} &
  \colhead{(4)} &
  \multicolumn{2}{c}{(5)} &
  \multicolumn{2}{c}{(6)} &
  \colhead{(7)} &
  \colhead{(8)}
}

\startdata
% Name                 z        kpc/"  size    S1.4   Serr   P1.4     Perr   Notes  Ref
1E 0657$-$56         & 0.2960 & 4.38 & 1.48 &  78.0 &  5.0 & 21.33 &  1.49 &  & \citet{liang2000}  \\
Abell 141            & 0.2300 & 3.64 & 1.20 &   1.3 &  0.1\tablenotemark{a} &  0.25 &  0.02 &  & \citet{duchesne2017}  \\
Abell 209            & 0.2060 & 3.34 & 1.40 &  16.9 &  1.0 &  2.04 &  0.12 & With a relic candidate & \citet{giovannini2009}  \\
Abell 399            & 0.0718 & 1.35 & 0.57 &  16.0 &  2.0 &  0.20 &  0.03 & Double with Abell 401 & \citet{murgia2010}  \\
Abell 401            & 0.0737 & 1.38 & 0.49 &  17.0 &  1.0 &  0.20 &  0.01 & Double with Abell 399 & \citet{bacchi2003}  \\
Abell 520            & 0.1990 & 3.25 & 0.99 &  34.4 &  1.5 &  3.17 &  0.14 &  & \citet{govoni2001}  \\
Abell 521            & 0.2533 & 3.91 & 1.17 &   5.9 &  0.5 &  1.12 &  0.09 & With a relic & \citet{giovannini2009}  \\
Abell 523            & 0.1000 & 1.82 & 1.30 &  59.0 &  5.0 &  1.47 &  0.12 &  & \citet{giovannini2011}  \\
Abell 545            & 0.1540 & 2.64 & 0.81 &  23.0 &  1.0 &  1.25 &  0.05 &  & \citet{bacchi2003}  \\
Abell 665            & 0.1818 & 3.03 & 1.66 &  43.1 &  2.2 &  3.28 &  0.17 &  & \citet{giovannini2000}  \\
Abell 697            & 0.2820 & 4.23 & 0.75 &   5.2 &  0.5 &  2.20 &  0.21 &  & \citet{vanWeeren2011}  \\
Abell 746            & 0.2320 & 3.67 & 0.85 &  18.0 &  4.0 &  3.80 &  0.84 & With a relic & \citet{vanWeeren2011}  \\
Abell 754            & 0.0542 & 1.04 & 0.95 &  86.0 &  4.0 &  0.56 &  0.03 & With a relic & \citet{bacchi2003}  \\
Abell 773            & 0.2170 & 3.48 & 1.13 &  12.7 &  1.3 &  1.39 &  0.14 &  & \citet{govoni2001}  \\
Abell 781            & 0.3004 & 4.42 & 1.60 &  20.5 &  5.0 &  5.90 &  1.44 & With a relic candidate & \citet{govoni2011}  \\
Abell 800            & 0.2223 & 3.55 & 1.28 &  10.6 &  0.9 &  1.52 &  0.13 &  & \citet{govoni2012}  \\
Abell 851            & 0.4069 & 5.40 & 1.08 &   3.7 &  0.3 &  2.14 &  0.17 &  & \citet{giovannini2009}  \\
Abell 1132           & 0.1369 & 2.39 & 0.74 &   3.3 &  1.5 &  0.16 &  0.07 &  & \citet{wilber2018}  \\
Abell 1213           & 0.0469 & 0.91 & 0.22 &  72.2 &  3.5 &  0.36 &  0.02 &  & \citet{giovannini2009}  \\
Abell 1300           & 0.3100 & 4.52 & 0.92 &  20.0 &  2.0 &  2.99 &  0.30 & With a relic & \citet{reid1999}  \\
Abell 1351           & 0.3220 & 4.64 & 1.08 &  32.4 &  XXX & 11.37 &  XXX  &  & \citet{giacintucci2011b}  \\
Abell 1443           & 0.2700 & 4.10 & 1.10 &  11.0 &  1.1\tablenotemark{b} &  2.53 &  0.30 & Halo candidate & \citet{bonafede2015}  \\
Abell 1451           & 0.1989 & 3.25 & 0.74 &   5.4 &  0.5 &  0.62 &  0.07 & With a relic candidate & \citet{cuciti2018}  \\
Abell 1550           & 0.2540 & 3.92 & 1.41 &   7.7 &  1.6 &  1.49 &  0.31 &  & \citet{govoni2012}  \\
Abell 1656           & 0.0232 & 0.46 & 0.58 & 530.0 & 50.0 &  0.31 &  0.03 & With a relic candidate & \citet{kim1990}  \\
Abell 1682           & 0.2272 & 3.61 & 0.85 &   2.3 &  0.5\tablenotemark{c} &  0.41 &  0.08 & Halo candidate & \citet{macario2013}  \\
Abell 1689           & 0.1832 & 3.05 & 0.73 &  10.0 &  2.9 &  0.92 &  0.27 &  & \citet{vacca2011}  \\
Abell 1758A          & 0.2790 & 4.20 & 0.63 &   3.9 &  0.4 &  0.93 &  0.10 & With a relic & \citet{giovannini2009}  \\
Abell 1914           & 0.1712 & 2.88 & 1.16 &  64.0 &  3.0 &  4.32 &  0.20 &  & \citet{bacchi2003}  \\
Abell 1995           & 0.3186 & 4.61 & 0.83 &   4.1 &  0.7 &  1.35 &  0.23 &  & \citet{giovannini2009}  \\
Abell 2034           & 0.1130 & 2.03 & 0.60 &   7.3 &  2.0 &  0.28 &  0.08 & With a relic & \citet{vanWeeren2011}  \\
Abell 2061           & 0.0784 & 1.46 & 1.68 &  16.9 &  4.2 &  0.25 &  0.06 & With a relic & \citet{farnsworth2013}  \\
Abell 2065           & 0.0726 & 1.36 & 1.08 &  32.9 & 11.0 &  0.41 &  0.14 &  & \citet{farnsworth2013}  \\
Abell 2069           & 0.1160 & 2.08 & 0.90 &   6.2 &  2.2\tablenotemark{d} &  0.25 &  0.05 & Halo may have two components & \citet{drabent2015}  \\
Abell 2142           & 0.0909 & 1.67 & 0.99 &  11.8 &  0.8 &  1.12 &  0.08 & Halo has two components & \citet{venturi2017}  \\
Abell 2163           & 0.2030 & 3.31 & 2.04 & 155.0 &  2.0 & 14.93 &  0.20 & With a relic & \citet{feretti2001}  \\
Abell 2218           & 0.1710 & 2.88 & 0.35 &   4.7 &  0.1 &  0.32 &  0.01 &  & \citet{giovannini2000}  \\
Abell 2219           & 0.2256 & 3.59 & 1.54 &  81.0 &  4.0 &  9.72 &  0.48 &  & \citet{bacchi2003}  \\
Abell 2254           & 0.1780 & 2.98 & 0.85 &  33.7 &  1.8 &  2.43 &  0.13 &  & \citet{govoni2001}  \\
Abell 2255           & 0.0806 & 1.50 & 0.90 &  56.0 &  3.0 &  0.87 &  0.05 & With a relic & \citet{govoni2005}  \\
Abell 2256           & 0.0594 & 1.13 & 0.81 & 103.4 &  1.1 &  0.82 &  0.01 & With a relic & \citet{clarke2006}  \\
Abell 2294           & 0.1780 & 2.98 & 0.54 &   5.8 &  0.5 &  0.51 &  0.04 &  & \citet{giovannini2009}  \\
Abell 2319           & 0.0524 & 1.01 & 0.93 & 153.0 &  8.0 &  0.54 &  0.03 &  & \citet{feretti1997}  \\
Abell 2680           & 0.1901 & 3.14 & 0.57 &   1.8 &  0.6\tablenotemark{e} &  0.16 &  0.05 & Halo candidate & \citet{duchesne2017}  \\
Abell 2693           & 0.1730 & 2.91 & 0.65 &   7.7 &  0.9\tablenotemark{f} &  0.61 &  0.07 & Halo candidate & \citet{duchesne2017}  \\
Abell 2744           & 0.3080 & 4.50 & 1.62 &  57.1 &  2.9 & 12.89 &  0.65 & With a relic & \citet{govoni2001}  \\
Abell 2811           & 0.1080 & 1.95 & 0.48 &   3.4 &  0.7\tablenotemark{g} &  0.10 &  0.02 &  & \citet{duchesne2017}  \\
Abell 3411           & 0.1687 & 2.85 & 0.90 &   4.8 &  0.5 &  0.46 &  0.05 & With a relic & \citet{vanWeeren2013}  \\
Abell 3562           & 0.0480 & 0.93 & 0.44 &  20.0 &  2.0 &  0.10 &  0.01 &  & \citet{venturi2003}  \\
Abell 3888           & 0.1510 & 2.60 & 0.99 &  27.6 &  3.1 &  1.85 &  0.19 &  & \citet{shakouri2016}  \\
Abell S84            & 0.1080 & 1.95 & 0.49 &   2.1 &  0.3\tablenotemark{h} &  0.06 &  0.01 & Halo candidate & \citet{duchesne2017}  \\
Abell S1121          & 0.3580 & 4.98 & 1.25 &   9.8 &  3.1\tablenotemark{h} &  4.54 &  1.44 &  & \citet{duchesne2017}  \\
ACT-CL J0102$-$4915  & 0.8700 & 7.73 & 2.17 &  10.7 &  1.1\tablenotemark{i} & 44.43 &  1.28 & With double relics & \citet{lindner2014}  \\
ACT-CL J0256.5+0006  & 0.3430 & 4.84 & 0.79 &   2.1 &  0.5\tablenotemark{i} &  0.97 &  0.29 &  & \citet{knowles2016}  \\
CIZA J0107.7+5408    & 0.1066 & 1.93 & 1.10 &  55.0 &  5.0 &  1.80 &  0.16 &  & \citet{vanWeeren2011}  \\
CIZA J0638.1+4747    & 0.1740 & 2.92 & 0.59 &   3.6 &  0.2 &  0.30 &  0.02 &  & \citet{cuciti2018}  \\
CIZA J1938.3+5409    & 0.2600 & 3.99 & 0.72 &   1.6 &  0.2\tablenotemark{b} &  0.36 &  0.05 &  & \citet{bonafede2015}  \\
CIZA J2242.8+5301    & 0.1921 & 3.16 & 1.77 &  33.5 &  6.2\tablenotemark{j} &  3.40 &  0.97 & With double relics & \citet{govoni2012}  \\
ClG 0016+16          & 0.5456 & 6.37 & 0.77 &   5.5 &  XXX &  4.42 &  XXX  &  & \citet{giovannini2000}  \\
ClG 0217+70          & 0.0655 & 1.24 & 0.73 &  58.6 &  0.9 &  0.54 &  0.01 & With double relics & \citet{brown2011}  \\
ClG 1446+26          & 0.3700 & 5.09 & 1.22 &   7.7 &  2.6 &  3.57 &  1.21 & With a relic & \citet{govoni2012}  \\
ClG 1821+64          & 0.2990 & 4.41 & 1.10 &  13.0 &  0.8\tablenotemark{k} &  3.70 &  0.10 &  & \citet{bonafede2014b}  \\
MACS J0416.1$-$2403  & 0.3960 & 5.31 & 0.64 &   1.7 &  0.8\tablenotemark{l} &  1.26 &  0.29 &  & \citet{pandeyPommier2015}  \\
MACS J0520.7$-$1328  & 0.3400 & 4.81 & 0.80 &   9.0 &  1.6 &  3.38 &  0.60 & Halo candidate & \citet{macario2014}  \\
MACS J0553.4$-$3342  & 0.4070 & 5.40 & 1.32 &   9.2 &  0.7\tablenotemark{b} &  6.73 &  0.61 &  & \citet{bonafede2012}  \\
MACS J0717.5+3745    & 0.5458 & 6.37 & 1.20 & 118.0 &  5.0 & 50.00 & 10.00 & With a relic & \citet{vanWeeren2009}  \\
MACS J0949.8+1708    & 0.3825 & 5.20 & 1.04 &   3.1 &  0.3\tablenotemark{b} &  1.63 &  0.15 &  & \citet{bonafede2015}  \\
MACS J1149.5+2223    & 0.5444 & 6.36 & 1.32 &   1.2 &  0.5 &  1.95 &  0.93 & Halo candidate; with double relics & \citet{bonafede2012}  \\
MACS J1752.0+4440    & 0.3660 & 5.05 & 1.65 &  14.2 &  1.4\tablenotemark{m} &  9.50 &  0.91 & With double relics & \citet{vanWeeren2012}  \\
MACS J2243.3$-$0935  & 0.4470 & 5.71 & 0.91 &   3.1 &  0.6\tablenotemark{n} &  3.11 &  0.58 & With a relic candidate & \citet{cantwell2016}  \\
PLCK G147.3$-$16.6   & 0.6500 & 6.92 & 1.80 &   2.5 &  0.4\tablenotemark{o} &  5.10 &  0.80 &  & \citet{vanWeeren2014}  \\
PLCK G171.9$-$40.7   & 0.2700 & 4.10 & 0.99 &  18.0 &  2.0 &  4.76 &  0.10 &  & \citet{giacintucci2013}  \\
PLCK G285.0$-$23.7   & 0.3900 & 5.26 & 0.73 &   2.9 &  0.4\tablenotemark{p} &  1.67 &  0.21 &  & \citet{martinezAviles2016}  \\
PLCK G287.0+32.9     & 0.3900 & 5.26 & 1.30 &   8.8 &  0.9 &  5.10 &  0.51 & With double relics & \citet{bonafede2014a}  \\
PSZ1 G108.18$-$11.53 & 0.3347 & 4.77 & 0.84 &   6.8 &  0.2 &  2.72 &  0.10 & With double relics & \citet{deGasperin2015}  \\
RXC J1234.2+0947     & 0.2290 & 3.63 & 0.92 &   2.0 &  XXX &  0.30 &  XXX  & Halo candidate & \citet{govoni2012}  \\
RXC J1314.4$-$2515   & 0.2474 & 3.85 & 1.27 &  20.3 &  0.8 &  1.45 &  0.06 & With double relics & \citet{feretti2005}  \\
RXC J1514.9$-$1523   & 0.2226 & 3.55 & 1.38 &  10.0 &  2.0 &  1.65 &  0.33 &  & \citet{giacintucci2011a}  \\
RXC J2003.5$-$2323   & 0.3171 & 4.59 & 1.38 &  35.0 &  2.0 & 11.96 &  0.68 &  & \citet{giacintucci2009}  \\
RXC J2351.0$-$1954   & 0.2477 & 3.85 & 0.64 &   4.5 &  0.9\tablenotemark{q} &  0.89 &  0.18 & Halo candidate & \citet{duchesne2017}  \\
\enddata

\tablenotetext{a}{Extrapolated from \SI{168}{\MHz} with spectral index $\alpha=2.1$.}
\tablenotetext{b}{Extrapolated from \SI{323}{\MHz} with spectral index $\alpha=1.3$.}
\tablenotetext{c}{Extrapolated from \SI{153}{\MHz} with spectral index $\alpha=1.7$.}
\tablenotetext{d}{Extrapolated from \SI{346}{\MHz} with spectral index $\alpha=1.0$.}
\tablenotetext{e}{Extrapolated from \SI{168}{\MHz} with spectral index $\alpha=1.2$.}
\tablenotetext{f}{Extrapolated from \SI{168}{\MHz} with spectral index $\alpha=0.88$.}
\tablenotetext{g}{Extrapolated from \SI{168}{\MHz} with spectral index $\alpha=1.5$.}
\tablenotetext{h}{Extrapolated from \SI{168}{\MHz} with spectral index $\alpha=1.3$.}
\tablenotetext{i}{Extrapolated from \SI{610}{\MHz} with spectral index $\alpha=1.2$.}
\tablenotetext{j}{Extrapolated from \SI{145}{\MHz} with spectral index $\alpha=1.03$.}
\tablenotetext{k}{Extrapolated from \SI{325}{\MHz} with spectral index $\alpha=1.04$.}
\tablenotetext{l}{Extrapolated from \SI{340}{\MHz} with spectral index $\alpha=1.5$.}
\tablenotetext{m}{Extrapolated from \SI{1714}{\MHz} with spectral index $\alpha=1.1$.}
\tablenotetext{o}{Extrapolated from \SI{610}{\MHz} with spectral index $\alpha=1.4$.}
\tablenotetext{p}{Extrapolated from \SI{1867}{\MHz} with spectral index $\alpha=1.3$.}
\tablenotetext{q}{Extrapolated from \SI{168}{\MHz} with spectral index $\alpha=1.4$.}
\tablecomments{\textbf{Columns:}
  \textbf{(1)} galaxy cluster name;
  \textbf{(2)} redshift;
  \textbf{(3)} \si{\kpc} per \si{\arcsec} at the cluster's redshift
               (converted to our adopted cosmology);
  \textbf{(4)} largest linear size of the observed radio halo, in units of \si{\Mpc};
  \textbf{(5)} measured flux density and the uncertainty at \SI{1.4}{\GHz};
  \textbf{(6)} \SI{1.4}{\GHz} radio power and the uncertainty
               (converted to our adopted cosmology);
  \textbf{(7)} extra notes;
  \textbf{(8)} references to the quoted halo properties.
}
\end{deluxetable*}
\end{longrotatetable}


%########################################################################
\section{Additional Formulas}
\label{sec:formulas}
%########################################################################

In a flat \lcdm{} cosmology as adopted in this work, the critical linear
overdensity as a function of redshift $z$ is \citep{kitayama1996,randall2002}
\begin{equation}
  \label{eq:delta-crit}
  \delta_c(z) = \frac{D(z=0)}{D(z)} \frac{3}{20} (12\pi)^{2/3}
    \left[1 + 0.0123 \log_{10} \Omega_f(z) \right],
\end{equation}
where $\Omega_f(z)$ is the mass density ratio at redshift $z$ defined as
\begin{equation}
  \label{eq:omega-fz}
  \Omega_f(z) = \frac{\Omega_m(1+z)^3}{\Omega_m(1+z)^3 + \Omega_{\Lambda}},
\end{equation}
and $D(z)$ is the growth factor given by \citep[Equation~(13.6)]{peebles1980}
\begin{equation}
  \label{eq:growth-factor}
  D(x) = \frac{(x^3 + 2)^{1/2}}{x^{3/2}}
    \int_0^x y^{3/2} (y^3 + 2)^{-3/2} \,\D{y},
\end{equation}
with $x_0 \equiv (2\Omega_{\Lambda}/\Omega_m)^{1/3}$ and $x = x_0 / (1+z)$.

The virial radius of a galaxy cluster is defined as
\begin{equation}
  \label{eq:radius-virial}
  r_{\R{vir}} = \left[
    \frac{3 M_{\R{vir}}}{4\pi \,\Delta_{\R{vir}}(z) \rho_{\R{crit}}(z)}
  \right]^{1/3},
\end{equation}
where $M_{\R{vir}}$ is the virial mass of the cluster,
$\rho_{\R{crit}}(z) = 3 H^2(z) / (8\pi G)$ is the critical density
at redshift $z$,
and $\Delta_{\R{vir}}(z)$ is the average overdensity of the cluster
at redshift $z$ which is given by \citep{kitayama1996,cassano2005}
\begin{equation}
  \label{eq:delta-vir}
  \Delta_{\R{vir}}(z) = 18\pi^2 \left[ 1 + 0.4093 \, w(z)^{0.9052} \right],
\end{equation}
where $w(z) \equiv \Omega_f^{-1}(z) - 1$.


%%%%%%%%%%%%%%%%%%%%%%%%%%%%%%%%%%%%%%%%%%%%%%%%%%%%%%%%%%%%%%%%%%%%%%%%%
\bibliography{references}


%% Include this line if you are using the \added, \replaced, \deleted
%% commands to see a summary list of all changes at the end of the article.
%\listofchanges

\end{document}

%% EOF
