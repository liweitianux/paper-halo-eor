%%
%% Copyright (c) 2017 The Authors.  All Rights Reserved.
%%
%% Weitian LI, Haiguang XU, et al.
%% Department of Astronomy, Shanghai Jiao Tong University
%%
%% 2017-07-18
%%

%%%%%%%%%%%
%% Preamble
%%%%%%%%%%%

\documentclass[modern]{aastex61}
%%
%% Defaults: single spaced, 10 point font article
%%
%%  twocolumn   : two text columns, 10 point font, single spaced article.
%%                This is the most compact and represent the final published
%%                derived PDF copy of the accepted manuscript from the publisher
%%  manuscript  : one text column, 12 point font, double spaced article.
%%  preprint    : one text column, 12 point font, single spaced article.
%%  preprint2   : two text columns, 12 point font, single spaced article.
%%  modern      : a stylish, single text column, 12 point font, article with
%%                wider left and right margins. This uses the Daniel
%%                Foreman-Mackey and David Hogg design.
%%
%% Note that you can submit to the AAS Journals in any of these 6 styles.
%%
%% There are other optional arguments one can envoke to allow other stylistic
%% actions. The available options are:
%%
%%  astrosymb    : Loads Astrosymb font and define \astrocommands.
%%  tighten      : Makes baselineskip slightly smaller, only works with
%%                 the twocolumn substyle.
%%  times        : uses times font instead of the default
%%  linenumbers  : turn on lineno package.
%%  trackchanges : required to see the revision mark up and print its output
%%  longauthor   : Do not use the more compressed footnote style (default) for
%%                 the author/collaboration/affiliations. Instead print all
%%                 affiliation information after each name. Creates a much
%%                 long author list but may be desirable for short author papers

%% Custom macros
\newcommand{\etal}{\textit{et al.}}
\newcommand{\D}[1]{\mathrm{d} #1}
\newcommand{\diff}[2]{\frac{\D{#1}}{\D{#2}}}
\newcommand{\pdiff}[2]{\frac{\partial #1}{\partial #2}}
\newcommand{\uG}{\mathrm{\mu G}}
\newcommand{\Msun}{\mathrm{M}_{\odot}}
\newcommand{\lcdm}{$\Lambda$CDM}


%% Journal information
\received{xxx, 2017}
\revised{xxx, 2017}
\accepted{xxx, 2017}
\submitjournal{ApJ}

\shorttitle{Low-frequency Radio Foregrounds Simulation}
\shortauthors{Li \etal}

%% You can add a light gray and diagonal water-mark to the first page
%% with this command:
\watermark{DRAFT}
%% where "text", e.g. DRAFT, is the text to appear.  If the text is
%% long you can control the water-mark size with:
%  \setwatermarkfontsize{dimension}
%% where dimension is any recognized LaTeX dimension, e.g. pt, in, etc.


%%%%%%%%%%%
%% Document
%%%%%%%%%%%

\begin{document}

\title{
  Low-frequency Radio Foregrounds Simulation:
  Extended Radio Emission from Galaxy Clusters
}

%% The \author command is the same as before except it now takes an optional
%% argument which is the 16 digit ORCID. The syntax is:
%% \author[xxxx-xxxx-xxxx-xxxx]{Author Name}
%%
%% This will hyperlink the author name to the author's ORCID page. Note that
%% during compilation, LaTeX will do some limited checking of the format of
%% the ID to make sure it is valid.
%%
%% Use \affiliation for affiliation information. The old \affil is now aliased
%% to \affiliation. AASTeX v6.1 will automatically index these in the header.
%% When a duplicate is found its index will be the same as its previous entry.
%%
%% Note that \altaffilmark and \altaffiltext have been removed and thus
%% can not be used to document secondary affiliations. If they are used latex
%% will issue a specific error message and quit. Please use multiple
%% \affiliation calls for to document more than one affiliation.
%%
%% The new \altaffiliation can be used to indicate some secondary information
%% such as fellowships. This command produces a non-numeric footnote that is
%% set away from the numeric \affiliation footnotes.  NOTE that if an
%% \altaffiliation command is used it must come BEFORE the \affiliation call,
%% right after the \author command, in order to place the footnotes in
%% the proper location.
%%
%% Use \email to set provide email addresses. Each \email will appear on its
%% own line so you can put multiple email address in one \email call. A new
%% \correspondingauthor command is available in V6.1 to identify the
%% corresponding author of the manuscript. It is the author's responsibility
%% to make sure this name is also in the author list.
%%
%% While authors can be grouped inside the same \author and \affiliation
%% commands it is better to have a single author for each. This allows for
%% one to exploit all the new benefits and should make book-keeping easier.
%%
%% If done correctly the peer review system will be able to
%% automatically put the author and affiliation information from the manuscript
%% and save the corresponding author the trouble of entering it by hand.

\correspondingauthor{Haiguang Xu}
\email{hgxu@sjtu.edu.cn}

\author[0000-0002-7527-380X]{Weitian Li}
\email{liweitianux@sjtu.edu.cn}
\affil{Department of Astronomy, Shanghai Jiao Tong University,
  800 Dongchuan Road, Shanghai 200240, China}

\author{Haiguang Xu}
\affil{Department of Astronomy, Shanghai Jiao Tong University,
  800 Dongchuan Road, Shanghai 200240, China}


%%
%% Abstract
%%

\begin{abstract}
TODO!!!
\end{abstract}

%% See the online documentation for the full list of available subject
%% keywords and the rules for their use.
%% http://journals.aas.org/authors/keywords2013.html
\keywords{
  dark ages, reionization, first stars ---
  galaxies: clusters: intracluster medium ---
  galaxies: halos ---
  radio continuum: galaxies
}

%%%%%%%
%% Body
%%%%%%%

\section{Introduction}
\label{sec:intro}

TODO!!!

Extended radio emissions are only detected in a fraction of massive and
luminous galaxy clusters, and they can be classified into several types
\citep[see][for recent reviews]{feretti2012rev,brunetti2014rev}:
\begin{description}
\item[halo]
  Giant radio halos extending on $\sim$Mpc scales;
  only appear in merging clusters;
\item[mini-halo]
  Radio mini-halos, smaller than the above giant halos (usually several
  hundreds of kpc); can be found in cool-core clusters;
\item[relic]
  Elongated relics, and roundish relics;
\end{description}

We adopt a flat \lcdm{} cosmology with
$H_0 = 100 h = 71 \,\mathrm{km}\,\mathrm{s}^{-1}\,\mathrm{Mpc}^{-1}$,
$\Omega_m = 0.27$, $\Omega_{\Lambda} = 1 - \Omega_m = 0.73$,
$\Omega_b = 0.046$, and $\sigma_8 = 0.81$ \citep{komatsu2011}.


\section{Foregrounds Simulation}
\label{sec:fg-simu}

TODO: OVERVIEW!!!

\subsection{Galactic Synchrotron Emission}
\label{sec:fg-gsync}

TODO!!!

\subsection{Galactic Free-free Emission}
\label{sec:fg-gfree}

TODO!!!

\subsection{Galactic Supernova Remnants}
\label{sec:fg-gsnr}

TODO!!!

Update to use a empirical model about the total number of Galactic SNRs
and their spatial distributions.  Meanwhile derive necessary scaling
relations from the current observed catalog maintained by Green.

\subsection{Clusters of Galaxies}
\label{sec:fg-clusters}

TODO: OVERVIEW!

In the framework of low-frequency radio foregrounds simulation for the
EoR signal detection, the most realistic consideration to the galaxy
clusters component until now is our work in \citeyear{wang2010} \citep{wang2010}.
In this work, we continue to improve our simulation method in several
ways:
(1) Predict radio halos by tracing the most recent major merger of clusters;
(2) Determine the halo brightness and spectrum by solving the time
evolution of relativistic electrons within the cluster ICM;
(3) Preliminary inclusion of radio relics (???).
We present our improved simulation method to the galaxy clusters in this
section.

\subsubsection{Distributions}
\label{sec:cluster-dist}

For a patch of sky, we first want to determine the mass and redshift
distributions of the galaxy clusters, which can be sampled from the
Press-Schechter (PS) mass function \citep{press1974}:
\begin{equation}
  \label{eq:ps-mass-func}
  n(M, z) \D{M} = \sqrt{\frac{2}{\pi}} \frac{\langle{\rho}\rangle}{M}
  \frac{\delta_c(z)}{\sigma^2(M)} \left| \diff{\sigma(M)}{M} \right|
  \exp\left[ -\frac{\delta_c^2(z)}{2\sigma^2{M}} \right] \D{M},
\end{equation}
where
$M$ is the cluster mass,
$\langle {\rho} \rangle$ is the current mean density of the Universe,
$\delta_c(z)$ is the critical linear overdensity for a region to collapse
at redshift $z$,
and $\sigma(M)$ is the current rms density fluctuations within a sphere
of mean mass $M$.

In cold dark matter (CDM) models and over the range of scales covered
by clusters, it is generally sufficient to consider a power-law spectrum
for the density perturbations \citep{sarazin2002,randall2002}:
\begin{equation}
  \label{eq:sigma-mass}
  \sigma(M) = \sigma_8 \left( \frac{M}{M_8} \right)^{-\alpha},
\end{equation}
where $\sigma_8$ is the present-day rms density fluctuations on a
scale of $8 h^{-1}$ Mpc,
$M_8 = (4\pi/3)(8 h^{-1} \,\mathrm{Mpc})^3 \langle{\rho}\rangle$
is the mass contained in a sphere of radius $8 h^{-1}$ Mpc.
The exponent $\alpha$ is given by $\alpha = (n+3)/6$, where
the power spectrum of fluctuations varies with wavenumber $k$
as $k^n$, and we adopt $n = -7/5$ for this work \citep{randall2002}.

We place a lower mass limit of $2 \times 10^{14} \,\Msun$ for clusters,
determine the expected total number of clusters within our simulation
sky patch, and then derive the mass and redshift for each cluster by
sampling from the PS mass function using Monte Carlo method
\citep{wang2010}.
The clusters are randomly distributed within the sky patch.

\subsubsection{Merging History}
\label{sec:merging-history}

Galaxy clusters are formed hierarchically by merging from smaller (i.e.,
lower mass) clusters, groups, and galaxies.
And the PS formalism provides a way to infer the formation history
of a cluster from its current status.
We follow the extended PS (EPS) method outlined by \citet{lacey1993} to
generate the merger trees.

The conditional probability that a ``parent'' cluster of mass $M_2$ at
a time $t_2$ had a ``progenitor'' of mass in the range
$M_1 \rightarrow M_1 + \D{M_1}$ at some earlier time $t_1$,
with $M_2 > M_1$ and $t_2 > t_1$, is given by
\citep{lacey1993,randall2002}:
\begin{equation}
  \label{eq:eps-condprob}
  P(M_1, t_1 | M_2, t_2) \D{M_1} = \frac{1}{\sqrt{2\pi}} \frac{M_2}{M_1}
  \frac{\delta_{c1} - \delta_{c2}}{(\sigma_1^2 - \sigma_2^2)^{3/2}}
  \left| \diff{\sigma_1^2}{M_1} \right|
  \exp \left[ -\frac{(\delta_{c1} - \delta_{c2})^2}
    {2(\sigma_1^2 - \sigma_2^2)} \right] \D{M_1},
\end{equation}
where
$\delta_{c1} \equiv \delta_c(t_1)$ and $\sigma_1 \equiv \sigma(M_1)$,
with similar definitions for $\delta_{c2}$ and $\sigma_2$.
By further defining $S \equiv \sigma^2(M)$ and $\omega \equiv \delta_c(t)$,
and let $K(\Delta S, \Delta \omega) \D{\Delta S}$ be the probability
that a cluster had a progenitor with a mass corresponding to a change
in $S$ of $\Delta S = \sigma_1^2 - \sigma_2^2$ in the range
$\Delta S \rightarrow \Delta S + \D{\Delta S}$ at an earlier time
corresponding to $\Delta \omega = \delta_{c1} - \delta_{c2}$,
then Equation~(\ref{eq:eps-condprob}) is simplified into:
\begin{equation}
  \label{eq:eps-condprob-simp}
  K(\Delta S, \Delta \omega) \D{\Delta S} = \frac{1}{\sqrt{2\pi}}
  \frac{\Delta\omega}{(\Delta S)^{3/2}}
  \exp \left[ -\frac{(\Delta\omega)^2}{2 \Delta S} \right] \D{\Delta S}.
\end{equation}

We employ the standard Monte Carlo technique to build the merger trees,
with each of them starts with a cluster with an initial mass $M_0$
and redshift $z_0$ simulated in the last section.
We trace each cluster back in time, using a step size $\Delta\omega$
adaptively determined by:
\begin{equation}
  \label{sec:dw-step}
  (\Delta\omega)_{\mathrm{step}} = \frac{1}{2} \sqrt{
    \left| \diff{\ln \sigma^2}{\ln M} \right|
    \left( \frac{\Delta M_c}{M} \right) S},
\end{equation}
where $M$ is the cluster mass at the current time step,
$\Delta M_c$ is the minimum mass change indicating the smallest
merging sub-cluster can be resolved in the merger trees;
and we choose $\Delta M_c = 10^{12} \,\Msun$.
In order to obey the probability distribution of mergers of different
$\Delta S$ (i.e., $\Delta M$), we sample the cumulative probability
distribution of the sub-cluster masses:
\begin{equation}
  \label{sec:cdf-sub-masses}
  P(<\Delta S, \Delta\omega) =
  \int_0^{\Delta S} K(\Delta S', \Delta\omega) \D{\Omega S'} =
  \mathrm{erfc} \left( \frac{\Delta \omega}{\sqrt{2 \Delta S}} \right),
\end{equation}
where $\mathrm{erfc}(x) = 1 - \mathrm{erf}(x)$ is the complementary error
function whose inversion is also analytical.
With a random $\Delta S$ drawn from this probability distribution,
the value of $S_1$ of the progenitor is then $S_1 = S_2 + \Delta S$,
and the mass of one of the progenitors is determined by
$\sigma^2(M_1) = S_1$, while for the other is $\Delta M = M_2 - M_1$.

Let $M_{\mathrm{main}} \equiv \mathrm{max}(M_1, \Delta M)$ and
$M_{\mathrm{sub}} \equiv \mathrm{min}(M_1, \Delta M)$.
If $M_{\mathrm{sub}} \leq \Delta M_c$, we regard this change in mass
to be due to accretion, and we continue build the merger tree from
$M_{\mathrm{main}}$ with the sub-cluster been ignored.
On the other hand, it is identified as a merger event if
$M_{\mathrm{sub}} > \Delta M_c$, then we trace both the main cluster
and the sub-cluster separately.

Considering that radio halos are only observed in
a fraction of massive merging clusters, suggesting they are closely
related to the recent major mergers, and have a lifetime
$\tau_{\mathrm{halo}}$ shorter than the timescale $\tau_m$ ($\sim$ 2--3 Gyr)
for the duration of merger-induced disturbance \citep{cassano2016}.
Therefore we only trace the merging history of the main cluster
to a maximum backward time of 3 Gyr.
Then we identify the most recent major merger for each cluster,
which is defined as $M_{\mathrm{main}}/M_{\mathrm{sub}} < 3$
\citep[e.g.,][]{wetzel2009},
and extract the $(M_{\mathrm{main}}, M_{\mathrm{sub}}, z_{\mathrm{merger}})$
data for use in the following radio halos simulation.
Meanwhile, those clusters without a recent major merger found
are thought to not host radio halos, and thus been ignored.

We also note that this technique allows only binary mergers,
but it is sufficient for our use here.

\subsubsection{Radio Halos}
\label{sec:halo}

Giant radio halos in galaxy clusters are associated with the on-going or
most recent major mergers, and they are widely believed to be radiated from
highly relativistic electrons re-accelerated by the merger-induced
large-scale turbulent fluctuations.
In addition, extended radio emissions of smaller scale (a few hundred kpc),
called mini-halos, have been found in some relaxed cool-core clusters.
We do not account for the radio mini-halos for the moment, and they
seems to have a different origin \citep{feretti2012rev}.

The approach we adopted here to simulate the giant radio halos
was largely inspired by \citet{cassano2005}, but we
made various modifications as well as simplifications for our needs.

For each cluster associated with a recent major merger, we assume that
the two sub-clusters undergo only central collisions.
Starting from an initial distance $d_0$ with zero velocity,
the relative impact velocity of two sub-clusters with mass
$M_{\mathrm{main}}$ and $M_{\mathrm{sub}}$ upon collide (at a distance
$R_{\mathrm{main}}$ between the centers) is given by
\citep{cassano2005,sarazin2002}:
\begin{equation}
  \label{eq:v-imp}
  v_{\mathrm{imp}} \simeq \left[
    \frac{2G (M_{\mathrm{main}} + M_{\mathrm{sub}})}{R_{\mathrm{main}}}
    \left( 1 - \frac{1}{\eta_v} \right)\right]^{1/2},
\end{equation}
where $d_0 = \eta_v R_{\mathrm{main}}$,
$\eta_v \simeq 4 (1 + M_{\mathrm{sub}}/M_{\mathrm{main}})^{1/3}$,
and $R_{\mathrm{main}}$ is the virial radius of the main cluster
[see Equation~(\ref{eq:radius-virial})].

Cluster mergers generate turbulence throughout the cluster over $\sim$Mpc
scales.  Energy can be transferred from the ICM into the non-thermal
component through the interactions between electrons and the
magnetohydrodynamics (MHD) turbulence, which accelerate the electrons
to be extremely relativistic and radiate synchrotron emissions.
%% TODO:
%% Describe energy transfer: merger -> turbulence -> electrons (eta_turb)
This scenario of particle re-acceleration by turbulence is the most
accepted mechanism responsible for the radio halos \citep{feretti2012rev}.

The turbulence re-acceleration likes a second-order Fermi process, i.e.,
related with a random process, may be effective for only a few $10^8$ yr
during a merger, and thus not quite efficient
\citep{feretti2012rev,enblin2011}.
We approximate the effective acceleration timescale with the merger
crossing time \citep{cassano2005}:
\begin{equation}
  \label{eq:tau-cross}
  \tau_{\mathrm{cross}} \simeq R_{\mathrm{main}} / v_{\mathrm{imp}}.
\end{equation}
In addition, the acceleration efficiency of electrons due to turbulent waves,
which described by the acceleration timescale, can be then estimated as
\citep{cassano2005,brunetti2011}:
\begin{equation}
  \label{eq:tau-acc}
  \tau_{\mathrm{acc}}(t) \simeq \left\{
    \begin{array}{ll}
      0.1 / \eta_{\mathrm{turb}}, & t \leq t_{\mathrm{merger}} + \tau_{\mathrm{cross}} \\
      0, & t > t_{\mathrm{merger}} + \tau_{\mathrm{cross}}
    \end{array}
  \right. \quad [\mathrm{Gyr}],
\end{equation}
where $\eta_{\mathrm{turb}}$ is the fraction of energy of the turbulence
in the form of magneto-sonic (MS) waves (which interact with electrons
and transfer the energy to accelerate them),
$t_{\mathrm{merger}}$ is the cosmic time when the merger begins
(i.e., at redshift $z_{\mathrm{merger}}$).

The time evolution of relativistic electrons spectrum with isotropic energy
distribution can be derived by solving the Fokker-Planck equation
\citep{eilek1991,schlickeiser2002}:
\begin{equation}
  \label{eq:fokkerplanck}
  \pdiff{n(\gamma,t)}{t} = \pdiff{}{\gamma} \left[ n(\gamma,t) \left(
      \left| \diff{\gamma}{t} \right|_{\mathrm{rad}} +
      \left| \diff{\gamma}{t} \right|_{\mathrm{ion}} -
      \frac{2}{\gamma} D_{\mathrm{\gamma\gamma}}(\gamma, t) \right) \right] +
  \pdiff{}{\gamma} \left[ D_{\mathrm{\gamma\gamma}} \pdiff{n(\gamma,t)}{p} \right] +
  Q_e(\gamma,t),
\end{equation}
where $\gamma = p / m_e c$ is the Lorentz factor of the electrons,
$n(\gamma, t)$ is the isotropic number density of relativistic
electrons, $|\mathrm{d}\gamma / \mathrm{d}t|_{(\cdot)}$ terms account for the
energy losses due to radiation and ionization,
$D_{\mathrm{\gamma\gamma}}(\gamma, t)$ is the diffusion coefficient describing
the interaction between electrons and turbulent MS waves, which is given by
\citep{brunetti2011}:
\begin{equation}
  \label{eq:diffusion-coef}
  D_{\mathrm{\gamma\gamma}}(\gamma, t) =
    \frac{\gamma^2}{4 \tau_{\mathrm{acc}}(t)},
\end{equation}
and $Q_e(\gamma, t)$ describes the electron injection.

The relativistic electrons lose energy via synchrotron radiation
and inverse Compton scattering off the CMB photons \citep{sarazin1999}:
\begin{equation}
  \label{eq:eloss-rad}
  \left( \diff{\gamma}{t} \right)_{\mathrm{rad}} =
  -1.37 \times 10^{-20} \gamma^2
  \left[ \left( \frac{B}{3.25 \,\uG} \right)^2 +
  (1+z)^4 \right] \quad [\mathrm{s}^{-1}],
\end{equation}
where $B$ is the mean magnetic field strength in the ICM.
Meanwhile, relativistic electrons in the ICM also lose energy through
ionization losses and Coulomb collisions \citep{sarazin1999}:
\begin{equation}
  \label{eq:eloss-ion}
  \left( \diff{\gamma}{t} \right)_{\mathrm{ion}} =
  -1.20 \times 10^{-12} n_{\mathrm{th}} \left[ 1 +
    \frac{\ln(\gamma/n_{\mathrm{th}})}{75} \right] \quad [\mathrm{s}^{-1}],
\end{equation}
where $n_{\mathrm{th}}$ is the ICM thermal number density in unit of
[$\mathrm{cm}^{-3}$]
[see Equation~(\ref{eq:nth})].

%% TODO: electron injection!

The ICM thermal energy density is:
\begin{equation}
  \label{eq:eth}
  \epsilon_{\mathrm{th}} = \frac{3}{2} n_{\mathrm{th}} k_B T,
\end{equation}
where $k_B T$ is the mean temperature of the ICM plasma, which is estimated
from the cluster mass using this $M$--$T$ scaling relation \citep{arnaud2005}:
\begin{equation}
  \label{eq:mt-relation}
  \left[ \frac{M_{200}}{10^{14} \,\Msun} \right] E(z) = (5.34 \pm 0.22)
  \left[ \frac{k_B T}{5 \,\mathrm{keV}} \right]^{(1.72 \pm 0.10)},
\end{equation}
where $E(z)$ is the redshift evolution factor [see Equation~(\ref{eq:ez-func})].
Note that we approximate the cluster virial mass with $M_{200}$ here.

For simplicity, we assume an uniform magnetic field throughout the ICM,
but allow it scales with the cluster mass as \citep{cassano2012}:
\begin{equation}
  \label{eq:magfield-mass}
  \left[ \frac{B}{\uG} \right] = \left[ \frac{B_{\langle {M} \rangle}}{\uG} \right]
  \left( \frac{M}{\langle {M} \rangle} \right)^b,
\end{equation}
where $\langle {M} \rangle = 1.6 \times 10^{15} \,\Msun$, and we take
$B_{\langle {M} \rangle} = 1.9 \,\uG$ and $b = 1.5$ for our simulations.

Observations show that radio halos have a radial size
$R_{\mathrm{halo}} \sim$ 3--6 times smaller than the virial radius
$R_{\mathrm{vir}}$, therefore we estimate the halo radius by
\citep{cassano2007,zandanel2014}:
\begin{equation}
  \label{eq:rhalo-rvir}
  R_{\mathrm{halo}} = \frac{1}{4} R_{\mathrm{vir}},
\end{equation}

%% TODO:
%% Fokker-Planck numerical solving scheme
%% initial electron spectrum
%% boundary condition & fix


\subsubsection{Radio Relics}
\label{sec:relic}

\citet[their Table 1]{nuza2017} compiled a collection of currently observed
radio relics.

\subsection{Extragalactic Point Sources}
\label{sec:pointsources}

Follow \citet{wang2010}???

\subsection{The \texttt{fg21sim} Simulation Software}
\label{sec:fg21sim}

We have implemented the above foreground simulation methods in the
\texttt{fg21sim} open-source software available at
\url{https://github.com/liweitianux/fg21sim}.

\section{Results}
\label{sec:results}

TODO!!!

Simulated images, scaling relations???

Radio halos (+relics): compare with observations!!!


\section{Discussions}
\label{sec:discussions}

TODO!!!


\section{Conclusions}
\label{sec:conclusions}

TODO!!!

%% If you wish to include an acknowledgments section in your paper,
%% separate it off from the body of the text using the \acknowledgments
%% command.
\acknowledgments

TODO!!!
We acknowledge ...


%% To help institutions obtain information on the effectiveness of their
%% telescopes the AAS Journals has created a group of keywords for telescope
%% facilities:
%% http://journals.aas.org/authors/aastex/facility.html
%
%% Following the acknowledgments section, use the following syntax and the
%% \facility{} or \facilities{} macros to list the keywords of facilities used
%% in the research for the paper.  Each keyword is check against the master
%% list during copy editing.  Individual instruments can be provided in
%% parentheses, after the keyword, but they are not verified.

\vspace{5mm}

% \facilities{???}

%% Similar to \facility{}, there is the optional \software command to allow
%% authors a place to specify which programs were used during the creation of
%% the manusscript. Authors should list each code and include either a
%% citation or url to the code inside ()s when available.

\software{
  Astropy \citep{astropy2013},
  NumPy (\url{http://www.numpy.org/}),
  Vim (\url{http://www.vim.org/}),
  Emacs (\url{https://www.gnu.org/s/emacs/})
}


%% Appendix material should be preceded with a single \appendix command.
%% There should be a \section command for each appendix. Mark appendix
%% subsections with the same markup you use in the main body of the paper.

%% Each Appendix (indicated with \section) will be lettered A, B, C, etc.
%% The equation counter will reset when it encounters the \appendix
%% command and will number appendix equations (A1), (A2), etc. The
%% Figure and Table counter will not reset.

\appendix

\section{Press-Schechter Formalism}
\label{sec:psf}

The Press-Schechter (PS) formalism is one of the standard method to predict
the mass function of galaxy clusters and its evolution as the cosmology.
It is originally developed by \citet{press1974}, and later extended to combine
with the cold dark matter (CDM) model \citep[e.g.,][]{bond1991,lacey1993}.


\section{Useful Formulas}
\label{sec:formulas}

In this section we collect some useful formulas we made use of in our
simulations.

Redshift evolution factor (flat \lcdm{} cosmology):
\begin{equation}
  \label{eq:ez-func}
  E(z) = \sqrt{\Omega_m(1+z)^3 + \Omega_{\Lambda}}.
\end{equation}

The virial radius of a cluster:
\begin{equation}
  \label{eq:radius-virial}
  R_{\mathrm{vir}} = \left[
    \frac{3 M_{\mathrm{vir}}}{4\pi \Delta_c(z) \rho_{\mathrm{crit}}(z)} \right],
\end{equation}
where $M_{\mathrm{vir}}$ is the virial mass (can also be regarded as
the total mass) of the cluster,
$\rho_{\mathrm{crit}}(z)$ is the cosmological critical density at redshift $z$,
$\Delta_c(z)$ is the average overdensity of the cluster at redshift $z$.

The thermal number density of the cluster ICM is approximately given by:
\begin{equation}
  \label{eq:nth}
  n_{\mathrm{th}} \simeq
    \frac{3 M_{\mathrm{baryon}}}{4\pi \mu m_u R_{\mathrm{vir}}^3},
\end{equation}
where $\mu \simeq 0.6$ is the mean molecular weight \citep{ettori2013},
$m_u$ is the atomic mass unit,
and $M_{\mathrm{baryon}} \simeq M_{\mathrm{vir}} f_b$ is the baryon mass
of the cluster with $f_b \simeq \Omega_b / \Omega_m$ the mean baryon
fraction assumed for clusters.


%%%%%%%%%%%%%
%% References
%%%%%%%%%%%%%

\bibliography{references}


%% Include this line if you are using the \added, \replaced, \deleted
%% commands to see a summary list of all changes at the end of the article.
%\listofchanges

\end{document}

%% EOF
