%%
%% Copyright (c) 2017-2018 The Authors.  All Rights Reserved.
%%
%% Weitian LI, et al.
%% School of Physics and Astronomy, Shanghai Jiao Tong University, Shanghai, China.
%%
%% 2017-07-18
%%

\documentclass[modern]{aastex62}
%\documentclass[twocolumn]{aastex62}
%%
%% Defaults: single spaced, 10 point font article
%%
%%  twocolumn   : two text columns, 10 point font, single spaced article.
%%                This is the most compact and represent the final published
%%                derived PDF copy of the accepted manuscript from the publisher
%%  manuscript  : one text column, 12 point font, double spaced article.
%%  preprint    : one text column, 12 point font, single spaced article.
%%  preprint2   : two text columns, 12 point font, single spaced article.
%%  modern      : a stylish, single text column, 12 point font, article with
%%                wider left and right margins. This uses the Daniel
%%                Foreman-Mackey and David Hogg design.
%%
%% Note that you can submit to the AAS Journals in any of these 6 styles.
%%
%% There are other optional arguments one can envoke to allow other stylistic
%% actions. The available options are:
%%
%%  astrosymb    : Loads Astrosymb font and define \astrocommands.
%%  tighten      : Makes baselineskip slightly smaller, only works with
%%                 the twocolumn substyle.
%%  times        : uses times font instead of the default
%%  linenumbers  : turn on lineno package.
%%  trackchanges : required to see the revision mark up and print its output
%%  longauthor   : Do not use the more compressed footnote style (default) for
%%                 the author/collaboration/affiliations. Instead print all
%%                 affiliation information after each name. Creates a much
%%                 long author list but may be desirable for short author papers

%% Extra packages
\usepackage{amsmath}
\usepackage{csquotes}  % `enquote'
\usepackage{bm}  % bold math
\usepackage{savesym}

\savesymbol{tablenum}  % workaround symbol conflict
                       % credit: https://tex.stackexchange.com/a/269074
\usepackage{siunitx}  % typeset units; from `texlive-science'
\restoresymbol{SI}{tablenum}  % `tablenum' from `siunitx' becomes `SItablenum'

%% Chinese
\usepackage{xeCJK}
\setCJKmainfont{SimSun}
\setCJKsansfont{KaiTi}
\setCJKmonofont{KaiTi}

%% Bibliography style
\bibliographystyle{aasjournal-links}

%% Custom macros
\newcommand{\etal}{\textit{et~al.}}
\newcommand{\R}[1]{\mathrm{#1}}
\newcommand{\D}[1]{\R{d} #1}
\newcommand{\diff}[2]{\frac{\D{#1}}{\D{#2}}}
\newcommand{\pdiff}[2]{\frac{\partial #1}{\partial #2}}
\newcommand{\lcdm}{$\Lambda$CDM}
\newcommand{\Halpha}{\text{H$\alpha$}}
\newcommand{\Hi}{H\textsc{i}}
\newcommand{\Hii}{H\textsc{ii}}
\newcommand{\klos}{\text{$k_{\parallel}$}}
\newcommand{\kperp}{\text{$k_{\bot}$}}

%% Marks
\newcommand{\NOTE}[1]{\textcolor{violet}{[\textbf{NOTE:}~\uwave{#1}]}}
\newcommand{\TODO}[1]{\textcolor{magenta}{[\textbf{TODO:}~\uuline{#1}]}}
\newcommand{\OK}[1]{\textcolor{cyan}{[\textbf{OK:}~\uline{#1}]}}
\newcommand{\XU}[1]{\textcolor{blue}{[\textbf{XU:}~\uline{#1}]}}
\newcommand{\LI}[1]{\textcolor{purple}{[\textbf{LI:}~\uline{#1}]}}

%% Custom math operators (requires `amsmath' package)
\DeclareMathOperator{\erf}{erf}
\DeclareMathOperator{\erfc}{erfc}

%% Settings for package `siunitx'
% credit: https://tex.stackexchange.com/a/194042
\sisetup{
  range-phrase=\text{--},
  range-units=single,
  product-units=power
}
%
%% Custom units (requires `siunitx' package)
\DeclareSIUnit\arcsec{arcsec}
\DeclareSIUnit\arcmin{arcmin}
\DeclareSIUnit\cMpc{cMpc}  % comoving Mpc
\DeclareSIUnit\cGpc{cGpc}  % comoving Gpc
\DeclareSIUnit\deg{deg}
\DeclareSIUnit\erg{erg}
\DeclareSIUnit\gauss{G}
\DeclareSIUnit\hour{hr}  % overwrite hour from 'h' to 'hr'
\DeclareSIUnit\hubble{\text{$h$}}
\DeclareSIUnit\jansky{Jy}
\DeclareSIUnit\lightyear{ly}
\DeclareSIUnit\parsec{pc}
\DeclareSIUnit\rayleigh{Rayleigh}
\DeclareSIUnit\solarmass{\text{M$_{\odot}$}}
\DeclareSIUnit\year{yr}
%
\DeclareSIUnit\keV{\kilo\electronvolt}
\DeclareSIUnit\kpc{\kilo\parsec}
\DeclareSIUnit\mJy{\milli\jansky}
\DeclareSIUnit\uG{\micro\gauss}
\DeclareSIUnit\Gyr{\giga\year}
\DeclareSIUnit\MHz{\mega\hertz}
\DeclareSIUnit\Mpc{\mega\parsec}
\DeclareSIUnit\Gpc{\giga\parsec}

%% Hyperref-related settings
\def\sectionautorefname{Section}
\def\subsectionautorefname{Section}
\def\subsubsectionautorefname{Section}
\def\figureautorefname{Figure}
\def\tableautorefname{Table}
% Credit: https://tex.stackexchange.com/a/66150
\def\equationautorefname~#1\null{Equation~(#1)\null}

%% Custom subsubsubsection ...
\newcounter{sssseccount}
\newcommand{\sssseclabel}{\alph{sssseccount}}
\newcommand{\ssssec}[1]{%
  \vspace{1ex}%
  \stepcounter{sssseccount}%
  \noindent\emph{\sssseclabel. #1}%
}

%% Custom settings
\setlength{\parindent}{1.5em}

%% Journal information
\received{xxx, 2018}
\revised{xxx, 2018}
\accepted{xxx, 2018}
\submitjournal{ApJ}

%% Add a light gray and diagonal water-mark to the first page
\watermark{DRAFT}  % watermark text: DRAFT
%% Control the water-mark size with:
%% \setwatermarkfontsize{dimension}

\shorttitle{Impacts of Radio Halos on EoR Detection}
\shortauthors{Li~\etal}


%%%%%%%%%%%%%%%%%%%%%%%%%%%%%%%%%%%%%%%%%%%%%%%%%%%%%%%%%%%%%%%%%%%%%%%%%

\begin{document}

\title{%
  Simulation of Radio Halos in Galaxy Clusters and
  the Impacts on the Epoch of Reionization Detection
}

%% The \author command is the same as before except it now takes an optional
%% argument which is the 16 digit ORCID. The syntax is:
%% \author[xxxx-xxxx-xxxx-xxxx]{Author Name}
%%
%% This will hyperlink the author name to the author's ORCID page. Note that
%% during compilation, LaTeX will do some limited checking of the format of
%% the ID to make sure it is valid.
%%
%% Use \affiliation for affiliation information.  Please use multiple
%% \affiliation calls for to document more than one affiliation.  AASTeX v6.1
%% will automatically index these in the header.  When a duplicate is found
%% its index will be the same as its previous entry.
%%
%% The new \altaffiliation can be used to indicate some secondary information
%% such as fellowships. This command produces a non-numeric footnote that is
%% set away from the numeric \affiliation footnotes.  NOTE that if an
%% \altaffiliation command is used it must come BEFORE the \affiliation call,
%% right after the \author command, in order to place the footnotes in
%% the proper location.
%%
%% Use \email to set provide email addresses. Each \email will appear on its
%% own line so you can put multiple email address in one \email call. A new
%% \correspondingauthor command is available in V6.1 to identify the
%% corresponding author of the manuscript. It is the author's responsibility
%% to make sure this name is also in the author list.
%%
%% While authors can be grouped inside the same \author and \affiliation
%% commands it is better to have a single author for each. This allows for
%% one to exploit all the new benefits and should make book-keeping easier.
%%
%% If done correctly the peer review system will be able to automatically
%% put the author and affiliation information from the manuscript and save
%% the corresponding author the trouble of entering it by hand.

\correspondingauthor{Haiguang Xu}
\email{hgxu@sjtu.edu.cn}

\author[0000-0002-7527-380X]{Weitian Li}
\email{liweitianux@sjtu.edu.cn}
\affiliation{School of Physics and Astronomy,
  Shanghai Jiao Tong University,
  800 Dongchuan Road, Shanghai 200240, China}

\author{Haiguang Xu}
\affiliation{School of Physics and Astronomy,
  Shanghai Jiao Tong University,
  800 Dongchuan Road, Shanghai 200240, China}
\affiliation{Tsung-Dao Lee Institute,
  Shanghai Jiao Tong University,
  800 Dongchuan Road, Shanghai 200240, China}
\affiliation{IFSA Collaborative Innovation Center,
  Shanghai Jiao Tong University,
  800 Dongchuan Road, Shanghai 200240, China}

\author{Zhixian Ma}
\affiliation{Department of Electronic Engineering,
  Shanghai Jiao Tong University,
  800 Dongchuan Road, Shanghai 200240, China}

\author{Zhenghao Zhu}
\affiliation{School of Physics and Astronomy,
  Shanghai Jiao Tong University,
  800 Dongchuan Road, Shanghai 200240, China}

\author{Dan Hu}
\affiliation{School of Physics and Astronomy,
  Shanghai Jiao Tong University,
  800 Dongchuan Road, Shanghai 200240, China}

\author{Chenxi Shan}
\affiliation{School of Physics and Astronomy,
  Shanghai Jiao Tong University,
  800 Dongchuan Road, Shanghai 200240, China}


%%%%%%%%%%%%%%%%%%%%%%%%%%%%%%%%%%%%%%%%%%%%%%%%%%%%%%%%%%%%%%%%%%%%%%%%%

\begin{abstract}
\TODO{write!!!}
\end{abstract}

%% Authors should select subject keywords from the given list.  No more
%% than 6 keywords should be given, and they should be listed in
%% alphabetical order.
%%
%% See the online documentation for the full list of available subject
%% keywords and the rules for their use.
%% http://journals.aas.org/authors/keywords2013.html
\keywords{
  \TODO{update};
  dark ages, reionization, first stars ---
  galaxies: clusters: intracluster medium ---
  methods: numerical ---
  radio continuum: galaxies
}


%########################################################################
\section{Introduction}
\label{sec:intro}
%########################################################################

The Epoch of Reionization (EoR; $z \sim \numrange{15}{6}$) is a
cosmological periodization referring to a period of our Universe preceded
by the Cosmic Dawn ($z \sim \numrange{30}{15}$) and the Dark Ages
($z \sim \numrange{200}{30}$).
During the EoR, which lasted from about 300 million to 1 billion years
after the Big Bang, the reionization of neutral hydrogen (\Hi) primarily
caused by the ultra-violet and soft X-ray photons from the first
generation celestial objects surpassed the cooling of the gas through
recombination of electrons and protons.
As a result the majority of baryonic matter was again in highly ionized state.
Comparing with the observations of distant quasars and cosmic microwave
background (CMB), which have provided some loose constraints on the
reionization process, the detection the redshifted 21 cm line emission
from \Hi, especially those from the adjacent regions of the ionized bubbles,
is recognized to be the decisive tool to directly reveal this epoch
\citep[see][for reviews]{zaroubi2013rev,furlanetto2016rev}.

In order to probe the EoR, a number of radio interferometric experiments
working at the \SIrange{\sim 100}{200}{\MHz} low-frequency band have been
designed to target the redshifted 21 cm signals.
These include the LOw Frequency ARray (LOFAR; \citealt{vanHaarlem2013}),
the 21 CentiMeter Array (21CMA; \citealt{zheng2016}),
the Giant Meterwave Radio Telescope (GMRT; \citealt{paciga2011}),
the MIT Epoch of Reionization Experiment (MITEoR; \citealt{zheng2014}),
the Donald C. Backer Precision Array for Probing the Epoch of
Reionization (PAPER; \citealt{parsons2010}),
the Murchison Widefield Array (MWA; \citealt{bowman2013,tingay2013}),
and those to be built and operated in the near future, such as the Square
Kilometre Array (SKA; \citealt{mellema2013rev,koopmans2015rev}) and
the Hydrogen Epoch of Reionization Arrays (HERA; \citealt{deboer2017}).

The challenges in these EoR experiments, however, are immense
due to a variety of complicated instrumental effects,
ionospheric distortions, radio frequency interference, and the
strong celestial foreground contamination, which overwhelms the
redshifted 21~cm signals by \numrange{4}{5} orders of magnitude
\citep[e.g.,][]{beardsley2016,murray2017,procopio2017}.
For a typical region beyond the Galactic plane,
while the Galactic diffuse radiation (including both the synchrotron
and free-free emissions) and extragalactic point sources are the two
most prominent components and contribute about \SI{99}{\percent} of
the foreground contamination in total, the extragalactic diffuse emissions,
which stem from galaxy clusters \citep[e.g.,][]{feretti2012rev,kale2016rev},
intergalactic medium \citep[e.g.,][]{keshet2004}, and large-scale cosmic
webs \citep[e.g.,][]{arayaMelo2012,vazza2015},
may be tens to hundreds of stronger than the 21~cm signals,
and thus can have significant impacts on the EoR detection.

Nowadays, diffuse radio emissions are observed in $\gtrsim 100$ galaxy
clusters, and can be classified into halos, relics, and mini-halos
according to their location in the cluster and to the cluster type
(merging or cool-core).
Radio halos are generally found at the center of merging clusters;
radio relics are located in peripheral regions of both merging and
relaxed clusters, and exhibit significant polarization;
and mini-halos are hosted at the center of cool-core clusters and
usually surround a powerful radio galaxy \citep{feretti2012rev}.
All these kinds of diffuse radio emissions are associated with the
intracluster medium (ICM) since there is no obvious connection to
the member galaxies.
This observational evidence reveals the existence of the non-thermal
component in galaxy clusters, which provides an unique probe to study
the very energetic processes operating within clusters (e.g., generation
and transportation of cosmic rays)
and the physical properties of the ICM (e.g., turbulence, viscosity).
However, many aspects of the diffuse radio emission are still poorly
understood, although there are various observational, analytical, and
simulation works dedicated to such studies
\citep[see][for a recent review]{brunetti2014rev}.
On the other hand, the future SKA1-Low is expected to observe several
thousands of radio halos \citep{cassano2015}, and there are much more
fainter ones waiting to be discovered, therefore the cumulative emission
of the radio halos may pose a significant impact on the detection of
21~cm signals, which is lesser discussed.
In this work, we attempt to investigate the effects of the contamination
contributed by radio halos in galaxy clusters within the framework of
EoR detection.
We have developed an open-source software
\texttt{FG21sim}\footnote{FG21sim: \url{https://github.com/liweitianux/fg21sim}}
to simulate the low frequency sky based on our previous works
\citep{wang2010,wang2013}, but have significantly enhanced the modeling
of radio halos by applying a merger-induced turbulence re-acceleration model.
In addition, we adopt the latest SKA1-Low layout configuration to
incorporate the practical instrumental effects.

This paper is organized as follows.
In \autoref{sec:sky-simu} we describe the simulations of the low-frequency
radio sky, including the radio halos in galaxy clusters, the Galactic
emissions, the extragalactic point sources, and the cosmological EoR signal.
In \autoref{sec:obs-simu} we perform the observational simulation of the
simulated sky in order to account for the practical instrumental effects.
We then quantitatively evaluate the contamination of radio halos on EoR
detection and discuss the results in \autoref{sec:results},
and finally summarize our work in \autoref{sec:summary}.
Throughout this work we adopt a flat \lcdm{} cosmology with
$H_0 = \SI{100}{\hubble} = \SI{71}{\km\per\second\per\Mpc}$,
$\Omega_m = 0.27$, $\Omega_{\Lambda} = 1 - \Omega_m = 0.73$,
$\Omega_b = 0.046$, $n_s = 0.96$, and $\sigma_8 = 0.81$.


%########################################################################
\section{Low-Frequency Sky Simulation}
\label{sec:sky-simu}
%########################################################################

Based on our previous works \citep{wang2010,wang2013}, we have used the
Python programming language to develop the \texttt{FG21sim} software to
simulate the low-frequency radio sky by taking into account the
contributions of Our Galaxy, extragalactic point sources, and radio halos
in galaxy clusters.
The software is open-sourced under the MIT License and is available at
\url{https://github.com/liweitianux/fg21sim},
where further documents and auxiliary data are provided.
Additional tools are also developed to aid the observational simulations
in order to integrate the instrumental responses of low-frequency radio
interferometers.

In this work we use \texttt{FG21sim} version 1.0??? to simulate the
radio sky maps and evaluate the effects of the contamination caused by
radio halos.
By consulting the latest SKA science book on the EoR survey strategies
\citep[e.g.,][]{koopmans2015rev}, we perform the simulations within
a \SI{8}{\MHz} band centered at \SI{158}{\MHz} for a \SI{10 x 10}{\deg}
sky region at (R.A., Dec.) = (\SI{0}{\degree}, \SI{-27}{\degree}),
which locates at high galactic latitude ($b = \SI{-78.5}{\degree}$)
and passes through the zenith of SKA1-Low\footnote{This sky region
  is also targeted by MWA (EoR0 field; \citealt{beardsley2016}).}.
The \SI{8}{\MHz} bandwidth is chosen to limit the cosmological evolution
effect for the 21~cm signal when calculating power spectra (cite???),
and the large region size of \SI{10 x 10}{\deg} will cover the first
side-lobe and also help the CLEAN imaging in \autoref{sec:obs-simu}.
Besides the 21~cm signal and radio halos, we also incorporate the
Galactic radiation (including both the synchrotron and free-free emissions)
and extragalactic point sources into the simulated sky maps.
The sky maps are pixelized into \num{1800 x 1800} with a pixel size of
\SI{20}{\arcsec} \TODO{improve to 10 arcsec/pixel???}.


%========================================================================
\subsection{Radio Halos in Galaxy Clusters}
\label{sec:cluster-halo}
%========================================================================

Compared to our previous work \citep{wang2010},
we have significantly improved the simulation of radio halos by
employing the merger-induced turbulence re-acceleration model.
For each galaxy cluster, we first simulate its merging history,
determine the turbulent acceleration efficiency and duration for each
merger event, then evolve the relativistic electron spectrum with time
by considering both the turbulent acceleration and energy losses, and
finally obtain the synchrotron emission and the radio image of the
radio halo.

%------------------------------------------------------------------------
\subsubsection{Mass Function}
\label{sec:distributions}
%------------------------------------------------------------------------

The Press-Schechter formalism was originally advanced as one of the standard
method to predict the mass function of galaxy clusters and its evolution
in the Universe \citep{press1974}, and has been extended to combine with
the cold dark matter (CDM) models \citep[e.g.,][]{bond1991,lacey1993}.
The number density (i.e., the number per unit comoving volume per unit
mass) of dark matter halos (i.e., galaxy clusters) at a certain redshift
$z$ is predicted by the Press-Schechter mass function \citep{press1974}:
\begin{equation}
  \label{eq:ps-mass-func}
  n(M, z) \,\D{M} = \sqrt{\frac{2}{\pi}} \frac{\langle{\rho}\rangle}{M}
  \frac{\delta_c(z)}{\sigma^2(M)} \left| \diff{\sigma(M)}{M} \right|
  \exp\!\left[ -\frac{\delta_c^2(z)}{2\sigma^2(M)} \right] \D{M},
\end{equation}
where
$M = (1 - f_{\R{gas}}) M_{\R{cl}} = (1 - \Omega_b/\Omega_m) M_{\R{cl}}$
is the dark matter halo mass with $M_{\R{cl}}$ the cluster mass,
$\langle {\rho} \rangle$ is the current mean density of the Universe,
$\delta_c(z)$ is the critical linear overdensity for a region to collapse
at redshift $z$ [see \autoref{eq:delta-crit}],
and $\sigma(M)$ is the current rms (root-mean-square) density fluctuations
within a sphere of mean mass $M$.

Considering the CDM model and the halo mass range covered by galaxy
clusters, it is reasonable to adopt such a power-law distribtuion
for the density perturbations \citep{sarazin2002,randall2002}:
\begin{equation}
  \label{eq:sigma-mass}
  \sigma(M) = \sigma_8 \left( \frac{M}{M_8} \right)^{-\alpha},
\end{equation}
where $\sigma_8$ is the present-day rms density fluctuations on a
scale of \SI{8}{\per\hubble\Mpc},
$M_8 = (4\pi/3)(8 \,\si{\per\hubble\Mpc})^3 \langle{\rho}\rangle$
is the mass contained in a sphere of a radius \SI{8}{\per\hubble\Mpc},
and the exponent $\alpha = (n+3)/6$ is related to the fluctuations
pattern whose power spectrum varies with wavenumber $k$ as $k^n$.
Here we follow \citet{randall2002} and adopt $n = -7/5$.

Then, by placing a conservative minimum cluster mass cut of
$M_{\R{min}} = \SI{e14}{\solarmass}$ and a maximum redshift cut
at $z_{\R{max}} = 4$, we are able to use \autoref{eq:ps-mass-func}
to calculate both the total number and the distribution of galaxy
clusters for any given sky patch.
The mass $M_{\R{obs}}$, redshift $z_{\R{obs}}$, and sky position of
each simulated galaxy cluster can be randomly sampled using the
Monte Carlo method \citep{wang2010}.


%------------------------------------------------------------------------
\subsubsection{Merging History}
\label{sec:merging-history}
%------------------------------------------------------------------------

Galaxy clusters are formed hierarchically by accretion and merging from
smaller clusters, groups, and galaxies.
The extended Press-Schechter theory outlined in \citet{lacey1993} provide
us with a way to simulate the growth history of a cluster (i.e., merger tree).
For each merger tree to be built, we start with the cluster's current
mass $M_{\R{obs}}$ and redshift $z_{\R{obs}}$, which are randomly sampled
from the mass function in \autoref{sec:distributions}.
We trace the growth history of this cluster back in time by running
Monte Carlo simulations to randomly determine the mass change
$\Delta M$ at each step, which may be regarded
either as a merger event (if $\Delta M > \Delta M_c$)
or as an accretion event (if $\Delta M \leq \Delta M_c$).
We choose $\Delta M_c = \SI{e13}{\solarmass}$ to focus on the major mergers
with which radio halos are associated,
and we only trace the merging history of the main cluster,
which are sufficient for this work.

We assume that during each growth step the cluster mass increases from
$M_1$ at time $t_1$ (earlier) to $M_2$ at $t_2$ (later).
Given $M_2$ and $t_2$, the conditional probability of the cluster had
a progenitor being in the mass range $M_1 \to M_1 + \D{M_1}$ at an
earlier time $t_1$ can be expressed as
\begin{equation}
  \label{eq:eps-condprob}
  \R{Pr}(M_1, t_1 | M_2, t_2) \,\D{M_1} = \frac{1}{\sqrt{2\pi}}
\frac{M_2}{M_1}
  \frac{\delta_{c1} - \delta_{c2}}{(\sigma_1^2 - \sigma_2^2)^{3/2}}
  \left| \diff{\sigma_1^2}{M_1} \right|
  \exp \!\left[ -\frac{(\delta_{c1} - \delta_{c2})^2}
    {2(\sigma_1^2 - \sigma_2^2)} \right] \D{M_1},
\end{equation}
where
$\delta_{ci} \equiv \delta_c(t_i)$, $\sigma_i \equiv \sigma(M_i)$, and $i =
1, 2$ is used to denote parameters defined at time $t_1$ and $t_2$,
respectively \citep{lacey1993,randall2002}.
By further introducing $S \equiv \sigma^2(M)$ and
$\omega \equiv \delta_c(t)$, this equation reduces to
\begin{equation}
  \label{eq:eps-condprob-simp}
  \R{Pr}(\Delta S, \Delta \omega) \,\D{\Delta S} = \frac{1}{\sqrt{2\pi}}
  \frac{\Delta\omega}{(\Delta S)^{3/2}}
  \exp \!\left[ -\frac{(\Delta\omega)^2}{2 \Delta S} \right] \D{\Delta S}.
\end{equation}

During the backward tracing of a cluster, the step size $\Delta\omega$
is chosen adaptively as \citep{randall2002}:
\begin{equation}
  \label{sec:dw-step}
  (\Delta\omega)_{\R{step}} \lesssim \frac{1}{2} \left[
    S \left| \diff{\ln \sigma^2}{\ln M} \right|
    \left( \frac{\Delta M_c}{M} \right) \right]^{1/2},
\end{equation}
where $M$ is the cluster mass at the corresponding step.
In order to obey the probability distribution of mergers with different
mass changes (i.e., $\Delta S$),
we sample the cumulative probability distribution of the sub-cluster masses
to determine $\Delta S$ (i.e., $\Delta M$) at each step:
\begin{equation}
  \label{sec:cdf-sub-masses}
  \R{Pr}(<\!\Delta S, \Delta\omega) =
  \int_0^{\Delta S} \R{Pr}(\Delta S', \Delta\omega) \,\D{\Delta S'} =
  \erfc \!\left( \frac{\Delta \omega}{\sqrt{2 \Delta S}} \right),
\end{equation}
where $\erfc(x) = 1 - \erf(x)$ is the complementary error function.
With a random $\Delta S$ (i.e., $M_{\R{sub}}$) drawn from this probability
distribution, the mass of the cluster's progenitor $M_1$ is obtained from
$S_1 = S_2 + \Delta S$.

Considering that radio halos are associated to recent major mergers and
have a lifetime $\tau_{\R{halo}} \lesssim \SI{1}{\Gyr}$
\citep{cassano2016} \TODO{change ref???},
we only trace the cluster merging history to a maximum backward time of
\SI{3}{\Gyr}.
Then for each cluster we extract the properties of all the merger events
for the following radio halo simulation.


%------------------------------------------------------------------------
\subsubsection{Radio Halo Emergence and Evolution}
\label{sec:halos}
%------------------------------------------------------------------------

In the adopted merger-induced turbulence re-acceleration model, there
exists a population of primary (or fossil) high-energy electrons permeating
the ICM, which are thought to be injected by active galactic nucleus (AGN)
activities (quasars, radio galaxies, etc.), and by star formation in
normal galaxies (supernovae, galactic winds, etc.) during the cluster
life \citep[see][for a review]{blasi2007rev},
then the turbulence generated by intense mergers interact with the ICM
and accelerate the fossil electrons to be highly relativistic, resulting
in the diffuse synchrotron radiation.

Besides the turbulent acceleration, electrons can loss energies via
synchrotron radiation, inverse Compton scattering off the CMB photons,
Coulomb collisions, etc. \citep{sarazin1999}.
Given a population of electrons with isotropic energy distribution, and
considering both the energy gain and loss mechanisms, the time evolution
of the electron spectrum is governed by the Fokker-Planck equation
\citep{eilek1991,schlickeiser2002}:
\begin{equation}
  \label{eq:fokkerplanck}
  \pdiff{n(\gamma,t)}{t} = \pdiff{}{\gamma} \left[ n(\gamma,t) \left(
      \left| \diff{\gamma}{t} \right| -
      \frac{2}{\gamma} D_{\gamma\gamma}(\gamma, t) \right) \right] +
    \pdiff{}{\gamma} \left[ D_{\gamma\gamma} \pdiff{n(\gamma,t)}{\gamma}
    \right] + Q_e(\gamma,t),
\end{equation}
where $\gamma \approx p / (m_e c)$ is the Lorentz factor of the
high-energy electrons with $\gamma \gg 1$,
$n(\gamma, t)$ is the isotropic number density of electrons,
$D_{\gamma\gamma}(\gamma, t)$ is the diffusion coefficient describing
the interactions between the turbulence and electrons,
$|\R{d}\gamma / \R{d}t|$ accounts for various energy loss mechanisms,
and $Q_e(\gamma, t)$ describes the electron injection.

%%%
\setcounter{sssseccount}{0}
\ssssec{Electron Injection}
%%%

It is reasonable to assume that the primary electrons were constantly
injected into the ICM during the cluster life and had a power-law
spectrum of index $s \sim 2.3$ \citep[e.g.,][]{sarazin1999}
\begin{equation}
  \label{eq:electron-inj}
  Q_e(\gamma, t) \equiv Q_e(\gamma) = K_e \,\gamma^{-s}.
\end{equation}
By further assuming that the total energy of injected electrons is a
fraction, $\eta_e$, of the total thermal energy \citep{cassano2005}
\begin{equation}
  \epsilon_e \,\tau_{\R{cl}} \int_{\gamma_{\R{min}}}^{\gamma_{\R{max}}}
  Q_e(\gamma') \gamma' \,\D{\gamma'}
  = \eta_e \,\epsilon_{\R{th}},
\end{equation}
and given $\gamma_{\R{min}} \ll \gamma_{\R{\max}}$,
the injection rate $K_e$ can be derived to be
\begin{equation}
  \label{eq:injrate}
  K_e \simeq \frac{(s-2)\,\eta_e\,\epsilon_{\R{th}}}{\epsilon_e\,\tau_{\R{cl}}}
    \gamma_{\R{min}}^{s-2},
\end{equation}
where $\tau_{\R{cl}} \simeq t_{\R{obs}}$ is the cluster's age at the
\enquote{observed} redshift $z_{\R{obs}}$ as simulated in \autoref{sec:distributions},
and $\epsilon_e = m_e c^2$ is the electron's still energy.
In addition, the thermal energy density of the ICM, $\epsilon_{\R{th}}$,
is calculated as
\begin{equation}
  \label{eq:energy-density-thermal}
  \epsilon_{\R{th}} = \frac{3}{2} \,n_{\R{th}} \,k_BT_{\R{cl}},
\end{equation}
and the thermal number density, $n_{\R{th}}$, is given by
\begin{equation}
  \label{eq:number-density-thermal}
  n_{\R{th}} \simeq \frac{3 f_{\R{gas}} M_{\R{vir}}}{4\pi \mu m_u \,r^3_{\R{vir}}},
\end{equation}
where
$r_{\R{vir}}$ is the cluster's virial radius,
the cluster mean temperature is approximated with the virial temperature:
$k_BT_{\R{cl}} \simeq k_BT_{\R{vir}} = G M_{\R{vir}} \,\mu m_u / (2 r_{\R{vir}})$,
$f_{\R{gas}} \simeq \Omega_b/\Omega_m$ is the gas fraction assumed
for clusters, $\mu \simeq 0.6$ is the mean molecular weight
\citep[e.g.,][]{ettori2013}, and $m_u$ is the atomic mass unit.

As for the initial spectrum of electrons for solving
\autoref{eq:fokkerplanck}, it is derived by evolving the accumulated spectrum
$n_e'(\gamma) = t_0 K_e \gamma^{-s}$ for \SI{\sim 1}{\Gyr} with only energy
losses, where $t_0$ is the cosmic age when the first merger event begins
\citep[e.g.,][]{brunetti2007}.

%%%
\ssssec{Turbulent Acceleration}
%%%

The interactions between the merger-induced turbulence and the cluster
ICM are complicated and still poorly understood.
Among the several different particle acceleration mechanisms could be
potentially triggered by the turbulence, the most important one is that
the turbulence dissipates its energy and accelerates particles by interacting
with the relativistic particles (e.g., cosmic rays) in the ICM via the
transit time damping process
\citep[and references therein]{brunetti2007,brunetti2011},
and the associated diffusion coefficient is given as:
\citep{pinzke2017,miniati2015}:
\begin{equation}
  \label{eq:dpp}
  D_{\gamma\gamma} = \gamma^2 \frac{4\pi \zeta}{X_{\R{CR}} L_{\R{turb}}}
    \frac{\langle (\delta{v_t})^2 \rangle^2}{c_s^3},
\end{equation}
where
$\zeta \sim \numrange{0.1}{0.3}$ is an efficiency factor characterizing
the plasma instabilities (e.g., due to spatial or temporal intermittency),
$X_{\R{CR}} = \epsilon_{\R{CR}} / \epsilon_{\R{th}} \sim \SI{1}{\percent}$
is the relative energy density of cosmic rays with respect to the thermal ICM,
$L_{\R{turb}} \simeq r_{\R{vir}}/3$ is the turbulence injection scale with
$r_{\R{vir}}$ the virial radius of the merged cluster \citep{miniati2015},
$\langle (\delta{v_t})^2 \rangle$ is the turbulence velocity dispersion,
and $c_s$ is the sound speed in the ICM.

The total kinetic energy carried by a merger can be estimated as:
\begin{equation}
  \label{eq:energy-merger}
  E_{\R{merger}} = \frac{1}{2} f_{\R{gas}} M_{\R{sub}} v^2_{\R{sub}}
    \simeq \frac{1}{2} f_{\R{gas}} M_{\R{sub}} \frac{G M_{\R{main}}}{r^2_{\R{main}}},
\end{equation}
where $M_{\R{main}}$ and $r_{\R{main}}$ are the virial mass and virial
radius of the main cluster, respectively,
and $f_{\R{gas}} \simeq \Omega_b/\Omega_m$ is the gas fraction assumed
for clusters.
To further estimate the velocity dispersion of the turbulence, we assume
that a fraction, $\eta_{\R{turb}}$, of the merger kinetic energy is
transformed into turbulent waves, i.e.,
\begin{equation}
  \label{eq:energy-turb}
  E_{\R{turb}} = \frac{1}{2} M_{\R{turb}} \langle (\delta{v_t})^2 \rangle
    = \eta_{\R{turb}} E_{\R{merger}},
\end{equation}
where
$M_{\R{turb}}$ is the gas mass enclosed in the turbulent region, i.e.,
within radius of $L_{\R{turb}}$.
By adopting a NFW density profile \citep{navarro1997}, the distribution
of mass in units of the virial mass follows:
\begin{equation}
  \label{eq:mass-dist-nfw}
  f_{\R{nfw}}(s) = \frac{M(s)}{M_{\R{vir}}} =
    \frac{\ln(1 + c s) - c s / (1 + c s)}{\ln(1 + c) - c / (1 + c)},
\end{equation}
where $s = r / r_{\R{vir}}$ is the distance from cluster center in units
of the virial radius, and $c \sim 5$ is the concentration parameter for
clusters \citep{lokas2001}.
Therefore we have:
\begin{equation}
  \label{eq:mass-turb}
  M_{\R{turb}} = f_{\R{gas}} M_{\R{merged}}(<L_{\R{turb}})
    = f_{\R{gas}} f_{\R{nfw}}(1/3) (M_{\R{main}} + M_{\R{sub}}),
\end{equation}
and by substituting this into \autoref{eq:energy-turb} we obtain
the turbulence velocity dispersion as:
\begin{equation}
  \label{eq:turb-velocity-dispersion}
  \langle (\delta{v_t})^2 \rangle =
    \frac{\eta_{\R{turb}}}{f_{\R{nfw}}(1/3)}
    \frac{M_{\R{sub}}}{M_{\R{main}} + M_{\R{sub}}}
    \frac{G M_{\R{main}}}{r^2_{\R{main}}}.
\end{equation}

Generally, the turbulence generated during major mergers is subsonic
with a Mach number of
$\mathcal{M}_{\R{turb}} = \sqrt{\langle (\delta{v_t})^2 \rangle} / c_s \sim \numrange{0.2}{0.5}$,
and its acceleration efficiency to electrons can be described by the
acceleration timescale (i.e., the reciprocal of acceleration rate)
$\tau_{\R{acc}} = 4 D_{\gamma\gamma} / \gamma^2 \sim \SI{0.1}{\Gyr}$,
which is independent of the electron momentum.

%%%
\ssssec{Acceleration Periods}
%%%

The whole merger process may last for \SIrange{1}{2}{\Gyr} \TODO{ref???},
but only when the strong turbulence exists can the turbulent acceleration
become effective, which persists for a relatively short period of time:
$\tau_{\R{turb}} \simeq 2 L_{\R{turb}} / v_{\R{imp}}$,
where $v_{\R{imp}}$ is the relative impact velocity between the merging
clusters \citep{miniati2015}.
Starting from a sufficiently large initial distance with zero velocity,
the relative impact velocity of two merging clusters with mass
$M_{\R{main}}$ and $M_{\R{sub}}$ is given by \citep{sarazin2002,cassano2005}:
\begin{equation}
  \label{eq:v-imp}
  v_{\R{imp}} \simeq \left[
    \frac{2G (M_{\R{main}} + M_{\R{sub}})}{r_{\R{main}}}
    \left( 1 - \frac{1}{\eta_v} \right)\right]^{1/2},
\end{equation}
with $\eta_v \simeq 4 (1 + M_{\R{sub}}/M_{\R{main}})^{1/3}$.
Therefore, we estimate that major mergers generally have an effective
turbulent acceleration duration $\tau_{\R{turb}} \sim \SI{0.5}{\Gyr}$.

For each merger event
$(M^{(i)}_{\R{main}}, M^{(i)}_{\R{sub}}, t^{(i)}_{\R{merger}})$
of a galaxy cluster as simulated in \autoref{sec:merging-history},
we calculate the corresponding turbulent acceleration duration
$\tau^{(i)}_{\R{turb}}$, and solve the above Fokker-Planck equation with
turbulent acceleration turned on together with energy losses among
periods $[t^{(i)}_{\R{merger}}, t^{(i)}_{\R{merger}}+\tau^{(i)}_{\R{turb}}]$,
while there is no acceleration at other times.

%%%
\ssssec{Energy Losses}
%%%

There are several different mechanisms through which the relativistic
electrons in the ICM may lose energies \citep{sarazin1999},
and we consider the following three major mechanisms in our simulation.
The first one is the inverse Compton scattering off the CMB photons:
\begin{equation}
  \label{eq:eloss-ic}
  \left( \diff{\gamma}{t} \right)_{\R{IC}} =
    \num{-4.32e-4} \,\gamma^2 (1+z)^4
    \quad [\si{\per\Gyr}].
\end{equation}

Secondly, the relativistic electrons generate synchrotron radiation
and the loss rate is given by:
\begin{equation}
  \label{eq:eloss-syn}
  \left( \diff{\gamma}{t} \right)_{\R{rad}} =
    \num{-4.10e-5} \,\gamma^2 \left( \frac{B}{\SI{1}{\uG}} \right)^2
    \quad [\si{\per\Gyr}],
\end{equation}
where $B$ is the magnetic field strength in the ICM, which is assumed
to be uniform and has an energy density
$u_B = B^2 / (8\pi) = \eta_B \,\epsilon_{\R{th}}$,
with $\epsilon_{\R{th}}$ being the thermal energy density (see below).

The last mechanism considered here is that electrons lose energy
by interacting with the thermal ICM via Coulomb collisions, which gives
the loss rate:
\begin{equation}
  \label{eq:eloss-coul}
  \left( \diff{\gamma}{t} \right)_{\R{Coul}} =
  \num{-3.79e4} \,n_{\R{th}} \left[ 1 +
    \frac{\ln(\gamma/n_{\R{th}})}{75} \right] \quad [\si{\per\Gyr}],
\end{equation}
where $n_{\R{th}}$ is the thermal number density (see below)
in units of [\si{\per\cm\cubed}].

The inverse Compton scattering and synchrotron radiation dominate
energy losses at the high-energy regime (e.g., $\gamma > 1000$),
and Coulomb collisions are the main energy-loss mechanism for lower
energy electrons (e.g., $\gamma < 100$).
Therefore, at intermediate energies (e.g., $\gamma \sim 300$),
electrons have a very long lifetime and can accumulate in the ICM
as the cluster grows \citep{sarazin1999}, which may become the fossil
electrons to be accelerated \emph{in situ} by turbulence.

%%%
\ssssec{Numerical Method}
%%%

Now with all the coefficients in \autoref{eq:fokkerplanck} been determined,
we can evolve the initial electron spectrum $n_e(\gamma, t=t_0)$ via
solving this Fokker-Planck equation, and derive the \enquote{current}
electron spectrum $n_e(\gamma, t=t_{\R{obs}})$, which corresponds to the
\enquote{observed} redshift $z_{\R{obs}}$ as simulated in \autoref{sec:distributions}.
An efficient numerical method proposed by \citet{chang1970} is utilized
to solve the equation, under the no-flux boundary condition \citep{park1996}.
However, to avoid unphysical pile-up of electrons around the lower boundary,
we define a \enquote{buffer} region below $\gamma_{\R{buf}}$, within which
the spectrum is replaced by extrapolating as a power law, based on the
spectrum above $\gamma_{\R{buf}}$ \citep{donnert2014}.
We adopt a logarithmic grid for $\gamma \in [1, 10^5]$ with 200 cells,
and let the lower buffer region span 10 cells.

%%%
\ssssec{Image Generation}
%%%

After deriving the \enquote{current} electron spectrum
$n_e(\gamma, t_{\R{obs}})$, the synchrotron emissivity at frequency
$\nu$ is given by \citep{rybicki1979}:
\begin{equation}
  \label{sec:jnu-sync}
  % unit: [erg/s/Hz/cm^3]
  J_{\R{syn}}(\nu) = \frac{\sqrt{3} \, e^3 B}{m_e c^2}
    \int_{\gamma_{\R{min}}}^{\gamma_{\R{max}}} \!\!\int_0^{\pi/2}
    n_e(\gamma, t_{\R{obs}}) F\!\left( \frac{\nu}{\nu_c} \right)
    \sin^2 \!\theta \,\D{\theta} \,\D{\gamma},
\end{equation}
where $\theta$ is the pitch angle of electrons,
and $F(\nu/\nu_c) = (\nu/\nu_c) \int_{\nu/\nu_c}^{\infty} K_{5/3}(x) \,\D{x}$
is the synchrotron kernel,
with $K_{5/3}(x)$ the modified Bessel function of 5/3 order,
$\nu_c = (3/2) \,\gamma^2 \nu_L \sin\theta
= 3 e B \gamma^2 \sin\theta / (4\pi m_e c)$
the electron's critical frequency.
The \enquote{observed} radio halo is assumed to be the same size as
the turbulent region \citep[e.g.,][]{vazza2011}:
$r_{\R{halo}} \simeq L_{\R{turb}} \simeq r_{\R{vir}}/3$,
therefore its radio power at a specific frequency (i.e., spectral luminosity) is:
\begin{equation}
  \label{eq:halo-power}
  % unit: [W/Hz]
  P(\nu) = \frac{4\pi}{3} r_{\R{halo}}^3 J_{\R{syn}}(\nu),
\end{equation}
and the flux density measured at frequency $\nu$ is \citep[e.g.,][]{hogg1999}:
\begin{equation}
  \label{eq:halo-flux}
  % unit: [Jy] = 1e-26 [W/Hz/m^2]
  S(\nu) = \frac{(1+z_{\R{obs}}) P(\nu(1+z_{\R{obs}}))}{4\pi D_L^2(z_{\R{obs}})},
\end{equation}
where $D_L(z_{\R{obs}})$ is the luminosity distance to the halo at
redshift $z_{\R{obs}}$.

To generate the image for the simulated radio halos, we adopt an exponential
profile for the azimuthally averaged brightness distribution \citep{murgia2009}:
\begin{equation}
  \label{eq:halo-profile}
  I_{\nu}(\theta) = I_{\nu,0} \exp(-3 \theta / \theta_{\R{halo}}),
\end{equation}
where $\theta = r / D_A(z_{\R{obs}})$ is the angular radius from the halo
center with $D_A(z_{\R{obs}})$ the angular diameter distance to the halo,
and $I_{\nu,0} = 9 S(\nu) / (2\pi \theta^2_{\R{halo}})$ is the central
brightness.
In addition, a random ellipticity $f_e$ and a random rotation angle
$\phi$ are applied to each of the halo images.

%%%
\ssssec{Parameters Tuning}
%%%

Our model has the following 6 free parameters:
(1) fraction of the merger energy transferred into the turbulence:
$\eta_{\R{turb}} \sim 0.1$;
(2) relative energy density in the magnetic field:
$\eta_B \sim \SI{0.1}{\percent}$;
(3) ratio of the total energy of injected electrons to the total thermal
energy: $\eta_e \sim \SI{0.1}{\percent}$;
(4) spectral index of the injected electrons: $s \sim \numrange{2.1}{2.5}$;
(5) efficiency of the ICM plasma instabilities:
$\zeta \sim \numrange{0.1}{0.3}$;
(6) relative energy density in the cosmic rays:
$X_{\R{CR}} \sim \SI{1}{\percent}$.
And all of them have reasonable constraints from either observational
evidences or simulation studies.

To further tune the model parameters in order to achieve good agreements
with the observations, we have collected all current observed radio halos
(\autoref{tab:halos-observed}), and compare the integrated flux function
between our simulated radio halos and the observations.
We have explored various parameter configurations,
and for each configuration we have repeated the simulation for 100??? times
to taking into account the Monte Carlo uncertainties as introduced in
Sections~\ref{sec:distributions} and \ref{sec:merging-history}.
Considering that the current observations are far from complete,
especially in the low-flux regime, our strategy was requiring that the
integrated flux function of simulations agreed with the observed one
in the high-flux regime within simulation uncertainties.
In consequence, we have chosen a set of model parameters with
$\eta_{\R{turb}} = ???$, $\eta_B = ???$, $\eta_e = ???$, $s = ???$,
$\zeta = ???$, and $X_{\R{CR}} = ???$.

\TODO{figures ...}

%========================================================================
\subsection{Other Foreground Components}
\label{sec:fg-other}
%========================================================================

Besides the radio halos in galaxy clusters, we also simulated the sky maps
of the Galactic synchrotron and free-free emissions, and the extragalactic
point sources, following our previous work \citep{wang2010}.
The Galactic synchrotron map was simulated by extrapolating the
Haslam \SI{408}{\MHz} all-sky map as the template to lower frequencies
with a power-law spectrum.
We have made use of the improved Haslam \SI{408}{\MHz} map\footnote{%
  The 2014 reprocessed Haslam \SI{408}{\MHz} map:
  \url{http://www.jb.man.ac.uk/research/cosmos/haslam_map/}.
  We use the $N_{\R{side}} = 2048$ (i.e., pixel size \SI{\sim 1.72}{\arcmin})
  high resolution map achieved by extrapolating the observed power spectrum
  to small scales.}
reprocessed by \citet{remazeilles2015},
and the all-sky synchrotron spectral index map made by \citet{giardino2002}.
The Galactic free-free emission was derived from the \Halpha{} survey data
\citep{finkbeiner2003}, which we corrected for dust absorption first
\citep{dickinson2003}, by employing the close relation between the \Halpha{}
and free-free emissions due to their common origins.
And the extragalactic foreground of point sources is contributed by the
following different types of sources: star-forming and starburst galaxies,
radio-quiet AGNs, Fanaroff-Riley class I and class II AGNs, GHz-peaked
spectrum AGNs, and compact steep spectrum AGNs.

\TODO{figures ...}


%========================================================================
\subsection{EoR Signal}
\label{sec:eor-signal}
%========================================================================

The sky maps of the 21~cm EoR signals were created using the 2016
data release from the Evolution Of 21~cm Structure project\footnote{%
  Evolution Of 21~cm Structure project:
  \url{http://homepage.sns.it/mesinger/EOS.html}},
which were generated with \texttt{21cmFAST} version 2 \citep{mesinger2011}
within a cube of 1.6 comoving \si{\Gpc} and 1024 cells along each side,
and covered a redshift range of \numrange{5.0}{86.5} \citep{mesinger2016}.
We extracted the image slices at needed frequencies (i.e., redshifts) from
the \enquote{light-cone cubes} of the recommended \emph{faint galaxies case},
and then tiled and re-scaled them to have the same sky coverage and pixel
size as our foreground maps.

\TODO{figures ...}


%########################################################################
\section{Observational Simulation}
\label{sec:obs-simu}
%########################################################################

To investigate the practical contamination in the EoR window due to the
galaxy clusters, we need to simulate the \enquote{observed} images of
the above simulated foreground sky and EoR signal.
For this purpose,
we adopt the latest SKA1-Low layout configuration\footnote{%
  SKA-TEL-SKO-0000422: SKA1-low Configuration Coordinates: \url{https://astronomers.skatelescope.org/wp-content/uploads/2016/09/SKA-TEL-SKO-0000422_02_SKA1_LowConfigurationCoordinates-1.pdf}}
released on 2016 May 21.
The SKA1-Low interferometer consists of 512 stations, with 224 of them
randomly distributed within the \enquote{core} of \SI{1000}{\meter} in
diameter, while rest of the stations are grouped into \enquote{clusters}
and arranged on 3 spiral arms extending up to a radius of
\SI{\sim 35}{\kilo\meter}.
Each station has 256 antennas randomly distributed with a minimum separation
of $d_{\R{min}} = \SI{1.5}{\meter}$ inside a circular region of
\SI{35}{\meter} in diameter.
With this layout configuration, the dense core makes the SKA1-Low very
sensitive to the EoR signal, while the long baselines greatly help the
foreground characterization and removal, as well as other scientific goals.
For example, at \SI{158}{\MHz}, the SKA1-Low has a FoV of
\SI{\sim 4 x 4}{\deg}, and angular resolution of \SI{\sim 7}{\arcsec}.

Both the observational simulation and imaging are very computational
intensive, but our computing resources are rather limited, so we can
only choose a set of conservative parameters.
The \SIrange{154}{162}{\MHz} band of our interest is divided into 51
channels with a frequency channel width of \SI{160}{\kilo\hertz}.
The sky maps simulated in previous sections are centered at (R.A., Dec.)
= (\SI{0}{\degree}, \SI{-27}{\degree}), such that the observing field
will pass through the zenith of the interferometer, and this position
is also the center of the \enquote{EoR0} field identified by the MWA
\citep{beardsley2016}.
At each frequency channel, the sky map is tapered by a Tukey window
function to suppress the side-lobe impacts, and then input into the
\texttt{OSKAR} simulator \citep{mort2010} to perform the observation
for \SI{6}{\hour} ($\pm \SI{3}{\hour}$ in Hour Angle) with a time
averaging step of \SI{5}{\minute}.

The simulated visibility data at each frequency channel are combined
to create the \enquote{observed} image cube using the \texttt{WSClean}
imager \citep{offringa2014} with Briggs weighting and joined-channel
deconvolution \citep{offringa2017}.
Finally, we crop out the central \SI{4 x 4}{\deg} to create the 2D
cylindrical and 1D spherical power spectra.


%########################################################################
\section{Results and Discussions}
\label{sec:results}
%########################################################################

\TODO{Write!!!}

\NOTE{Collect all the parameters (e.g., fg21sim, SimFast21, OSKAR, WSClean)
  as well as the input data/templates into to the Appendix!
  Also create a page in the software repository to hold further information.}

On the foregrounds simulation part, compare our simulated radio halos with
current observations as well as previous results,
e.g., halo luminosity function, (and ???).

Then we evaluate the impacts of radio halos on EoR detection in both the
image domain and the (2D) power spectrum domain:
\begin{description}
  \item[Image]
    Take the above simulated image cube, subtract the spectral-smooth
    foregrounds fitting a smooth polynomial along the frequency axis,
    evaluate the loss of EoR signals (use image-domain indexes???
    can also use the total power spectrum???)

    May consider two cases: (1) only radio halos is superimposed onto the
    EoR signals; (2) radio halos as well as Galactic foregrounds are
    added to the EoR signals.

  \item[Power spectrum]
    Take the above simulated image cube, calculate the 2D
    (i.e., cylindrically averaged) power spectrum, evaluate the foregrounds
    contamination within the (theoretically predicted) EoR window,
    compare to the case that only EoR signals exist in the EoR windows.
\end{description}


%% TODO: future works/improvements:
One of the main purposes of our foregrounds simulation is to support
the development and evaluation of foregrounds removal or separation
methods in our forthcoming works.
We will combine our foregrounds simulation with interferometric
observational simulation, to generate the \enquote{observed} visibility
data as well as images (e.g., by using the \textsc{clean} algorithm).
In this way, we incorporate the realistic instrumental effects into
our simulation pipeline, making our developed methods more versatile
and prepare for future real observations.

In addition to the foregrounds removal aiming at the EoR signal
detection, the methods to effectively identify and extract the faint
extended radio emissions (e.g., halos, relics) from observation images
is of the same importance.
With future more sensitive observations combined with a better detection
method, we are able to build a big sample of galaxy clusters, which
may provide us with plenty of opportunities to study them in the radio
regime.

Moreover, there is still much room to improve the current foregrounds
simulation, e.g., taking radio relics into account, more realistic
source morphologies by utilizing the deep learning techniques,
full polarization support.
We will continue to improve our foregrounds simulation works and make
them available in our public software \texttt{FG21sim} in the future.

In addition, galaxy clusters are also important objects of study in their
own right:
\begin{itemize}
  \item elements to form the large-scale structures, key to understand the
    structure formation and evolution of the Universe;
  \item merger-induced turbulence and shocks accelerate electrons to produce
    extended halos and relics, respectively, enable us to trace cluster
    mergers and formations with radio techniques;
  \item great laboratory to study the origin of cosmic rays, particle
    acceleration, magnetic growth and distribution, etc.
\end{itemize}

Due to their low surface brightness and steep spectra, these extended
radio emissions from galaxy clusters are difficult to find at middle-
or high-frequency ranges (e.g., \SI{1.4}{\GHz}).
However, they become much brighter and thus easier to be detected in
the low-frequency radio band.  The large field of view (FoV) of the
low-frequency interferometer also helps discover them more efficiently.
Owing to its excellent sensitivity, the SKA will possibly discover more
than several thousands of halos and relics, providing us with invaluable
data to understand and solve the puzzles with regard to galaxy clusters.
Furthermore, the faint radio sky can be very different, and we may have
unexpected findings as we go deeper \citep{padovani2016rev,herreraRuiz2017}.


%%%%%%%%%%%%%%%%%%%%%%%%%%%%%%%%%%%%%%%%%%%%%%%%%%%%%%%%%%%%%%%%%%%%%%%%%
\section{Summary}
\label{sec:summary}

\TODO{Write!!!}


%%%%%%%%%%%%%%%%%%%%%%%%%%%%%%%%%%%%%%%%%%%%%%%%%%%%%%%%%%%%%%%%%%%%%%%%%
\acknowledgments

\TODO{update}
We give thanks
to Mathieu Remazeilles for providing us with the high-resolution Haslam
\SI{408}{\MHz} Galactic synchrotron map,
to Giovanna Giardino for the all-sky Galactic synchrotron spectral index map,
to Fred Dulwich for sharing the latest SKA1-Low layout configuration as well
as the guidance on using the \texttt{OSKAR} simulator,
to Andr\'e Offringa for the help on radio imaging with \texttt{WSClean}.
We also thank Emma Chapman, and Uri Keshet for their help.
This work is supported by the National Science Foundation of China
(grant Nos. 11433002, xxx).


%% To help institutions obtain information on the effectiveness of their
%% telescopes the AAS Journals has created a group of keywords for telescope
%% facilities:
%% http://journals.aas.org/authors/aastex/facility.html
%
%% Following the acknowledgments section, use the following syntax and the
%% \facility{} or \facilities{} macros to list the keywords of facilities used
%% in the research for the paper.  Each keyword is check against the master
%% list during copy editing.  Individual instruments can be provided in
%% parentheses, after the keyword, but they are not verified.

\vspace{5mm}

% \facilities{???}

%% Similar to \facility{}, there is the optional \software command to allow
%% authors a place to specify which programs were used during the creation of
%% the manuscript. Authors should list each code and include either a
%% citation or url to the code inside ()s when available.

\software{
  OSKAR \citep{mort2010},
  WSClean \citep{offringa2014},
  Astropy \citep{astropy2013,astropy2018},
  NumPy (\url{http://www.numpy.org/}),
  SciPy (\url{https://scipy.org/}),
  IPython (\url{https://ipython.org/}),
  Jupyter Notebook (\url{https://jupyter.org/}),
  Matplotlib (\url{https://matplotlib.org/}),
  Pandas (\url{https://pandas.pydata.org/})
}


%%%%%%%%%%%%%%%%%%%%%%%%%%%%%%%%%%%%%%%%%%%%%%%%%%%%%%%%%%%%%%%%%%%%%%%%%
%%
%% Appendix material should be preceded with a single \appendix command.
%% There should be a \section command for each appendix. Mark appendix
%% subsections with the same markup you use in the main body of the paper.
%%
%% Each Appendix (indicated with \section) will be lettered A, B, C, etc.
%% The equation counter will reset when it encounters the \appendix
%% command and will number appendix equations (A1), (A2), etc. The
%% Figure and Table counter will not reset.
%%
\appendix

%########################################################################
\section{Collection of Current Observed Radio Halos}
\label{sec:halos-collection}
%########################################################################

\startlongtable
\begin{deluxetable*}{lcccr@{$\,\pm\,$}lr@{$\,\pm\,$}lp{12em}l}
\tabletypesize{\scriptsize}
%\tablewidth{0pt}  % set to table's natural width
\tablecaption{Collection of Current Observed Radio Halos%
  \tablenotemark{a}%
  \label{tab:halos-observed}%  % NOTE: must be placed inside the caption
}
\tablehead{
  \colhead{Cluster} &
  \colhead{Redshift} &
  \colhead{kpc/arcsec} &
  \colhead{Size} &
  \multicolumn{2}{c}{$S_{\SI{1.4}{\GHz}}$} &
  \multicolumn{2}{c}{$P_{\SI{1.4}{\GHz}}$} &
  \colhead{Notes} &
  \colhead{References} \\
  \colhead{} &  % cluster
  \colhead{} &  % redshift
  \colhead{} &  % kpc/arcsec
  \colhead{(\si{\Mpc})} &  % size
  \multicolumn{2}{c}{(\si{\mJy})} &  % flux
  \multicolumn{2}{c}{(\SI{e24}{\watt\per\hertz})} &  % power
  \colhead{} &  % notes
  \colhead{} \\
  \colhead{(1)} &
  \colhead{(2)} &
  \colhead{(3)} &
  \colhead{(4)} &
  \multicolumn{2}{c}{(5)} &
  \multicolumn{2}{c}{(6)} &
  \colhead{(7)} &
  \colhead{(8)}
}

\startdata
1E 0657$-$56 & 0.2960 & 4.38 & 1.48 & 78.0 & 5.0 & 21.33 & 1.49 &  & \citet{liang2000}  \\
Abell 141 & 0.2300 & 3.64 & 1.20 & 1.3 & 0.1 & 0.25 & 0.02 &  & \citet{duchesne2017}  \\
Abell 209 & 0.2060 & 3.34 & 1.40 & 16.9 & 1.0 & 2.04 &  & With a relic candidate & \citet{giovannini2009}  \\
Abell 399 & 0.0718 & 1.35 & 0.57 & 16.0 & 2.0 & 0.20 &  & Double with Abell 401 & \citet{murgia2010}  \\
Abell 401 & 0.0737 & 1.38 & 0.49 & 17.0 & 1.0 & 0.20 &  & Double with Abell 399 & \citet{bacchi2003}  \\
Abell 520 & 0.1990 & 3.25 & 0.99 & 34.4 & 1.5 & 3.17 &  &  & \citet{govoni2001}  \\
Abell 521 & 0.2533 & 3.91 & 1.17 & 5.9 & 0.5 & 1.12 &  & With a relic & \citet{giovannini2009}  \\
Abell 523 & 0.1000 & 1.82 & 1.30 & 59.0 & 5.0 & 1.47 &  &  & \citet{giovannini2011}  \\
Abell 545 & 0.1540 & 2.64 & 0.81 & 23.0 & 1.0 & 1.25 &  &  & \citet{bacchi2003}  \\
Abell 665 & 0.1818 & 3.03 & 1.66 & 43.1 & 2.2 & 3.28 &  &  & \citet{giovannini2000}  \\
Abell 697 & 0.2820 & 4.23 & 0.75 & 5.2 & 0.5 & 2.20 &  &  & \citet{vanWeeren2011}  \\
Abell 746 & 0.2320 & 3.67 & 0.85 & 18.0 & 4.0 & 3.80 &  & With a relic & \citet{vanWeeren2011}  \\
Abell 754 & 0.0542 & 1.04 & 0.95 & 86.0 & 4.0 & 0.56 &  & With a relic & \citet{bacchi2003}  \\
Abell 773 & 0.2170 & 3.48 & 1.13 & 12.7 & 1.3 & 1.39 &  &  & \citet{govoni2001}  \\
Abell 781 & 0.3004 & 4.42 & 1.60 & 20.5 & 5.0 & 5.90 &  & With a relic candidate & \citet{govoni2011}  \\
Abell 800 & 0.2223 & 3.55 & 1.28 & 10.6 & 0.9 & 1.52 &  &  & \citet{govoni2012}  \\
Abell 851 & 0.4069 & 5.40 & 1.08 & 3.7 & 0.3 & 2.14 &  &  & \citet{giovannini2009}  \\
Abell 1132 & 0.1369 & 2.39 & 0.74 & 3.3 & 1.5 & 0.16 & 0.07 &  & \citet{wilber2018}  \\
Abell 1213 & 0.0469 & 0.91 & 0.22 & 72.2 & 3.5 & 0.36 &  &  & \citet{giovannini2009}  \\
Abell 1300 & 0.3100 & 4.52 & 0.92 & 20.0 & 2.0 & 2.99 &  & With a relic & \citet{reid1999}  \\
Abell 1351 & 0.3220 & 4.64 & 1.08 & 32.4 & ... & 11.37 &  &  & \citet{giacintucci2011b}  \\
Abell 1443 & 0.2700 & 4.10 & 1.10 & 11.0 & 1.1 & 2.53 & 0.30 & Halo candidate & \citet{bonafede2015}  \\
Abell 1451 & 0.1989 & 3.25 & 0.74 & 5.4 & 0.5 & 0.62 & 0.07 & With a relic candidate & \citet{cuciti2018}  \\
Abell 1550 & 0.2540 & 3.92 & 1.41 & 7.7 & 1.6 & 1.49 &  &  & \citet{govoni2012}  \\
Abell 1656 & 0.0232 & 0.46 & 0.58 & 530.0 & 50.0 & 0.31 &  & With a relic candidate & \citet{kim1990}  \\
Abell 1682 & 0.2272 & 3.61 & 0.85 & 14.9 & ... & 2.13 &  & Halo candidate & \citet{venturi2008}  \\
Abell 1689 & 0.1832 & 3.05 & 0.73 & 10.0 & 2.9 & 0.92 &  &  & \citet{vacca2011}  \\
Abell 1758A & 0.2790 & 4.20 & 0.63 & 3.9 & 0.4 & 0.93 &  & With a relic & \citet{giovannini2009}  \\
Abell 1914 & 0.1712 & 2.88 & 1.16 & 64.0 & 3.0 & 4.32 &  &  & \citet{bacchi2003}  \\
Abell 1995 & 0.3186 & 4.61 & 0.83 & 4.1 & 0.7 & 1.35 &  &  & \citet{giovannini2009}  \\
Abell 2034 & 0.1130 & 2.03 & 0.60 & 7.3 & 2.0 & 0.28 &  & With a relic & \citet{vanWeeren2011}  \\
Abell 2061 & 0.0784 & 1.46 & 1.68 & 16.9 & 4.2 & 0.25 &  & With a relic & \citet{farnsworth2013}  \\
Abell 2065 & 0.0726 & 1.36 & 1.08 & 32.9 & 11.0 & 0.41 &  &  & \citet{farnsworth2013}  \\
Abell 2069 & 0.1160 & 2.08 & 0.90 & 6.2 & 2.2 & 0.25 & 0.05 & Halo may have two components & \citet{drabent2015}  \\
Abell 2142 & 0.0909 & 1.67 & 0.99 & 11.8 & 0.8 & 1.12 &  & Halo has two components & \citet{venturi2017}  \\
Abell 2163 & 0.2030 & 3.31 & 2.04 & 155.0 & 2.0 & 14.93 & 0.20 & With a relic & \citet{feretti2001}  \\
Abell 2218 & 0.1710 & 2.88 & 0.35 & 4.7 & 0.1 & 0.32 &  &  & \citet{giovannini2000}  \\
Abell 2219 & 0.2256 & 3.59 & 1.54 & 81.0 & 4.0 & 9.72 &  &  & \citet{bacchi2003}  \\
Abell 2254 & 0.1780 & 2.98 & 0.85 & 33.7 & 1.8 & 2.43 &  &  & \citet{govoni2001}  \\
Abell 2255 & 0.0806 & 1.50 & 0.90 & 56.0 & 3.0 & 0.87 &  & With a relic & \citet{govoni2005}  \\
Abell 2256 & 0.0594 & 1.13 & 0.81 & 103.4 & 1.1 & 0.82 &  & With a relic & \citet{clarke2006}  \\
Abell 2294 & 0.1780 & 2.98 & 0.54 & 5.8 & 0.5 & 0.51 &  &  & \citet{giovannini2009}  \\
Abell 2319 & 0.0524 & 1.01 & 0.93 & 153.0 & 8.0 & 0.54 &  &  & \citet{feretti1997}  \\
Abell 2680 & 0.1901 & 3.14 & 0.57 & 1.8 & 0.6 & 0.16 & 0.05 & Halo candidate & \citet{duchesne2017}  \\
Abell 2693 & 0.1730 & 2.91 & 0.65 & 7.7 & 0.9 & 0.61 & 0.07 & Halo candidate & \citet{duchesne2017}  \\
Abell 2744 & 0.3080 & 4.50 & 1.62 & 57.1 & 2.9 & 12.89 &  & With a relic & \citet{govoni2001}  \\
Abell 2811 & 0.1080 & 1.95 & 0.48 & 3.4 & 0.7 & 0.10 & 0.02 &  & \citet{duchesne2017}  \\
Abell 3411 & 0.1687 & 2.85 & 0.90 & 4.8 & 0.5 & 0.46 &  & With a relic & \citet{vanWeeren2013}  \\
Abell 3562 & 0.0480 & 0.93 & 0.44 & 20.0 & 2.0 & 0.10 &  &  & \citet{venturi2003}  \\
Abell 3888 & 0.1510 & 2.60 & 0.99 & 27.6 & 3.1 & 1.85 & 0.19 &  & \citet{shakouri2016}  \\
Abell S84 & 0.1080 & 1.95 & 0.49 & 2.1 & 0.3 & 0.06 & 0.01 & Halo candidate & \citet{duchesne2017}  \\
Abell S1121 & 0.3580 & 4.98 & 1.25 & 9.8 & 3.1 & 4.54 & 1.44 &  & \citet{duchesne2017}  \\
ACT-CL J0102$-$4915 & 0.8700 & 7.73 & 2.17 & 10.7 & 1.1 & 44.43 & 1.28 & With double relics & \citet{lindner2014}  \\
ACT-CL J0256.5+0006 & 0.3430 & 4.84 & 0.79 & 2.1 & 0.5 & 0.97 & 0.29 &  & \citet{knowles2016}  \\
CIZA J0107.7+5408 & 0.1066 & 1.93 & 1.10 & 55.0 & 5.0 & 1.80 &  &  & \citet{vanWeeren2011}  \\
CIZA J0638.1+4747 & 0.1740 & 2.92 & 0.59 & 3.6 & 0.2 & 0.30 & 0.02 &  & \citet{cuciti2018}  \\
CIZA J1938.3+5409 & 0.2600 & 3.99 & 0.72 & 1.6 & 0.2 & 0.36 & 0.05 &  & \citet{bonafede2015}  \\
CIZA J2242.8+5301 & 0.1921 & 3.16 & 1.77 & 33.5 & 6.2 & 3.40 & 0.97 & With double relics & \citet{govoni2012}  \\
ClG 0016+16 & 0.5456 & 6.37 & 0.77 & 5.5 & ... & 4.42 &  &  & \citet{giovannini2000}  \\
ClG 0217+70 & 0.0655 & 1.24 & 0.73 & 58.6 & 0.9 & 0.54 & 0.01 & With double relics & \citet{brown2011}  \\
ClG 1446+26 & 0.3700 & 5.09 & 1.22 & 7.7 & 2.6 & 3.57 &  & With a relic & \citet{govoni2012}  \\
ClG 1821+64 & 0.2990 & 4.41 & 1.10 & 13.0 & 0.8 & 3.70 & 0.10 &  & \citet{bonafede2014b}  \\
MACS J0416.1$-$2403 & 0.3960 & 5.31 & 0.64 & 1.7 & 0.8 & 1.26 & 0.29 &  & \citet{pandeyPommier2015}  \\
MACS J0520.7$-$1328 & 0.3400 & 4.81 & 0.80 & 9.0 & 1.6 & 3.38 &  & Halo candidate & \citet{macario2014}  \\
MACS J0553.4$-$3342 & 0.4070 & 5.40 & 1.32 & 9.2 & 0.7 & 6.73 & 0.61 &  & \citet{bonafede2012}  \\
MACS J0717.5+3745 & 0.5458 & 6.37 & 1.20 & 118.0 & 5.0 & 50.00 & 10.00 & With a relic & \citet{vanWeeren2009}  \\
MACS J0949.8+1708 & 0.3825 & 5.20 & 1.04 & 3.1 & 0.3 & 1.63 & 0.15 &  & \citet{bonafede2015}  \\
MACS J1149.5+2223 & 0.5444 & 6.36 & 1.32 & 1.2 & 0.5 & 1.95 & 0.93 & Halo candidate; with double relics & \citet{bonafede2012}  \\
MACS J1752.0+4440 & 0.3660 & 5.05 & 1.65 & 14.2 & 1.4 & 9.50 &  & With double relics & \citet{vanWeeren2012}  \\
MACS J2243.3$-$0935 & 0.4470 & 5.71 & 0.91 & 3.1 & 0.6 & 3.11 & 0.58 & With a relic candidate & \citet{cantwell2016}  \\
PLCK G147.3$-$16.6 & 0.6500 & 6.92 & 1.80 & 2.5 & 0.4 & 5.10 & 0.80 &  & \citet{vanWeeren2014}  \\
PLCK G171.9$-$40.7 & 0.2700 & 4.10 & 0.99 & 18.0 & 2.0 & 4.76 & 0.10 &  & \citet{giacintucci2013}  \\
PLCK G285.0$-$23.7 & 0.3900 & 5.26 & 0.73 & 2.9 & 0.4 & 1.67 & 0.21 &  & \citet{martinezAviles2016}  \\
PLCK G287.0+32.9 & 0.3900 & 5.26 & 1.30 & 8.8 & 0.9 & 5.10 &  & With double relics & \citet{bonafede2014a}  \\
PSZ1 G108.18$-$11.53 & 0.3347 & 4.77 & 0.84 & 6.8 & 0.2 & 2.72 & 0.10 & With double relics & \citet{deGasperin2015}  \\
RXC J1234.2+0947 & 0.2290 & 3.63 & 0.92 & 2.0 & ... & 0.30 &  & Halo candidate & \citet{govoni2012}  \\
RXC J1314.4$-$2515 & 0.2474 & 3.85 & 1.27 & 20.3 & 0.8 & 1.45 &  & With double relics & \citet{feretti2005}  \\
RXC J1514.9$-$1523 & 0.2226 & 3.55 & 1.38 & 10.0 & 2.0 & 1.65 &  &  & \citet{giacintucci2011a}  \\
RXC J2003.5$-$2323 & 0.3171 & 4.59 & 1.38 & 35.0 & 2.0 & 11.96 &  &  & \citet{giacintucci2009}  \\
RXC J2351.0$-$1954 & 0.2477 & 3.85 & 0.64 & 4.5 & 0.9 & 0.89 & 0.18 & Halo candidate & \citet{duchesne2017}  \\
\enddata

\tablenotetext{a}{As of 2018 January.}
\tablecomments{%
  \uline{Column (1)}: ???
}
\end{deluxetable*}

%%%%%%%%%%%%%%%%%%%%%%%%%%%%%%%%%%%%%%%%%%%%%%%%%%%%%%%%%%%%%%%%%%%%%%%%%
\section{Useful Formulas}
\label{sec:formulas}


On the other hand, the critical linear overdensity as a function of
redshift $z$ is \citep{kitayama1996,randall2002}:
\begin{equation}
  \label{eq:delta-crit}
  \delta_c(z) = \frac{D(z=0)}{D(z)}
  \left[ \frac{3 (12\pi)^{2/3}}{20} \right]
  \left[1 + 0.0123 \log_{10} \Omega_f(z) \right],
\end{equation}
where $\Omega_f(z)$ is the mass density ratio at redshift $z$:
\begin{equation}
  \label{eq:omega-fz}
  \Omega_f(z) = \frac{\Omega_m(1+z)^3}{\Omega_m(1+z)^3 + \Omega_{\Lambda}},
\end{equation}
and $D(z)$ is the growth factor given by
\citep[Equation~(13.6)]{peebles1980}:
\begin{equation}
  \label{eq:growth-factor}
  D(x) = \frac{(x^3 + 2)^{1/2}}{x^{3/2}}
    \int_0^x y^{3/2} (y^3 + 2)^{-3/2} \,\D{y},
\end{equation}
with $x_0 \equiv (2\Omega_{\Lambda}/\Omega_m)^{1/3}$ and
$x = x_0 / (1+z)$.

The virial radius of a cluster is:
\begin{equation}
  \label{eq:radius-virial}
  R_{\R{vir}} = \left[
    \frac{3 M_{\R{vir}}}{4\pi \Delta_{\R{vir}}(z) \rho_{\R{crit}}(z)}
  \right]^{1/3},
\end{equation}
where $M_{\R{vir}}$ is the virial mass (can also be regarded as
the total mass) of the cluster,
$\rho_{\R{crit}}(z)$ is the critical density at redshift $z$,
and $\Delta_{\R{vir}}(z)$ is the average overdensity of the cluster
at redshift $z$ given by \citep{kitayama1996,cassano2005}:
\begin{equation}
  \label{eq:delta-vir}
  \Delta_{\R{vir}}(z) = 18\pi^2 \left[ 1 + 0.4093 \, w(z)^{0.9052} \right],
\end{equation}
where $w(z) \equiv \Omega_f^{-1}(z) - 1$.

In a flat \lcdm{} cosmology as we adopted, the age of the Universe
at redshift $z$ takes an analytical form
\citep[their Equation~(18)]{thomas2000}:
\begin{align}
  \label{eq:universe-age}
  t(z; \Omega_m, \Omega_{\Lambda}=1-\Omega_m)
    & = \frac{1}{H_0} \int_z^{\infty}
      \frac{\D{z'}}{(1+z')\sqrt{1 + \Omega_m z' (3+3z'+z'^2)}} \nonumber \\
    & = \frac{2}{3 H_0 \sqrt{1-\Omega_m}} \sinh^{-1} \!\left(
      \sqrt{\frac{\Omega_m^{-1} - 1}{(1+z)^3}} \right).
\end{align}


%%%%%%%%%%%%%%%%%%%%%%%%%%%%%%%%%%%%%%%%%%%%%%%%%%%%%%%%%%%%%%%%%%%%%%%%%
\bibliography{references}


%% Include this line if you are using the \added, \replaced, \deleted
%% commands to see a summary list of all changes at the end of the article.
%\listofchanges

\end{document}

%% EOF
