%%
%% Weitian LI, Haiguang XU, et al.
%% Department of Astronomy, Shanghai Jiao Tong University
%%

%%%%%%%%%%%
%% Preamble
%%%%%%%%%%%

\documentclass[modern]{aastex61}
%%
%% Defaults: single spaced, 10 point font article
%%
%%  twocolumn   : two text columns, 10 point font, single spaced article.
%%                This is the most compact and represent the final published
%%                derived PDF copy of the accepted manuscript from the publisher
%%  manuscript  : one text column, 12 point font, double spaced article.
%%  preprint    : one text column, 12 point font, single spaced article.
%%  preprint2   : two text columns, 12 point font, single spaced article.
%%  modern      : a stylish, single text column, 12 point font, article with
%%                wider left and right margins. This uses the Daniel
%%                Foreman-Mackey and David Hogg design.
%%
%% Note that you can submit to the AAS Journals in any of these 6 styles.
%%
%% There are other optional arguments one can envoke to allow other stylistic
%% actions. The available options are:
%%
%%  astrosymb    : Loads Astrosymb font and define \astrocommands.
%%  tighten      : Makes baselineskip slightly smaller, only works with
%%                 the twocolumn substyle.
%%  times        : uses times font instead of the default
%%  linenumbers  : turn on lineno package.
%%  trackchanges : required to see the revision mark up and print its output
%%  longauthor   : Do not use the more compressed footnote style (default) for
%%                 the author/collaboration/affiliations. Instead print all
%%                 affiliation information after each name. Creates a much
%%                 long author list but may be desirable for short author papers

%% Custom macros
\newcommand{\vdag}{(v)^\dagger}
\newcommand\aastex{AAS\TeX}
\newcommand\latex{La\TeX}
\newcommand{\etal}{\textit{et al.}}

\received{xxx, 2017}
\revised{xxx, 2017}
\accepted{xxx, 2017}
\submitjournal{ApJ}

\shorttitle{Low-frequency Radio Foregrounds Simulation}
\shortauthors{Li \etal}

%% You can add a light gray and diagonal water-mark to the first page
%% with this command:
\watermark{DRAFT}
%% where "text", e.g. DRAFT, is the text to appear.  If the text is
%% long you can control the water-mark size with:
%  \setwatermarkfontsize{dimension}
%% where dimension is any recognized LaTeX dimension, e.g. pt, in, etc.


%%%%%%%%%%%
%% Document
%%%%%%%%%%%

\begin{document}

\title{Low-frequency Radio Foregrounds Simulation: Galaxy Clusters}

%% The \author command is the same as before except it now takes an optional
%% argument which is the 16 digit ORCID. The syntax is:
%% \author[xxxx-xxxx-xxxx-xxxx]{Author Name}
%%
%% This will hyperlink the author name to the author's ORCID page. Note that
%% during compilation, LaTeX will do some limited checking of the format of
%% the ID to make sure it is valid.
%%
%% Use \affiliation for affiliation information. The old \affil is now aliased
%% to \affiliation. AASTeX v6.1 will automatically index these in the header.
%% When a duplicate is found its index will be the same as its previous entry.
%%
%% Note that \altaffilmark and \altaffiltext have been removed and thus
%% can not be used to document secondary affiliations. If they are used latex
%% will issue a specific error message and quit. Please use multiple
%% \affiliation calls for to document more than one affiliation.
%%
%% The new \altaffiliation can be used to indicate some secondary information
%% such as fellowships. This command produces a non-numeric footnote that is
%% set away from the numeric \affiliation footnotes.  NOTE that if an
%% \altaffiliation command is used it must come BEFORE the \affiliation call,
%% right after the \author command, in order to place the footnotes in
%% the proper location.
%%
%% Use \email to set provide email addresses. Each \email will appear on its
%% own line so you can put multiple email address in one \email call. A new
%% \correspondingauthor command is available in V6.1 to identify the
%% corresponding author of the manuscript. It is the author's responsibility
%% to make sure this name is also in the author list.
%%
%% While authors can be grouped inside the same \author and \affiliation
%% commands it is better to have a single author for each. This allows for
%% one to exploit all the new benefits and should make book-keeping easier.
%%
%% If done correctly the peer review system will be able to
%% automatically put the author and affiliation information from the manuscript
%% and save the corresponding author the trouble of entering it by hand.

\correspondingauthor{Haiguang Xu}
\email{hgxu@sjtu.edu.cn}

\author[0000-0002-7527-380X]{Weitian Li}
\email{liweitianux@sjtu.edu.cn}
\affil{Department of Astronomy, Shanghai Jiao Tong University,
  800 Dongchuan Road, Shanghai 200240, China}

\author{Haiguang Xu}
\affil{Department of Astronomy, Shanghai Jiao Tong University,
  800 Dongchuan Road, Shanghai 200240, China}


%%
%% Abstract
%%

\begin{abstract}
TODO!!!
\end{abstract}

%% See the online documentation for the full list of available subject
%% keywords and the rules for their use.
%% http://journals.aas.org/authors/keywords2013.html
\keywords{
  dark ages, reionization, first stars ---
  galaxies: clusters: intracluster medium ---
  galaxies: halos ---
  radio continuum: galaxies
}

%%%%%%%
%% Body
%%%%%%%

\section{Introduction}
\label{sec:intro}

TODO!!!


\section{Foregrounds Simulation}
\label{sec:fg-simu}

\subsection{Galactic Synchrotron Emission}
\label{sec:fg-gsync}


\section{Discussions}
\label{sec:discussions}

TODO!!!


\section{Conclusion}
\label{sec:conclusion}

TODO!!!

%% If you wish to include an acknowledgments section in your paper,
%% separate it off from the body of the text using the \acknowledgments
%% command.
\acknowledgments

TODO!!!
We thank all the people that have made this AASTeX what it is today.  This
includes but not limited to Bob Hanisch, Chris Biemesderfer, Lee Brotzman,
Pierre Landau, Arthur Ogawa, Maxim Markevitch, Alexey Vikhlinin and Amy
Hendrickson. Also special thanks to David Hogg and Daniel Foreman-Mackey
for the new ``modern'' style design. It is super cool!


%% To help institutions obtain information on the effectiveness of their
%% telescopes the AAS Journals has created a group of keywords for telescope
%% facilities.
%
%% Following the acknowledgments section, use the following syntax and the
%% \facility{} or \facilities{} macros to list the keywords of facilities used
%% in the research for the paper.  Each keyword is check against the master
%% list during copy editing.  Individual instruments can be provided in
%% parentheses, after the keyword, but they are not verified.

\vspace{5mm}

\facilities{???}

%% Similar to \facility{}, there is the optional \software command to allow
%% authors a place to specify which programs were used during the creation of
%% the manusscript. Authors should list each code and include either a
%% citation or url to the code inside ()s when available.

\software{
  AstroPy \citep{2013A&A...558A..33A},
  NumPy (\url{http://www.numpy.org/}),
  Vim (\url{http://www.vim.org/}),
  Emacs (\url{https://www.gnu.org/s/emacs/})
}


%% Appendix material should be preceded with a single \appendix command.
%% There should be a \section command for each appendix. Mark appendix
%% subsections with the same markup you use in the main body of the paper.

%% Each Appendix (indicated with \section) will be lettered A, B, C, etc.
%% The equation counter will reset when it encounters the \appendix
%% command and will number appendix equations (A1), (A2), etc. The
%% Figure and Table counter will not reset.

% \appendix


%%%%%%%%%%%%%
%% References
%%%%%%%%%%%%%

\begin{thebibliography}{}
\bibitem[Astropy Collaboration et al.(2013)]
  {2013A&A...558A..33A}
  Astropy Collaboration, Robitaille, T.~P., Tollerud, E.~J., \etal{}
  2013, \aap, 558, A33
\end{thebibliography}


%% Include this line if you are using the \added, \replaced, \deleted
%% commands to see a summary list of all changes at the end of the article.
%\listofchanges

\end{document}

%% EOF
