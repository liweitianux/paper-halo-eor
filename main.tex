%%
%% Copyright (c) 2017 The Authors.  All Rights Reserved.
%%
%% Weitian LI <liweitianux@sjtu.edu.cn>, Haiguang XU <hgxu@sjtu.edu.cn>, et al.
%% Department of Astronomy, Shanghai Jiao Tong University, Shanghai, China.
%%
%% 2017-07-18
%%

\documentclass[modern]{aastex61}
%%
%% Defaults: single spaced, 10 point font article
%%
%%  twocolumn   : two text columns, 10 point font, single spaced article.
%%                This is the most compact and represent the final published
%%                derived PDF copy of the accepted manuscript from the publisher
%%  manuscript  : one text column, 12 point font, double spaced article.
%%  preprint    : one text column, 12 point font, single spaced article.
%%  preprint2   : two text columns, 12 point font, single spaced article.
%%  modern      : a stylish, single text column, 12 point font, article with
%%                wider left and right margins. This uses the Daniel
%%                Foreman-Mackey and David Hogg design.
%%
%% Note that you can submit to the AAS Journals in any of these 6 styles.
%%
%% There are other optional arguments one can envoke to allow other stylistic
%% actions. The available options are:
%%
%%  astrosymb    : Loads Astrosymb font and define \astrocommands.
%%  tighten      : Makes baselineskip slightly smaller, only works with
%%                 the twocolumn substyle.
%%  times        : uses times font instead of the default
%%  linenumbers  : turn on lineno package.
%%  trackchanges : required to see the revision mark up and print its output
%%  longauthor   : Do not use the more compressed footnote style (default) for
%%                 the author/collaboration/affiliations. Instead print all
%%                 affiliation information after each name. Creates a much
%%                 long author list but may be desirable for short author papers

%% Extra packages
\usepackage{amsmath}
\usepackage{xfrac}  % split-level fractions
\usepackage{bm}  % bold math
\usepackage{savesym}

\savesymbol{tablenum}  % workaround symbol conflict
                       % credit: https://tex.stackexchange.com/a/269074
\usepackage{siunitx}  % typeset units; from `texlive-science'
\restoresymbol{SIX}{tablenum}  % `tablenum' from `siunitx' becomes `SIXtablenum'

%% Custom macros
\newcommand{\etal}{\textit{et~al.}}
\newcommand{\R}[1]{\mathrm{#1}}
\newcommand{\D}[1]{\R{d} #1}
\newcommand{\diff}[2]{\frac{\D{#1}}{\D{#2}}}
\newcommand{\pdiff}[2]{\frac{\partial #1}{\partial #2}}
\newcommand{\lcdm}{$\Lambda$CDM}
\newcommand{\Halpha}{\text{H$\alpha$}}
\newcommand{\Hi}{H\textsc{i}}
\newcommand{\Hii}{H\textsc{ii}}
%
\newcommand{\NOTE}[1]{\textcolor{violet}{\textbf{NOTE:}}~\uwave{#1}}
\newcommand{\TODO}[1]{\textcolor{magenta}{\textbf{TODO:}}~\uuline{#1}}
\newcommand{\OK}[1]{\textcolor{cyan}{\textbf{OK:}}~\uline{#1}}

%% Custom math operators (requires `amsmath' package)
\DeclareMathOperator{\erf}{erf}
\DeclareMathOperator{\erfc}{erfc}

%% Settings for package `siunitx'
% credit: https://tex.stackexchange.com/a/194042
\sisetup{range-phrase=\text{--}}
%
%% Custom units (requires `siunitx' package)
\DeclareSIUnit\arcsec{arcsec}
\DeclareSIUnit\arcmin{arcmin}
\DeclareSIUnit\erg{erg}
\DeclareSIUnit\gauss{G}
\DeclareSIUnit\hubble{\text{$h$}}
\DeclareSIUnit\jansky{Jy}
\DeclareSIUnit\lightyear{ly}
\DeclareSIUnit\parsec{pc}
\DeclareSIUnit\rayleigh{Rayleigh}
\DeclareSIUnit\solarmass{\text{M$_{\odot}$}}
\DeclareSIUnit\year{yr}
%
\DeclareSIUnit\keV{\kilo\electronvolt}
\DeclareSIUnit\uG{\micro\gauss}
\DeclareSIUnit\Gyr{\giga\year}
\DeclareSIUnit\MHz{\mega\hertz}
\DeclareSIUnit\Mpc{\mega\parsec}


%% Journal information
\received{xxx, 2017}
\revised{xxx, 2017}
\accepted{xxx, 2017}
\submitjournal{ApJ}

\shorttitle{Low-frequency Radio Foregrounds Simulation}
\shortauthors{Li \etal}

%% You can add a light gray and diagonal water-mark to the first page
%% with this command:
\watermark{DRAFT}
%% where "text", e.g. DRAFT, is the text to appear.  If the text is
%% long you can control the water-mark size with:
%  \setwatermarkfontsize{dimension}
%% where dimension is any recognized LaTeX dimension, e.g. pt, in, etc.


%%%%%%%%%%%%%%%%%%%%%%%%%%%%%%%%%%%%%%%%%%%%%%%%%%%%%%%%%%%%%%%%%%%%%%%%%

\begin{document}

\title{
  Low-frequency Radio Foregrounds Simulation:
  Extended Radio Emission from Galaxy Clusters
}

%% The \author command is the same as before except it now takes an optional
%% argument which is the 16 digit ORCID. The syntax is:
%% \author[xxxx-xxxx-xxxx-xxxx]{Author Name}
%%
%% This will hyperlink the author name to the author's ORCID page. Note that
%% during compilation, LaTeX will do some limited checking of the format of
%% the ID to make sure it is valid.
%%
%% Use \affiliation for affiliation information. The old \affil is now aliased
%% to \affiliation. AASTeX v6.1 will automatically index these in the header.
%% When a duplicate is found its index will be the same as its previous entry.
%%
%% Note that \altaffilmark and \altaffiltext have been removed and thus
%% can not be used to document secondary affiliations. If they are used latex
%% will issue a specific error message and quit. Please use multiple
%% \affiliation calls for to document more than one affiliation.
%%
%% The new \altaffiliation can be used to indicate some secondary information
%% such as fellowships. This command produces a non-numeric footnote that is
%% set away from the numeric \affiliation footnotes.  NOTE that if an
%% \altaffiliation command is used it must come BEFORE the \affiliation call,
%% right after the \author command, in order to place the footnotes in
%% the proper location.
%%
%% Use \email to set provide email addresses. Each \email will appear on its
%% own line so you can put multiple email address in one \email call. A new
%% \correspondingauthor command is available in V6.1 to identify the
%% corresponding author of the manuscript. It is the author's responsibility
%% to make sure this name is also in the author list.
%%
%% While authors can be grouped inside the same \author and \affiliation
%% commands it is better to have a single author for each. This allows for
%% one to exploit all the new benefits and should make book-keeping easier.
%%
%% If done correctly the peer review system will be able to
%% automatically put the author and affiliation information from the manuscript
%% and save the corresponding author the trouble of entering it by hand.

\correspondingauthor{Haiguang Xu}
\email{hgxu@sjtu.edu.cn}

\author[0000-0002-7527-380X]{Weitian Li}
\email{liweitianux@sjtu.edu.cn}
\affil{Department of Astronomy, Shanghai Jiao Tong University,
  800 Dongchuan Road, Shanghai 200240, China}

\author{Haiguang Xu}
\affil{Department of Astronomy, Shanghai Jiao Tong University,
  800 Dongchuan Road, Shanghai 200240, China}


%%%%%%%%%%%%%%%%%%%%%%%%%%%%%%%%%%%%%%%%%%%%%%%%%%%%%%%%%%%%%%%%%%%%%%%%%

\begin{abstract}
\TODO{write the abstract!!!}
\end{abstract}

%% Authors should select subject keywords from the given list.  No more
%% than 6 keywords should be given, and they should be listed in
%% alphabetical order.
%%
%% See the online documentation for the full list of available subject
%% keywords and the rules for their use.
%% http://journals.aas.org/authors/keywords2013.html
\keywords{
  \TODO{update ($\leqslant 6$)!}
  dark ages, reionization, first stars ---
  galaxies: clusters: intracluster medium ---
  galaxies: halos ---
  methods: numerical ---
  radio continuum: galaxies
}

%%%%%%%%%%%%%%%%%%%%%%%%%%%%%%%%%%%%%%%%%%%%%%%%%%%%%%%%%%%%%%%%%%%%%%%%%

\section{Introduction}
\label{sec:intro}

\TODO{REWRITE, add more citations!!!}

\OK{This is OK :-)}

\NOTE{Take a note.}

The epoch of reionization (EoR) and cosmic dawn are one of the periods
that totally lack observational evidences, while a better understanding
of these important stages will greatly help us comprehend the whole
evolution of the Universe, especially the formation of large-scale structures,
and evolution of galaxy clusters.
The \SI{21}{\cm} line emission from the neutral hydrogen (\Hi) throughout
the Universe provides us with an excellent means to probe these ``dark''
stages \citep[see][for a recent review]{zaroubi2013rev}.
However, the current generation of low-frequency facilities,
e.g., LOFAR\footnote{LOFAR: LOw-Frequency ARray; \url{http://lofar.org/}}
\citep{vanHaarlem2013},
MWA\footnote{MWA: Murchison Widefield Array;
  \url{http://www.mwatelescope.org/}} \citep{tingay2013},
PAPER\footnote{PAPER: Precision Array for Probing the Epoch of Reionization;
  \url{http://eor.berkeley.edu/}} \citep{parsons2010},
are lacking the sensitivity to detect the very faint \Hi{} EoR signals.
The next-generation SKA\footnote{SKA: Square Kilometre Array;
  \url{https://skatelescope.org/}} \citep{mellema2013rev,koopmans2015rev}
with its unprecedented power, will enable
us not only to successfully detect them through power spectrum measurements,
but also to reveal the appearance of the reionization processes.

The greatest challenge to make the successful detection of the EoR signals
is the overwhelming foreground pollution, which is generally \numrange{4}{5}
orders of magnitude in strength with respect to the target EoR signals.
To accurately remove or separate such strong foregrounds, a realistic
foregrounds model is of great importance, and is very necessary to help
develop and evaluate foreground removal methods, since very small residuals
will damage the faint signal or bias its measurements \citep{procopio2017}.
In the framework of EoR detection, the galaxy clusters, as one of the
major foreground contamination components, lack enough attentions in the
literature,
but we believe that they are worth a much more serious treatment:
\begin{itemize}
  \item bright: $\sim\! \numrange{2}{3}$ orders of magnitude stronger
    than the EoR signals;
  \item extended: relatively large angular sizes, may extend to several tens
    of arcminutes, corrupting the EoR signals underneath;
  \item complicated morphology: very hard to model and subtract them,
    and may severely suffers from various point spread function (PSF) fringes
    due to interferometers' incomplete $(u,v)$-plane coverage, and all kinds
    of instrumental effects;
  \item varying spectral indexes/shapes across the extended emission;
  \item different types of extended emission, e.g., giant halo, mini-halo,
    and relics;
  \item highly-polarized radio relics may cause more troubles both in the
    image domain and in the power spectra.
\end{itemize}

In addition, galaxy clusters are also important objects of study in their
own right:
\begin{itemize}
  \item elements to form the large-scale structures, key to understand the
    structure formation and evolution of the Universe;
  \item merger-induced turbulence and shocks accelerate electrons to produce
    extended halos and relics, respectively, enable us to trace cluster
    mergers and formations with radio techniques;
  \item great laboratory to study the origin of cosmic rays, particle
    acceleration, magnetic growth and distribution, etc.
\end{itemize}

Extended radio emissions are only observed in a small fraction of massive
and luminous galaxy clusters, and they are more rarely found in smaller
clusters.
These extended radio emissions can be roughly classified into three types
\citep[see][for recent reviews]{feretti2012rev,brunetti2014rev}:
\begin{description}
  \item[Halos]
    giant and extend on $\sim \!\si{\Mpc}$ scales;
    They are only found at the center of galaxy clusters showing major
    merging features, and have regular morphology without significant
    polarization.

    The \emph{turbulence re-acceleration model} (also known as the
    \emph{primary electrons model}) is the most accepted model to explain
    the origin of radio halos.
    In this model, major mergers between clusters generate turbulence
    throughout the cluster ICM, which interact with the existing low-energy
    electrons to accelerate them to higher energies (e.g., $\gamma > 1000$),
    and then these relativistic electrons radiate synchrotron emissions
    within ICM's \si{\uG}-level magnetic field, to be responsible for the
    radio halos.
    On the other hand, there is the \emph{hadronic model}, which is also
    known as the \emph{secondary electrons model}.

    Currently we have only discovered $\lesssim\,$80 radio halos
    \citep{kale2016rev,duchesne2017}, and a much larger sample is a
    necessity to well constrain the halo models, in order to further reveal
    the physical processes.

  \item[Mini-halos]
    smaller than the above giant halos (usually several hundreds of kpc);
    currently only $\sim\,$20 of them have been found, generally located
    at the center of relaxed cool-core clusters \citep{bravi2016};
    their origin is still poorly known, but is different to that of giant
    halos, and is argued to be related with the turbulence in the cool-core
    region and AGN feedback \citep{gitti2015rev}.

  \item[Relics]
    typically are elongated and may stretch over $\sim \!\si{\Mpc}$ scales;
    located in cluster peripheral regions of both merging and relaxed
    clusters, suggesting that they may be related to minor or off-axis
    mergers, as well as to major mergers \citep{feretti2012rev}.
    The strongly polarized ($\sim\,$20--30\%) emission is one of the
    most important distinction compared to halos and mini-halos.

    Apart from the typical elongated relics, there are also the so-called
    roundish relics, showing a more regular and roundish morphology,
    and they are generally distributed closer to the cluster center.
    Their different properties with respect to the elongated relics
    suggest a different origin.

    Only in $\sim\,$70 clusters have been observed radio relics
    \citep[their Table 1]{nuza2017} for the moment, and more observational
    evidences are required to better understand their nature.
\end{description}

Due to their low surface brightness and steep spectra, these extended
radio emissions from galaxy clusters are difficult to find at middle-
or high-frequency ranges (e.g., \SI{1.4}{\GHz}).
However, they become much brighter and thus easier to be detected in
the low-frequency radio band.  The large field of view (FoV) of the
low-frequency interferometer also helps discover them more efficiently.
Owing to its excellent sensitivity, the SKA will possibly discover more
than several thousands of halos and relics, providing us with invaluable
data to understand and solve the puzzles with regard to galaxy clusters.
Furthermore, the faint radio sky can be very different, and we may have
unexpected findings as we go deeper \citep{padovani2016rev,herreraRuiz2017}.

In this work, we develop a new low-frequency radio foregrounds simulation
technique, with a specific focus on a physically motivated simulation method
for radio halos from galaxy clusters,
meanwhile we bear in mind that the SKA has the unparalleled sensitivity,
large FoV, and much finer angular resolution.
We implemented the proposed foregrounds simulation technique in the
\texttt{fg21sim} open-source software, which is written in Python and
is available at \url{https://github.com/liweitianux/fg21sim}.

\TODO{Continue with radio halos' possible impacts on EoR detection!!!}

We adopt a flat \lcdm{} cosmology throughout this work with
$H_0 = \SI{100}{\hubble} = \SI{71}{\km\per\second\per\Mpc}$,
$\Omega_m = 0.27$, $\Omega_{\Lambda} = 1 - \Omega_m = 0.73$,
$\Omega_b = 0.046$, and $\sigma_8 = 0.81$ \citep{komatsu2011}.


%%%%%%%%%%%%%%%%%%%%%%%%%%%%%%%%%%%%%%%%%%%%%%%%%%%%%%%%%%%%%%%%%%%%%%%%%

\section{Foregrounds Simulation}
\label{sec:fg-simu}

\TODO{improve this overview!!!}

The low-frequency radio foregrounds overwhelming stronger than the EoR
signals of our interest, making foreground removal an extreme challenge.
To develop such a targeted and ambitious foreground removal method,
an as realistic as possible simulation of the radio foregrounds is of
great importance, which can provide both instructions and evaluation data.

In this section, we first briefly introduce the simulation methods for
Galactic synchrotron (unpolarized) emission, Galactic free-free emission,
Galactic supernova remnants (SNRs) emission (\TODO{ignore this???}),
and extragalactic point sources emission (\TODO{ignore this???}).
Then we describe in details the physically motivated method we proposed
to simulate the extended radio halos emission from galaxy clusters.

Since we focus on simulating the radio halos from galaxy clusters and
evaluating their impacts on the EoR detection experiments, we leave out
the Galactic SNRs and extragalactic point sources in this work.


%%%%%%%%%%%%%%%%%%%%%%%%%%%%%%%%%%%%%%%%%%%%%%%%%%%%%%%%%%%%%%%%%%%%%%%%%
\subsection{Galactic Synchrotron Emission}
\label{sec:fg-gsync}

The diffuse Galactic synchrotron emission dominates the low-frequency
radio sky and accounts for $\sim\,$70\% of the foregrounds,
It has a power-law spectrum\footnote{We adopt the power-law
  spectrum of form: $S(\nu) \propto \nu^{-\alpha}$.}
with the spectral index $\alpha_{\R{syn}}$ varying as sky positions.

We simulate the Galactic synchrotron emission at desirable low frequency
$\nu$ based on the Haslam \SI{408}{\MHz} all-sky map \citep{haslam1982},
using the improved source-removed and destriped version reprocessed by
\citet{remazeilles2015}\footnote{The 2014 reprocessed Haslam \SI{408}{\MHz}
  map: \url{http://www.jb.man.ac.uk/research/cosmos/haslam_map/}}.
We then extrapolate from this template to lower frequencies by assuming
a power-law spectrum \citep{wang2010,bonaldi2015}:
\begin{equation}
  \label{eq:gsync-extrap}
  T_b^{\R{syn}}(\bm{r}, \nu) = T_b^{\R{Haslam}}(\bm{r})
    \left( \frac{\nu}{\SI{408}{\MHz}} \right)^{-\alpha_{\R{syn}}(\bm{r})},
\end{equation}
where $T_b^{\R{syn}}(\bm{r}, \nu)$ is the Galactic synchrotron intensity
expressed in brightness temperature at sky position $\bm{r}$ and
frequency $\nu$, and $\alpha_{\R{syn}}(\bm{r}) \sim \numrange{2.7}{3.2}$ is
the position-dependent synchrotron spectral index \citep{davies1996}.
We use the all-sky map of synchrotron spectral index provided by the
Planck Sky Model\footnote{Planck Sky Model:
  \url{http://www.apc.univ-paris7.fr/~delabrou/PSM/psm.html}}
\citep{delabrouille2013,alonso2015}.


%%%%%%%%%%%%%%%%%%%%%%%%%%%%%%%%%%%%%%%%%%%%%%%%%%%%%%%%%%%%%%%%%%%%%%%%%
\subsection{Galactic Free-free Emission}
\label{sec:fg-gfree}

\NOTE{Whether to incorporate some works from Lian Xiaoli??}

The diffuse Galactic free-free emission due to the bremsstrahlung
radiation from diffuse ionized interstellar medium (ISM),
e.g., \Hii{} regions, contributes only $\sim\,$1\% of the total
foregrounds in the low frequencies of interest.
It cannot be directly measured by current facilities since it is too
weak compared to the above synchrotron emission.
Nevertheless, the ionized ISM generates both radio free-free continuum
and \Halpha{} emission, and the relationship between these two kinds of
emissions can be determined under certain assumptions \citep{dickinson2003}.
Therefore we can derive the Galactic free-free emission based on the
all-sky \Halpha{} survey data.
However, the \Halpha{} emission suffers from dust absorption, which
may be very serious near the Galactic plane (e.g., $|b| < \SI{5}{\degree}$).
So dust correction should be applied to the \Halpha{} emission first.

We follow the approach proposed by \citet{dickinson2003} to obtain
the Galactic free-free emission maps.
The observed \Halpha{} maps can be corrected for dust absorption by:
\begin{equation}
  \label{eq:halpha-dust-corr}
  I_{\Halpha}^{\R{corr}}(\bm{r}) = I_{\Halpha}(\bm{r}) \times
    10^{D(\bm{r}) \times 0.0185 \times f_d},
\end{equation}
where $I_{\Halpha}(\bm{r})$ is the observed \Halpha{} intensity at sky
position $\bm{r}$ in units of \si{\rayleigh},
$D(\bm{r})$ is the measured dust column density in units of
\si{\mega\jansky\per\steradian},
and $f_d$ is the effective dust fraction along the line of sight actually
absorbing the \Halpha{} emission.

Then we can convert the absorption-corrected \Halpha{} intensity
$I_{\Halpha}^{\R{corr}}(\bm{r})$ to the free-free emission brightness
temperature $T_b^{\R{ff}}(\bm{r}, \nu)$
using the relationship \citep{dickinson2003}
\footnote{Their formula has a superfluous factor of $10^3$.}:
\begin{equation}
  \label{eq:halpha-ff}
  T_b^{\R{ff}}(\bm{r}, \nu) = 38.86 \, a(\nu, T_e) \nu^{-2.1}
    10^{(290 / T_e)} \, T_e^{0.667}
    \left[ \frac{I_{\Halpha}^{\R{corr}}(\bm{r})}{\si{\rayleigh}} \right]
    \quad [\si{\kelvin}],
\end{equation}
where $\nu$ is the target frequency in units of \si{\MHz},
$T_e$ is the electron temperature in units of \si{\kelvin},
and the factor $a(\nu, T_e)$ is given by:
\begin{equation}
  a(\nu, T_e) = 0.183 \,\nu^{0.1} T_e^{-0.15}
    \, [ 3.91 - \ln \nu + 1.5 \ln T_e ].
\end{equation}

In this work, we use the all-sky \Halpha{} map from \citet{finkbeiner2003},
the all-sky dust map from \citet{schlegel1998}, and
adopt effective dust fraction $f_d = 0.33$ and
electron temperature $T_e = \SI{7000}{\kelvin}$
as suggested \citep{dickinson2003}.


%%%%%%%%%%%%%%%%%%%%%%%%%%%%%%%%%%%%%%%%%%%%%%%%%%%%%%%%%%%%%%%%%%%%%%%%%
\subsection{Galactic Supernova Remnants}
\label{sec:fg-gsnr}

\TODO{Write!!!  Or ignore this minor component???}

\NOTE{Cannot find references describing the surface brightness (profile)
of SNRs!}

Update to use a empirical model about the total number of Galactic SNRs
and their spatial distributions.  Meanwhile derive necessary scaling
relations from the current observed catalog maintained by Green.


%%%%%%%%%%%%%%%%%%%%%%%%%%%%%%%%%%%%%%%%%%%%%%%%%%%%%%%%%%%%%%%%%%%%%%%%%
\subsection{Extragalactic Point Sources}
\label{sec:pointsources}

\TODO{To follow Wang2010???  Or ignore this component???}


%%%%%%%%%%%%%%%%%%%%%%%%%%%%%%%%%%%%%%%%%%%%%%%%%%%%%%%%%%%%%%%%%%%%%%%%%
\subsection{Galaxy Clusters}
\label{sec:fg-clusters}

\TODO{Rewrite this overview!!!}

In the framework of low-frequency radio foregrounds simulation for the
EoR signal detection, the most realistic consideration to the galaxy
clusters component to date is our previous work in \citeyear{wang2010}
\citep{wang2010}.
In this work, we continue to improve our simulation method in several
ways:
(1) Predict radio halos by tracing the most recent major merger of clusters;
(2) Determine the halo brightness and spectrum by solving the time
evolution of relativistic electrons within the cluster ICM.
We present our improved simulation method to the galaxy clusters in this
section.


%%%%%%%%%%%%%%%%%%%%%%%%%%%%%%%%%%%%%%%%%%%%%%%%%%%%%%%%%%%%%%%%%%%%%%%%%
\subsubsection{Distributions}
\label{sec:distributions}

For a patch of sky, we first want to determine the mass and redshift
distributions of the galaxy clusters, which can be sampled from the
Press-Schechter (PS) mass function \citep{press1974}:
\begin{equation}
  \label{eq:ps-mass-func}
  n(M, z) \,\D{M} = \sqrt{\frac{2}{\pi}} \frac{\langle{\rho}\rangle}{M}
  \frac{\delta_c(z)}{\sigma^2(M)} \left| \diff{\sigma(M)}{M} \right|
  \exp\!\left[ -\frac{\delta_c^2(z)}{2\sigma^2(M)} \right] \D{M},
\end{equation}
where
$M$ is the cluster mass,
$\langle {\rho} \rangle$ is the current mean density of the Universe,
$\delta_c(z)$ is the critical linear overdensity for a region to collapse
at redshift $z$ [see Equation~(\ref{eq:delta-crit})],
and $\sigma(M)$ is the current rms density fluctuations within a sphere
of mean mass $M$.

In cold dark matter (CDM) models and over the range of scales covered
by clusters, it is generally sufficient to consider a power-law spectrum
for the density perturbations \citep{sarazin2002,randall2002}:
\begin{equation}
  \label{eq:sigma-mass}
  \sigma(M) = \sigma_8 \left( \frac{M}{M_8} \right)^{-\alpha},
\end{equation}
where $\sigma_8$ is the present-day rms density fluctuations on a
scale of \SI{8}{\per\hubble\Mpc},
$M_8 = (4\pi/3)(8 \,\si{\per\hubble\Mpc})^3 \langle{\rho}\rangle$
is the mass contained in a sphere of radius \SI{8}{\per\hubble\Mpc}.
The exponent $\alpha$ is given by $\alpha = (n+3)/6$, where
the power spectrum of fluctuations varies with wavenumber $k$
as $k^n$, and we adopt $n = -7/5$ for this work \citep{randall2002}.

We place a lower mass limit of $M_{\R{min}} = \SI{2e14}{\solarmass}$
for clusters,
determine the expected total number of clusters within our simulation
sky patch, and then derive the mass and redshift for each cluster by
sampling from the PS mass function using Monte Carlo method
\citep{wang2010}.
The clusters are randomly distributed within the sky patch.


%%%%%%%%%%%%%%%%%%%%%%%%%%%%%%%%%%%%%%%%%%%%%%%%%%%%%%%%%%%%%%%%%%%%%%%%%
\subsubsection{Merging History}
\label{sec:merging-history}

Galaxy clusters are formed hierarchically by merging from smaller (i.e.,
lower mass) clusters, groups, and galaxies.
And the PS formalism provides a way to infer the formation history
of a cluster from its current status.
We follow the extended PS (EPS) method outlined by \citet{lacey1993} to
generate the merger trees.

The conditional probability that a ``parent'' cluster of mass $M_2$ at
a time $t_2$ had a ``progenitor'' of mass in the range
$M_1 \to M_1 + \D{M_1}$ at some earlier time $t_1$,
with $M_2 > M_1$ and $t_2 > t_1$, is given by
\citep{lacey1993,randall2002}:
\begin{equation}
  \label{eq:eps-condprob}
  P(M_1, t_1 | M_2, t_2) \,\D{M_1} = \frac{1}{\sqrt{2\pi}} \frac{M_2}{M_1}
  \frac{\delta_{c1} - \delta_{c2}}{(\sigma_1^2 - \sigma_2^2)^{3/2}}
  \left| \diff{\sigma_1^2}{M_1} \right|
  \exp \!\left[ -\frac{(\delta_{c1} - \delta_{c2})^2}
    {2(\sigma_1^2 - \sigma_2^2)} \right] \D{M_1},
\end{equation}
where
$\delta_{c1} \equiv \delta_c(t_1)$ and $\sigma_1 \equiv \sigma(M_1)$,
with similar definitions for $\delta_{c2}$ and $\sigma_2$.
By further defining $S \equiv \sigma^2(M)$ and $\omega \equiv \delta_c(t)$,
and let $K(\Delta S, \Delta \omega) \,\D{\Delta S}$ be the probability
that a cluster had a progenitor with a mass corresponding to a change
in $S$ of $\Delta S = \sigma_1^2 - \sigma_2^2$ in the range
$\Delta S \to \Delta S + \D{\Delta S}$ at an earlier time
corresponding to $\Delta \omega = \delta_{c1} - \delta_{c2}$,
then Equation~(\ref{eq:eps-condprob}) is simplified into:
\begin{equation}
  \label{eq:eps-condprob-simp}
  K(\Delta S, \Delta \omega) \,\D{\Delta S} = \frac{1}{\sqrt{2\pi}}
  \frac{\Delta\omega}{(\Delta S)^{3/2}}
  \exp \!\left[ -\frac{(\Delta\omega)^2}{2 \Delta S} \right] \D{\Delta S}.
\end{equation}

We employ the standard Monte Carlo technique to build the merger trees,
with each of them starts with a cluster with an initial mass $M_0$
and redshift $z_0$ simulated in the last section.
We trace each cluster back in time, using a step size $\Delta\omega$
adaptively determined by \citep{randall2002}:
\begin{equation}
  \label{sec:dw-step}
  (\Delta\omega)_{\R{step}} \lesssim \frac{1}{2} \left[
    S \left| \diff{\ln \sigma^2}{\ln M} \right|
    \left( \frac{\Delta M_c}{M} \right) \right]^{1/2},
\end{equation}
where $M$ is the cluster mass at the current time step,
$\Delta M_c$ is the minimum mass change indicating the smallest
merging sub-cluster can be resolved in the merger trees;
and we choose $\Delta M_c = \SI{e12}{\solarmass}$.
In order to obey the probability distribution of mergers of different
$\Delta S$ (i.e., $\Delta M$), we sample the cumulative probability
distribution of the sub-cluster masses:
\begin{equation}
  \label{sec:cdf-sub-masses}
  P(<\!\Delta S, \Delta\omega) =
  \int_0^{\Delta S} K(\Delta S', \Delta\omega) \,\D{\Delta S'} =
  \erfc \!\left( \frac{\Delta \omega}{\sqrt{2 \Delta S}} \right),
\end{equation}
where $\erfc(x) = 1 - \erf(x)$ is the complementary error
function whose inversion is also analytical.
With a random $\Delta S$ drawn from this probability distribution,
the value of $S_1$ of the progenitor is then $S_1 = S_2 + \Delta S$,
and the mass of one of the progenitors is determined by
$\sigma^2(M_1) = S_1$, while for the other is $\Delta M = M_2 - M_1$.

Let $M_{\R{main}} \equiv \max(M_1, \Delta M)$ and
$M_{\R{sub}} \equiv \min(M_1, \Delta M)$.
If $M_{\R{sub}} \leq \Delta M_c$, we regard this change in mass
to be due to accretion, and we continue build the merger tree from
$M_{\R{main}}$ with the sub-cluster been ignored.
On the other hand, it is identified as a merger event if
$M_{\R{sub}} > \Delta M_c$, then we trace both the main cluster
and the sub-cluster separately.

Considering that radio halos are only observed in
a fraction of massive merging clusters, suggesting they are closely
related to the recent major mergers, and have a lifetime
$\tau_{\R{halo}}$ shorter than the timescale
$\tau_m$ ($\sim\! \SIrange[range-units=single]{2}{3}{\Gyr}$)
for the duration of merger-induced disturbance \citep{cassano2016}.
Therefore we only trace the merging history of the main cluster
to a maximum backward time of \SI{3}{\Gyr}.
Then we identify the most recent major merger for each cluster,
which is defined as $M_{\R{main}}/M_{\R{sub}} < 3$
\citep[e.g.,][]{wetzel2009},
and extract the $(M_{\R{main}}, M_{\R{sub}}, z_{\R{merger}})$
data for use in the following radio halos simulation.
Meanwhile, those clusters without a recent major merger found
are thought to not host radio halos, and thus been ignored.

We also note that this technique allows only binary mergers,
but it is sufficient for our use here.


%%%%%%%%%%%%%%%%%%%%%%%%%%%%%%%%%%%%%%%%%%%%%%%%%%%%%%%%%%%%%%%%%%%%%%%%%
\subsubsection{Radio Halos}
\label{sec:halo}

Giant radio halos in galaxy clusters are associated with the on-going or
most recent major mergers, and they are widely believed to be radiated from
highly relativistic electrons re-accelerated by the merger-induced
large-scale turbulent fluctuations.
In addition, extended radio emissions of smaller scale (a few hundred kpc),
called mini-halos, have been found in some relaxed cool-core clusters.
We do not account for the radio mini-halos for the moment, and they
seems to have a different origin \citep{feretti2012rev}.

The approach we adopted here to simulate the giant radio halos
was largely inspired by \citet{cassano2005}, but we
made various modifications as well as simplifications for our needs.

For each cluster associated with a recent major merger, we assume that
the two sub-clusters undergo only central collisions.
Starting from an initial distance $d_0$ with zero velocity,
the relative impact velocity of two merging clusters with mass
$M_{\R{main}}$ and $M_{\R{sub}}$ upon collision (at a distance
$R_{\R{main}}$ between the centers) is given by
\citep{cassano2005,sarazin2002}:
\begin{equation}
  \label{eq:v-imp}
  v_{\R{imp}} \simeq \left[
    \frac{2G (M_{\R{main}} + M_{\R{sub}})}{R_{\R{main}}}
    \left( 1 - \frac{1}{\eta_v} \right)\right]^{1/2},
\end{equation}
where $d_0 = \eta_v R_{\R{main}}$,
$\eta_v \simeq 4 (1 + M_{\R{sub}}/M_{\R{main}})^{1/3}$,
and $R_{\R{main}}$ is the virial radius of the main cluster
[see Equation~(\ref{eq:radius-virial})].

Cluster mergers generate turbulence throughout the cluster over
$\sim \!\si{\Mpc}$ scales.
Energy can be transferred from the ICM into the non-thermal
component through the interactions between electrons and the
magneto-hydrodynamics (MHD) turbulence, which accelerate the electrons
to be extremely relativistic and radiate synchrotron emissions.
This scenario of particle re-acceleration by turbulence is the most
accepted mechanism responsible for the radio halos \citep{feretti2012rev}.

The turbulence re-acceleration likes a second-order Fermi process, i.e.,
related with a random process, may be effective for only a few
\SI{e8}{\year} during a merger, and thus not quite efficient
\citep{feretti2012rev,enblin2011}.
We approximate the effective acceleration timescale with the merger
crossing time \citep{cassano2005}:
\begin{equation}
  \label{eq:tau-cross}
  \tau_{\R{cross}} \simeq R_{\R{main}} / v_{\R{imp}},
\end{equation}
which is typically $\lesssim \SI{1}{\Gyr}$.
In addition, the acceleration efficiency of electrons due to turbulent waves,
which described by the acceleration timescale, can be then estimated as
\citep{cassano2005,brunetti2011}:
\begin{equation}
  \label{eq:tau-acc}
  \tau_{\R{acc}}(t) \simeq \left\{
    \begin{array}{ll}
      0.1 / \beta_{\R{turb}}, & t \leq t_{\R{merger}} + \tau_{\R{cross}} \\
      +\infty, & t > t_{\R{merger}} + \tau_{\R{cross}}
    \end{array}
  \right. \quad [\si{\Gyr}],
\end{equation}
%% TODO:
%% add a footnote about tau = infty, which is actually ~100 Gyr implemented
%% in the software, to avoid numerical instabilities when solving FP equation.
where $\beta_{\R{turb}} \sim 1$ is a free parameter describing the
intensity of the turbulence acceleration
(we assume $\beta_{\R{turb}} = 1$ in this work),
and $t_{\R{merger}}$ is the cosmic time when the merger begins
(i.e., at redshift $z_{\R{merger}}$).

The time evolution of relativistic electrons spectrum with isotropic energy
distribution can be derived by solving the Fokker-Planck equation
\citep{eilek1991,schlickeiser2002}:
\begin{equation}
  \label{eq:fokkerplanck}
  \pdiff{n(\gamma,t)}{t} = \pdiff{}{\gamma} \left[ n(\gamma,t) \left(
      \left| \diff{\gamma}{t} \right|_{\R{rad}} +
      \left| \diff{\gamma}{t} \right|_{\R{ion}} -
      \frac{2}{\gamma} D_{\R{\gamma\gamma}}(\gamma, t) \right) \right] +
  \pdiff{}{\gamma} \left[ D_{\R{\gamma\gamma}} \pdiff{n(\gamma,t)}{p} \right] +
  Q_e(\gamma,t),
\end{equation}
where $\gamma = p / m_e c$ is the Lorentz factor of the electrons,
$n(\gamma, t)$ is the isotropic number density of relativistic
electrons, $Q_e(\gamma, t)$ describes the electron injection,
$|\R{d}\gamma / \R{d}t|_{(\cdot)}$ terms account for the
energy losses due to radiation and ionization,
and $D_{\R{\gamma\gamma}}(\gamma, t)$ is the diffusion coefficient
describing the interactions between electrons and turbulent waves,
which is given by \citep{brunetti2011}:
\begin{equation}
  \label{eq:diffusion-coef}
  D_{\R{\gamma\gamma}}(\gamma, t) =
    \frac{\gamma^2}{4 \tau_{\R{acc}}(t)}.
\end{equation}
In some energy ranges (e.g., $\gamma \sim 1000$), the diffusion may
contribute more energies than those lost, therefore can accelerate electrons
to higher energies, making them synchrotron bright.

The electrons that will be re-accelerated \emph{in situ}
by the turbulence permeate the cluster ICM, which are thought to be
injected by AGN activities (quasars, radio galaxies, etc.), or by
star formation in normal galaxies (supernovae, galactic winds, etc.)
during the cluster life \citep[see][for a review]{blasi2007rev}.
The electrons can be reasonably assumed to be constantly injected
into the cluster ICM with a power-law spectrum of index $s$
\citep{cassano2005,sarazin1999}:
\begin{equation}
  \label{eq:electron-inj}
  Q_e(\gamma, t) = Q_e(\gamma) = K_e \gamma^{-s},
\end{equation}
where $K_e$ is the constant injection rate, which can be parametrized
by assuming that the total energy injected in the electrons during
the cluster life is a fraction, $\eta_e$, of the total thermal energy
of the cluster \citep{cassano2005}:
\begin{equation}
  \label{eq:injrate-param}
  \left( \frac{4\pi}{3} R_{\R{halo}}^3 \right) \epsilon_e \tau_{\R{cluster}}
  \int_{\gamma_{\R{min}}}^{\gamma_{\R{max}}}
  Q_e(\gamma') \gamma' \,\D{\gamma'} \simeq
  \eta_e \epsilon_{\R{th}} \left( \frac{4\pi}{3} R_{\R{vir}}^3 \right),
\end{equation}
where $R_{\R{halo}}$ and $R_{\R{vir}}$ are the radius of the radio halo
will be generated and the virial radius of the cluster, respectively,
$\tau_{\R{cluster}} \simeq t_0$ is the cluster age at redshift $z_0$ as
simulated in Section~\ref{sec:distributions},
$\epsilon_e = m_e c^2$ is the energy of one electron at still,
$\epsilon_{\R{th}}$ is the thermal energy density of the cluster ICM,
and we adopt $\gamma_{\R{max}} = 10^5 \gg \gamma_{\R{\min}} = 50$
for our simulation configurations.
Therefore we derive the electron injection rate as:
\begin{equation}
  \label{eq:injrate}
  K_e \simeq \frac{(s-2)\eta_e\epsilon_{\R{th}}}{\epsilon_e\tau_{\R{cluster}}}
  \left( \frac{R_{\R{vir}}}{R_{\R{halo}}} \right)^3 \gamma_{\R{min}}^{s-2}.
\end{equation}

The ICM thermal energy density is calculated as:
\begin{equation}
  \label{eq:eth}
  \epsilon_{\R{th}} = \frac{3}{2} n_{\R{th}} k_B T,
\end{equation}
where $n_{\R{th}}$ is the thermal number density [see Equation~(\ref{eq:nth})],
and $k_B T$ is the ICM mean temperature, which is estimated
from the cluster mass using this $M \!-\! T$ scaling relation
\citep{arnaud2005}:
\begin{equation}
  \label{eq:mt-relation}
  \left[ \frac{M_{200}}{\SI{e14}{\solarmass}} \right] E(z) = (5.34 \pm 0.22)
  \left[ \frac{k_B T}{\SI{5}{\keV}} \right]^{(1.72 \pm 0.10)},
\end{equation}
where $E(z)$ is the redshift evolution factor [see Equation~(\ref{eq:ez-func})].
Note that we approximate the cluster virial mass with $M_{200}$ here.

Observations show that radio halos have a radial size
$R_{\R{halo}} \sim\! \numrange{3}{6}$ times smaller than the virial radius
$R_{\R{vir}}$, therefore we estimate the halo radius by
\citep{cassano2007,zandanel2014}:
\begin{equation}
  \label{eq:rhalo-rvir}
  R_{\R{halo}} = \frac{1}{4} R_{\R{vir}}.
\end{equation}
The weak \si{\uG}-level magnetic fields are also observed in clusters.
We assume an uniform magnetic field throughout the ICM,
but allow its strength to scale with the cluster mass as \citep{cassano2012}:
\begin{equation}
  \label{eq:magfield-mass}
  B = B_{\langle{M}\rangle} \left( \frac{M}{\langle{M}\rangle} \right)^b,
\end{equation}
where $\langle {M} \rangle = \SI{1.6e15}{\solarmass}$, and we take
$B_{\langle {M} \rangle} = \SI{1.9}{\uG}$ and $b = 1.5$ in this work.

On the other hand,
relativistic electrons lose energy via synchrotron radiation
and inverse Compton scattering off the CMB photons \citep{sarazin1999}:
\begin{equation}
  \label{eq:eloss-rad}
  \left( \diff{\gamma}{t} \right)_{\R{rad}} =
  \num{-1.37e-20} \,\gamma^2
  \left[ \left( \frac{B}{\SI{3.25}{\uG}} \right)^2 +
  (1+z)^4 \right] \quad [\si{\per\second}],
\end{equation}
where $B$ is the magnetic field strength in the ICM.
Meanwhile, electrons in the ICM also lose energy through
ionization and Coulomb collisions \citep{sarazin1999}:
\begin{equation}
  \label{eq:eloss-ion}
  \left( \diff{\gamma}{t} \right)_{\R{ion}} =
  \num{-1.20e-12} \,n_{\R{th}} \left[ 1 +
    \frac{\ln(\gamma/n_{\R{th}})}{75} \right] \quad [\si{\per\second}],
\end{equation}
with $n_{\R{th}}$ in units of [\si{\per\cm\cubed}].

Having all the coefficients of the Fokker-Planck
equation~(\ref{eq:fokkerplanck}) been determined, we can now derive
the time evolution of the electron spectrum.
An efficient numerical method proposed by \citet{chang1970} is
utilized to solve the Fokker-Planck equation, under the no-flux boundary
condition \citep{park1996}.
However, to avoid unphysical pile-up of electrons at the lower boundary,
we define a ``buffer'' region below $\gamma_{\R{buf}}$, within which
the spectrum are replaced by extrapolating as a power law, based on
the spectrum above $\gamma_{\R{buf}}$ \citep{donnert2014}.
We adopt a logarithmic grid for $\gamma \in [10, 10^5]$ with 200 cells,
and let the lower buffer region span 5 cells.

We solve the Fokker-Planck equation from the time $t_{\R{merger}}$
(i.e., $z_{\R{merger}}$) when the identified major merger begins,
until to the time $t_0$ (i.e., $z_0$) which corresponds to the current
time of our simulated ``observations.''
The low-energy electrons (e.g., $\gamma \sim 300$) have very long
lifetimes, they may be accumulated and retained as the cluster
grows \citep{sarazin1999}.
We therefore estimate the initial electron spectrum at $t_{\R{merger}}$ as:
\begin{equation}
  \label{eq:ne-init}
  n(\gamma, t=t_{\R{merger}}) = t_{\R{merger}} K_e \gamma^{-s}.
\end{equation}

Once the current electron spectrum $n(\gamma, t_0)$ been derived,
the synchrotron emissivity in units of
[\si{\erg\per\second\per\cm\cubed\per\hertz}],
is determined by \citep{rybicki1979}:
\begin{equation}
  \label{sec:jnu-sync}
  J_{\R{syn}}(\nu, t_0) = \frac{\sqrt{3}e^3 B}{m c^2}
  \int_{\gamma_{\R{min}}}^{\gamma_{\R{max}}} \int_0^{\pi/2}
  n(\gamma, t_0) F\!\left( \frac{\nu}{\nu_c} \right) \sin^2 \!\theta
  \,\D{\theta} \,\D{\gamma},
\end{equation}
where $F(\nu/\nu_c)$ is the synchrotron kernel:
\begin{equation}
  \label{eq:sync-kernel}
  F\!\left( \frac{\nu}{\nu_c} \right) = \frac{\nu}{\nu_c}
  \int_{\nu/\nu_c}^{\infty} K_{5/3}(y) \,\D{y},
\end{equation}
with $K_{5/3}(x)$ the modified Bessel function of \sfrac{5}{3} order,
$\nu_c = (3/2) \gamma^2 \nu_L \sin\theta$ the critical frequency,
and $\nu_L = e B / (2\pi m_e c)$ the Larmor frequency of electrons.
Observations suggest that radio halos have average synchrotron emissivity
$\langle {J_{\R{syn}}} \rangle \sim \SI{e-42}{\erg\per\second\per\cm\cubed\per\hertz}$
at 1.4 GHz \citep{murgia2009,feretti2012rev}.

The synchrotron emissivity is the same throughout the radio halo,
according to our previous assumptions, we can then calculate the radio
power at a specific frequency (i.e., spectral luminosity) as:
\begin{equation}
  \label{eq:halo-power}
  P_{\nu} \equiv P(\nu) = \frac{4\pi}{3} R_{\R{halo}}^3 J_{\R{syn}}(\nu),
\end{equation}
and the halo flux density $S_{\nu}$ is given by:
\begin{equation}
  \label{eq:halo-flux-density}
  S_{\nu} = \frac{P_{\nu}}{4\pi D_L^2},
\end{equation}
where $D_L$ is the luminosity distance to the halo at redshift $z_0$.
We adopt the exponential profile for the azimuthally averaged brightness
distribution of radio halos, which is expressed as \citep{murgia2009}:
\begin{equation}
  \label{eq:halo-exprofile}
  I_{\nu}(\theta) = I_{\nu,0} \exp(-\theta/\theta_e),
\end{equation}
where $\theta = R / D_A$ is the angular radius from the halo center,
with $D_A$ the angular diameter distance to the halo,
$\theta_e \simeq \theta_{\R{halo}}/3$ is the $e$-folding radius,
and $I_{\nu,0}$ is the central brightness, which is constrained by the
above flux density $S_{\nu}$ and can be derived to be:
\begin{equation}
  \label{eq:central-brightness}
  I_{\nu,0} = \frac{S_{\nu}}{2\pi \,\theta_e^2}.
\end{equation}

Finally, we simulate the appearance of each radio halo as an ellipse by
randomly assigning an ellipticity $f_e$ and a rotation angle $\phi$,
and then superimpose all the halo images to the sky patch to obtain the
simulated images of radio halos at each frequency of interest.


%%%%%%%%%%%%%%%%%%%%%%%%%%%%%%%%%%%%%%%%%%%%%%%%%%%%%%%%%%%%%%%%%%%%%%%%%
\subsection{The \texttt{fg21sim} Simulation Software}
\label{sec:fg21sim}

We have implemented the above foreground simulation methods in the
\texttt{fg21sim} open-source software available at
\url{https://github.com/liweitianux/fg21sim}.
The software is written in the Python programming language from scratch.
It is well designed and implemented, and we keep in mind that it should
be easy to use and to extend in the future.
The overall simulation speed is satisfactory,
and the Numba\footnote{Numba: A compiler for Python array and numerical
  functions; \url{http://numba.pydata.org/}} compiler
is used to optimize the speed of some codes when necessary.

\TODO{more introduction to our software, features (and limitations??)}


%%%%%%%%%%%%%%%%%%%%%%%%%%%%%%%%%%%%%%%%%%%%%%%%%%%%%%%%%%%%%%%%%%%%%%%%%
\section{EoR Signal Simulation}
\label{sec:eor-simulation}

\TODO{Write!!!}

We simulate the cosmological EoR signal using \texttt{SimFast21}\footnote{
  SimFast21: \url{https://github.com/mariogrs/Simfast21}}
\citep{santos2010,hassan2016}.


%%%%%%%%%%%%%%%%%%%%%%%%%%%%%%%%%%%%%%%%%%%%%%%%%%%%%%%%%%%%%%%%%%%%%%%%%
\section{Observation Simulation}
\label{sec:obs-simulation}

\TODO{Write!!!}

\begin{enumerate}
\item Generate the SKA1-low configuration\footnote{
    SKA-TEL-SKO-0000422: SKA1-low Configuration Coordinates: \url{https://astronomers.skatelescope.org/wp-content/uploads/2016/09/SKA-TEL-SKO-0000422_02_SKA1_LowConfigurationCoordinates-1.pdf}};

  \item Simulate the visibility data for the sky image/model at each
    frequency channel using \texttt{OSKAR}\footnote{
      OSKAR: A GPU accelerated simulator for SKA;
      \url{https://github.com/OxfordSKA/OSKAR}} \citep{mort2010};

  \item Create image cube with CLEAN algorithm using
    \texttt{WSClean}\footnote{WSClean: A fast widefield interferometric
      imager; \url{https://sourceforge.net/p/wsclean/}} \citep{offringa2014}.
\end{enumerate}


%%%%%%%%%%%%%%%%%%%%%%%%%%%%%%%%%%%%%%%%%%%%%%%%%%%%%%%%%%%%%%%%%%%%%%%%%
\section{Results}
\label{sec:results}

\TODO{Write!!!}

\TODO{Distribute the results (e.g., simulated images) into previous sections,
  and then merge this section (e.g., scaling relations and comparisons)
  with the next ``discussions'' section???}

\NOTE{Collect all the parameters (e.g., fg21sim, SimFast21, OSKAR, WSClean)
  as well as the input data/templates into to the Appendix!
  Also create a page in the software repository to hold further information.}

On the foregrounds simulation part, compare our simulated radio halos with
current observations as well as previous results,
e.g., halo luminosity function, (and ???).

Then we evaluate the impacts of radio halos on EoR detection in both the
image domain and the (2D) power spectrum domain:
\begin{description}
  \item[Image]
    Take the above simulated image cube, subtract the spectral-smooth
    foregrounds fitting a smooth polynomial along the frequency axis,
    evaluate the loss of EoR signals (use image-domain indexes???
    can also use the total power spectrum???)

    May consider too cases: (1) only radio halos is superimposed onto the
    EoR signals; (2) radio halos as well as Galactic foregrounds are
    added to the EoR signals.

  \item[Power spectrum]
    Take the above simulated image cube, calculate the 2D
    (i.e., cylindrically averaged) power spectrum, evaluate the foregrounds
    contamination within the (theoretically predicted) EoR window,
    compare to the case that only EoR signals exist in the EoR windows.
\end{description}


%%%%%%%%%%%%%%%%%%%%%%%%%%%%%%%%%%%%%%%%%%%%%%%%%%%%%%%%%%%%%%%%%%%%%%%%%
\section{Discussions}
\label{sec:discussions}

\TODO{Write!!!}

\NOTE{Current limitations???}

%% TODO: future works/improvements:
One of the main purposes of our foregrounds simulation is to support
the development and evaluation of foregrounds removal or separation
methods in our forthcoming works.
We will combine our foregrounds simulation with interferometric
observation simulation, to generate the ``observed'' visibility data
as well as images (e.g., by using the \textsc{clean} algorithm).
In this way, we incorporate the realistic instrumental effects into
our simulation pipeline, making our developed methods more versatile
and prepare for future real observations.
(\NOTE{Move this paragraph to above simulation section?})

In addition to the foregrounds removal aiming at the EoR signal
detection, the methods to effectively identify and extract the faint
extended radio emissions (e.g., halos, relics) from observation images
is of the same importance.
With future more sensitive observations combined with a better detection
method, we are able to build a big sample of galaxy clusters, which
may provide us with plenty of opportunities to study them in the radio
regime.

Moreover, there is still much room to improve the current foregrounds
simulation, e.g., taking radio relics into account, more realistic
source morphologies by utilizing the deep learning techniques,
full polarization support.
We will continue to improve our foregrounds simulation works and make
them available in our public software \texttt{fg21sim} in the future.


%%%%%%%%%%%%%%%%%%%%%%%%%%%%%%%%%%%%%%%%%%%%%%%%%%%%%%%%%%%%%%%%%%%%%%%%%
\section{Conclusions}
\label{sec:conclusions}

\TODO{Write!!!}


%%%%%%%%%%%%%%%%%%%%%%%%%%%%%%%%%%%%%%%%%%%%%%%%%%%%%%%%%%%%%%%%%%%%%%%%%
%% If you wish to include an acknowledgments section in your paper,
%% separate it off from the body of the text using the \acknowledgments
%% command.
\acknowledgments

\TODO{Write acknowledgments ...}

We acknowledge ...


%% To help institutions obtain information on the effectiveness of their
%% telescopes the AAS Journals has created a group of keywords for telescope
%% facilities:
%% http://journals.aas.org/authors/aastex/facility.html
%
%% Following the acknowledgments section, use the following syntax and the
%% \facility{} or \facilities{} macros to list the keywords of facilities used
%% in the research for the paper.  Each keyword is check against the master
%% list during copy editing.  Individual instruments can be provided in
%% parentheses, after the keyword, but they are not verified.

\vspace{5mm}

% \facilities{???}

%% Similar to \facility{}, there is the optional \software command to allow
%% authors a place to specify which programs were used during the creation of
%% the manusscript. Authors should list each code and include either a
%% citation or url to the code inside ()s when available.

\software{
  Astropy \citep{astropy2013},
  NumPy (\url{http://www.numpy.org/}),
  Vim (\url{http://www.vim.org/}),
  Emacs (\url{https://www.gnu.org/s/emacs/})
}


%%%%%%%%%%%%%%%%%%%%%%%%%%%%%%%%%%%%%%%%%%%%%%%%%%%%%%%%%%%%%%%%%%%%%%%%%
%%
%% Appendix material should be preceded with a single \appendix command.
%% There should be a \section command for each appendix. Mark appendix
%% subsections with the same markup you use in the main body of the paper.
%%
%% Each Appendix (indicated with \section) will be lettered A, B, C, etc.
%% The equation counter will reset when it encounters the \appendix
%% command and will number appendix equations (A1), (A2), etc. The
%% Figure and Table counter will not reset.
%%
\appendix

\section{Press-Schechter Formalism}
\label{sec:psf}

\TODO{Is this appendix necessary???}

The Press-Schechter (PS) formalism is one of the standard method to predict
the mass function of galaxy clusters and its evolution as the cosmology.
It is originally developed by \citet{press1974}, and later extended to combine
with the cold dark matter (CDM) model \citep[e.g.,][]{bond1991,lacey1993}.


%%%%%%%%%%%%%%%%%%%%%%%%%%%%%%%%%%%%%%%%%%%%%%%%%%%%%%%%%%%%%%%%%%%%%%%%%
\section{Useful Formulas}
\label{sec:formulas}

In this section we collect some useful formulas we made use of in our
simulations.

The Hubble parameter at redshift $z$:
\begin{equation}
  \label{eq:hubble-z}
  H(z) = H_0 E(z),
\end{equation}
where $H_0$ is the present-day Hubble value,
and $E(z)$ is the redshift evolution factor given by:
\begin{equation}
  \label{eq:ez-func}
  E(z) = \sqrt{\Omega_m(1+z)^3 + \Omega_{\Lambda}}.
\end{equation}
(Note that we adopt a flat \lcdm{} cosmology so that
$\Omega_k = 1 - \Omega_m - \Omega_{\Lambda} = 0$.)

The critical density of the Universe at redshift $z$ is:
\begin{equation}
  \label{eq:rho-crit}
  \rho_{\R{crit}}(z) = \frac{3 H^2(z)}{8 \pi G},
\end{equation}
where $G$ is the gravitational constant.

On the other hand, the critical linear overdensity as a function of
redshift $z$ is \citep{kitayama1996,randall2002}:
\begin{equation}
  \label{eq:delta-crit}
  \delta_c(z) = \frac{D(z=0)}{D(z)}
  \left[ \frac{3 (12\pi)^{2/3}}{20} \right]
  \left[1 + 0.0123 \log_{10} \Omega_f(z) \right],
\end{equation}
where $\Omega_f(z)$ is the mass density ratio at redshift $z$:
\begin{equation}
  \label{eq:omega-fz}
  \Omega_f(z) = \frac{\Omega_m(1+z)^3}{\Omega_m(1+z)^3 + \Omega_{\Lambda}},
\end{equation}
and $D(z)$ is the growth factor given by
\citep[Equation~(13.6)]{peebles1980}:
\begin{equation}
  \label{eq:growth-factor}
  D(x) = \frac{(x^3 + 2)^{1/2}}{x^{3/2}}
    \int_0^x y^{3/2} (y^3 + 2)^{-3/2} \,\D{y},
\end{equation}
with $x_0 \equiv (2\Omega_{\Lambda}/\Omega_m)^{1/3}$ and
$x = x_0 / (1+z)$.

The virial radius of a cluster is:
\begin{equation}
  \label{eq:radius-virial}
  R_{\R{vir}} = \left[
    \frac{3 M_{\R{vir}}}{4\pi \Delta_{\R{vir}}(z) \rho_{\R{crit}}(z)}
  \right]^{1/3},
\end{equation}
where $M_{\R{vir}}$ is the virial mass (can also be regarded as
the total mass) of the cluster,
$\rho_{\R{crit}}(z)$ is the critical density at redshift $z$,
and $\Delta_{\R{vir}}(z)$ is the average overdensity of the cluster
at redshift $z$ given by \citep{kitayama1996,cassano2005}:
\begin{equation}
  \label{eq:delta-vir}
  \Delta_{\R{vir}}(z) = 18\pi^2 \left[ 1 + 0.4093 \, w(z)^{0.9052} \right],
\end{equation}
where $w(z) \equiv \Omega_f^{-1}(z) - 1$.

The thermal number density of the cluster ICM is approximately given by:
\begin{equation}
  \label{eq:nth}
  n_{\R{th}} \simeq
    \frac{3 M_{\R{baryon}}}{4\pi \mu m_u R_{\R{vir}}^3},
\end{equation}
where $\mu \simeq 0.6$ is the mean molecular weight \citep{ettori2013},
$m_u$ is the atomic mass unit,
and $M_{\R{baryon}} \simeq M_{\R{vir}} f_b$ is the baryon mass
of the cluster with $f_b \simeq \Omega_b / \Omega_m$ the mean baryon
fraction assumed for clusters.

In a flat \lcdm{} cosmology as we adopted, the age of the Universe
at redshift $z$ takes an analytical form
\citep[their Equation~(18)]{thomas2000}:
\begin{align}
  \label{eq:universe-age}
  t(z; \Omega_m, \Omega_{\Lambda}=1-\Omega_m)
    & = \frac{1}{H_0} \int_z^{\infty}
      \frac{\D{z'}}{(1+z')\sqrt{1 + \Omega_m z' (3+3z'+z'^2)}} \nonumber \\
    & = \frac{2}{3 H_0 \sqrt{1-\Omega_m}} \sinh^{-1} \!\left(
      \sqrt{\frac{\Omega_m^{-1} - 1}{(1+z)^3}} \right).
\end{align}
This formula can also be inverted analytically to derive the redshift
$z$ corresponding to a cosmic time $t$ (i.e., age of the Universe),
which is used in our building of cluster merger trees
in Section~\ref{sec:merging-history}.

The \emph{angular diameter distance} $D_A$ is defined as the ratio of
an object's physical transverse size to its (observed) angular size
(in radians).  Note that it does \emph{not} increases indefinitely
as $z \to \infty$, therefore more distant (e.g., $z > 1$)
objects of same physical size may actually appear larger!
The angular diameter distance is used to convert the observed angular
separations between sources into their proper separations, and it is
related to the \emph{transverse comoving distance} $D_M$ by
\citep{weinberg1972,peebles1993,hogg1999}:
\begin{equation}
  \label{eq:da-dm}
  D_A(z) = \frac{D_M(z)}{1 + z}.
\end{equation}

The \emph{luminosity distance} $D_L$ is defined by the relationship
between the measured bolometric (i.e., integrated over all frequencies)
flux $S_{\R{bolo}}$ and the object's intrinsic bolometric luminosity
$L_{\R{bolo}}$:
\begin{equation}
  \label{eq:dl-def}
  D_L \equiv \sqrt{\frac{L_{\R{bolo}}}{4\pi S_{\R{bolo}}}}.
\end{equation}
And it is related to the transverse comoving distance and angular
diameter distance by \citep{weinberg1972,hogg1999,ellis2007}:
\begin{equation}
  \label{eq:dl-dm-da}
  D_L(z) = (1+z) D_M(z) = (1+z)^2 D_A(z).
\end{equation}

In the radio frequency regime, the Rayleigh-Jeans approximation
always holds, therefore the spectral brightness $I_{\nu}$ can be
equivalently expressed by \emph{brightness temperature} $T_b(\nu)$
through the relation \citep{condon2016}:
\begin{equation}
  \label{eq:brightness-temp}
  T_b(\nu) \equiv \frac{I_{\nu} c^2}{2 k_B \nu^2}.
\end{equation}


%%%%%%%%%%%%%%%%%%%%%%%%%%%%%%%%%%%%%%%%%%%%%%%%%%%%%%%%%%%%%%%%%%%%%%%%%
\bibliography{references}


%% Include this line if you are using the \added, \replaced, \deleted
%% commands to see a summary list of all changes at the end of the article.
%\listofchanges

\end{document}

%% EOF
